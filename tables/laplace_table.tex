% !TEX root = ../main.tex
\begin{center}
    \begin{tabular}{c c}
        $
            \Phi(x) = \frac{1}{\sqrt{2\pi}} 
            \int\limits_{0}^{x} e^{-\frac{t^2}{2}} dt
        $
        &
        \begin{tikzpicture}[baseline={(current bounding box.center)}, yscale=3, 
            scale = 1]
            \fill [lightgray, domain=0:1, smooth, variable = \x] plot ({\x}, 
            {
                (0.3989422804) * e^(- (\x * \x / 2))
            }) -- (1, 0) -- (0, 0) -- (0, 0.3989422804);
            \draw [->] (-3, 0) -- (3, 0);
            \draw [->] (0, -0.2) -- (0, 0.7);
            \draw [domain=-3:3, smooth, variable = \x, ultra thick] plot ({\x}, 
            {
                (0.3989422804) * e^(- (\x * \x / 2))
            });
            \node [below] at (1, 0) {$x$};
            \draw [dashed] (1, 0) -- (1, 0.25);
            \draw [->, thick] (1.5, 0.4) -- (0.7, 0.2);
            \node [right] at (1.5, 0.4) {площа дорівнює $\Phi(x)$};
            \draw [->] (-0.8, 0.5) -- (-0.495, 0.355);
            \node [left] at (-0.8, 0.5) {$\frac{1}{\sqrt{2\pi}}e^{-\frac{t^2}{2}}$};
        \end{tikzpicture}
    \end{tabular}
\end{center}
В таблиці наведено лише дробову частину усіх значень, оскільки усі цілі частини рівні 0:
наприклад, .48745 треба читати як 0.48745.
\begin{center}
    \small
    \noindent \begin{tabular}{|c|c|c|c|c|c|c|c|c|c|c|}
\hline
$\pmb{x}$ & \textbf{ 0.00 } & \textbf{ 0.01 } & \textbf{ 0.02 } & \textbf{ 0.03 } & \textbf{ 0.04 } & \textbf{ 0.05 } & \textbf{ 0.06 } & \textbf{ 0.07 } & \textbf{ 0.08 } & \textbf{ 0.09 } \\ 
 \hline 
\textbf{ 0.00 } & .00000 & .00399 & .00798 & .01197 & .01595 & .01994 & .02392 & .02790 & .03188 & .03586\\ 
\hline
\textbf{ 0.10 } & .03983 & .04380 & .04776 & .05172 & .05567 & .05962 & .06356 & .06749 & .07142 & .07535\\ 
\hline
\textbf{ 0.20 } & .07926 & .08317 & .08706 & .09095 & .09483 & .09871 & .10257 & .10642 & .11026 & .11409\\ 
\hline
\textbf{ 0.30 } & .11791 & .12172 & .12552 & .12930 & .13307 & .13683 & .14058 & .14431 & .14803 & .15173\\ 
\hline
\textbf{ 0.40 } & .15542 & .15910 & .16276 & .16640 & .17003 & .17364 & .17724 & .18082 & .18439 & .18793\\ 
\hline
\textbf{ 0.50 } & .19146 & .19497 & .19847 & .20194 & .20540 & .20884 & .21226 & .21566 & .21904 & .22240\\ 
\hline
\textbf{ 0.60 } & .22575 & .22907 & .23237 & .23565 & .23891 & .24215 & .24537 & .24857 & .25175 & .25490\\ 
\hline
\textbf{ 0.70 } & .25804 & .26115 & .26424 & .26730 & .27035 & .27337 & .27637 & .27935 & .28230 & .28524\\ 
\hline
\textbf{ 0.80 } & .28814 & .29103 & .29389 & .29673 & .29955 & .30234 & .30511 & .30785 & .31057 & .31327\\ 
\hline
\textbf{ 0.90 } & .31594 & .31859 & .32121 & .32381 & .32639 & .32894 & .33147 & .33398 & .33646 & .33891\\ 
\hline
\textbf{ 1.00 } & .34134 & .34375 & .34614 & .34849 & .35083 & .35314 & .35543 & .35769 & .35993 & .36214\\ 
\hline
\textbf{ 1.10 } & .36433 & .36650 & .36864 & .37076 & .37286 & .37493 & .37698 & .37900 & .38100 & .38298\\ 
\hline
\textbf{ 1.20 } & .38493 & .38686 & .38877 & .39065 & .39251 & .39435 & .39617 & .39796 & .39973 & .40147\\ 
\hline
\textbf{ 1.30 } & .40320 & .40490 & .40658 & .40824 & .40988 & .41149 & .41309 & .41466 & .41621 & .41774\\ 
\hline
\textbf{ 1.40 } & .41924 & .42073 & .42220 & .42364 & .42507 & .42647 & .42785 & .42922 & .43056 & .43189\\ 
\hline
\textbf{ 1.50 } & .43319 & .43448 & .43574 & .43699 & .43822 & .43943 & .44062 & .44179 & .44295 & .44408\\ 
\hline
\textbf{ 1.60 } & .44520 & .44630 & .44738 & .44845 & .44950 & .45053 & .45154 & .45254 & .45352 & .45449\\ 
\hline
\textbf{ 1.70 } & .45543 & .45637 & .45728 & .45818 & .45907 & .45994 & .46080 & .46164 & .46246 & .46327\\ 
\hline
\textbf{ 1.80 } & .46407 & .46485 & .46562 & .46638 & .46712 & .46784 & .46856 & .46926 & .46995 & .47062\\ 
\hline
\textbf{ 1.90 } & .47128 & .47193 & .47257 & .47320 & .47381 & .47441 & .47500 & .47558 & .47615 & .47670\\ 
\hline
\textbf{ 2.00 } & .47725 & .47778 & .47831 & .47882 & .47932 & .47982 & .48030 & .48077 & .48124 & .48169\\ 
\hline
\textbf{ 2.10 } & .48214 & .48257 & .48300 & .48341 & .48382 & .48422 & .48461 & .48500 & .48537 & .48574\\ 
\hline
\textbf{ 2.20 } & .48610 & .48645 & .48679 & .48713 & .48745 & .48778 & .48809 & .48840 & .48870 & .48899\\ 
\hline
\textbf{ 2.30 } & .48928 & .48956 & .48983 & .49010 & .49036 & .49061 & .49086 & .49111 & .49134 & .49158\\ 
\hline
\textbf{ 2.40 } & .49180 & .49202 & .49224 & .49245 & .49266 & .49286 & .49305 & .49324 & .49343 & .49361\\ 
\hline
\textbf{ 2.50 } & .49379 & .49396 & .49413 & .49430 & .49446 & .49461 & .49477 & .49492 & .49506 & .49520\\ 
\hline
\textbf{ 2.60 } & .49534 & .49547 & .49560 & .49573 & .49585 & .49598 & .49609 & .49621 & .49632 & .49643\\ 
\hline
\textbf{ 2.70 } & .49653 & .49664 & .49674 & .49683 & .49693 & .49702 & .49711 & .49720 & .49728 & .49736\\ 
\hline
\textbf{ 2.80 } & .49744 & .49752 & .49760 & .49767 & .49774 & .49781 & .49788 & .49795 & .49801 & .49807\\ 
\hline
\textbf{ 2.90 } & .49813 & .49819 & .49825 & .49831 & .49836 & .49841 & .49846 & .49851 & .49856 & .49861\\ 
\hline
\end{tabular} 

\noindent \begin{tabular}{|c|c|c|c|c|c|c|c|c|c|c|}
\hline
$\pmb{x}$ & \textbf{ 0.00 } & \textbf{ 0.01 } & \textbf{ 0.02 } & \textbf{ 0.03 } & \textbf{ 0.04 } & \textbf{ 0.05 } & \textbf{ 0.06 } & \textbf{ 0.07 } & \textbf{ 0.08 } & \textbf{ 0.09 } \\ 
 \hline 
\textbf{ 3.00 } & .49865 & .49869 & .49874 & .49878 & .49882 & .49886 & .49889 & .49893 & .49896 & .49900\\ 
\hline
\textbf{ 3.10 } & .49903 & .49906 & .49910 & .49913 & .49916 & .49918 & .49921 & .49924 & .49926 & .49929\\ 
\hline
\textbf{ 3.20 } & .49931 & .49934 & .49936 & .49938 & .49940 & .49942 & .49944 & .49946 & .49948 & .49950\\ 
\hline
\textbf{ 3.30 } & .49952 & .49953 & .49955 & .49957 & .49958 & .49960 & .49961 & .49962 & .49964 & .49965\\ 
\hline
\textbf{ 3.40 } & .49966 & .49968 & .49969 & .49970 & .49971 & .49972 & .49973 & .49974 & .49975 & .49976\\ 
\hline
\textbf{ 3.50 } & .49977 & .49978 & .49978 & .49979 & .49980 & .49981 & .49981 & .49982 & .49983 & .49983\\ 
\hline
\textbf{ 3.60 } & .49984 & .49985 & .49985 & .49986 & .49986 & .49987 & .49987 & .49988 & .49988 & .49989\\ 
\hline
\textbf{ 3.70 } & .49989 & .49990 & .49990 & .49990 & .49991 & .49991 & .49992 & .49992 & .49992 & .49992\\ 
\hline
\textbf{ 3.80 } & .49993 & .49993 & .49993 & .49994 & .49994 & .49994 & .49994 & .49995 & .49995 & .49995\\ 
\hline
\textbf{ 3.90 } & .49995 & .49995 & .49996 & .49996 & .49996 & .49996 & .49996 & .49996 & .49997 & .49997\\ 
\hline
\textbf{ 4.00 } & .49997 & .49997 & .49997 & .49997 & .49997 & .49997 & .49998 & .49998 & .49998 & .49998\\ 
\hline
\textbf{ 4.10 } & .49998 & .49998 & .49998 & .49998 & .49998 & .49998 & .49998 & .49998 & .49999 & .49999\\ 
\hline
\textbf{ 4.20 } & .49999 & .49999 & .49999 & .49999 & .49999 & .49999 & .49999 & .49999 & .49999 & .49999\\ 
\hline
\textbf{ 4.30 } & .49999 & .49999 & .49999 & .49999 & .49999 & .49999 & .49999 & .49999 & .49999 & .49999\\ 
\hline
\textbf{ 4.40 } & .49999 & .49999 & .50000 & .50000 & .50000 & .50000 & .50000 & .50000 & .50000 & .50000\\ 
\hline
\textbf{ 4.50 } & .50000 & .50000 & .50000 & .50000 & .50000 & .50000 & .50000 & .50000 & .50000 & .50000\\ 
\hline
\textbf{ 4.60 } & .50000 & .50000 & .50000 & .50000 & .50000 & .50000 & .50000 & .50000 & .50000 & .50000\\ 
\hline
\textbf{ 4.70 } & .50000 & .50000 & .50000 & .50000 & .50000 & .50000 & .50000 & .50000 & .50000 & .50000\\ 
\hline
\textbf{ 4.80 } & .50000 & .50000 & .50000 & .50000 & .50000 & .50000 & .50000 & .50000 & .50000 & .50000\\ 
\hline
\textbf{ 4.90 } & .50000 & .50000 & .50000 & .50000 & .50000 & .50000 & .50000 & .50000 & .50000 & .50000\\ 
\hline
\end{tabular}
\end{center}