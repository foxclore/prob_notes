% !TEX root = ../main.tex

Значення функції $t_{\alpha,n}$ визначаються з рівняння 
$ \int\limits_{t_{\alpha,n}}^{+\infty} f(x) dx = \alpha$
де $f(x)$ --- щільність розподілу $\mathrm{St}_n$ (Стьюдента з $n$ ступенями вільності).

\begin{center}
    \begin{tikzpicture}[yscale = 6, xscale = 1.3, baseline={(current bounding box.center)}]
        \pgfmathsetmacro{\b}{12}; % n = 4
        \pgfmathsetmacro{\p}{1.5};
    
        \fill [lightgray, domain=\p:5, smooth, variable = \x] plot ({\x}, 
        {
            \b/((4+(\x)^2)^((4+1)/2))
        }) -- (5, 0) -- (\p, 0) -- (\p, {\b/((4+(\p)^2)^((4+1)/2))});
        \draw [->] (-5, 0) -- (5, 0);
        \draw [->] (0, -0.05) -- (0, 0.41);
        \draw [domain=-5:5, smooth, variable = \x, ultra thick] plot ({\x}, {\b/((4+(\x)^2)^((4+1)/2))});
        \node [below] at (5.2, 0) {$x$};
        \node [left] at (0, 0.41) {$f(x)$};
        \node [below left] at (0, 0) {$0$};
        \node [below] at (\p, 0) {$t_{\alpha,n}$};
        \draw [dashed] (\p, 0) -- (\p, {\b/((4+(\p)^2)^((4+1)/2))});
        \draw [->, thick] (\p+1, 0.2) -- (\p+0.3, 0.03);
        \node [right] at (\p+1, 0.2) {площа дорівнює $\alpha$};
    \end{tikzpicture}
\end{center}
\begin{center}
    \begin{tabular}{c c}
        \noindent\begin{tabular}{*{5}{|c}|}
    \hline
    \diagbox{$\pmb{n}$}{$\pmb{\alpha}$} & \textbf{0.050} & \textbf{0.025} & \textbf{0.010} & \textbf{0.005} \\
    \hline 
\textbf{1} & 6.314 &12.706 &31.821 &63.657 \\
\hline 
\textbf{2} & 2.920 &4.303 &6.965 &9.925 \\
\hline 
\textbf{3} & 2.353 &3.182 &4.541 &5.841 \\
\hline 
\textbf{4} & 2.132 &2.776 &3.747 &4.604 \\
\hline 
\textbf{5} & 2.015 &2.571 &3.365 &4.032 \\
\hline 
\textbf{6} & 1.943 &2.447 &3.143 &3.707 \\
\hline 
\textbf{7} & 1.895 &2.365 &2.998 &3.499 \\
\hline 
\textbf{8} & 1.860 &2.306 &2.896 &3.355 \\
\hline 
\textbf{9} & 1.833 &2.262 &2.821 &3.250 \\
\hline 
\textbf{10} & 1.812 &2.228 &2.764 &3.169 \\
\hline 
\textbf{11} & 1.796 &2.201 &2.718 &3.106 \\
\hline 
\textbf{12} & 1.782 &2.179 &2.681 &3.055 \\
\hline 
\textbf{13} & 1.771 &2.160 &2.650 &3.012 \\
\hline 
\textbf{14} & 1.761 &2.145 &2.624 &2.977 \\
\hline 
\textbf{15} & 1.753 &2.131 &2.602 &2.947 \\
\hline 
\textbf{16} & 1.746 &2.120 &2.583 &2.921 \\
\hline 
\textbf{17} & 1.740 &2.110 &2.567 &2.898 \\
\hline 
\textbf{18} & 1.734 &2.101 &2.552 &2.878 \\
\hline 
\textbf{19} & 1.729 &2.093 &2.539 &2.861 \\
\hline 
\textbf{20} & 1.725 &2.086 &2.528 &2.845 \\
\hline 
\end{tabular} 

 &
        \noindent\begin{tabular}{*{9}{|c}|}
    \hline
    \diagbox{$\pmb{n}$}{$\pmb{\alpha}$} & \textbf{0.050} & \textbf{0.025} & \textbf{0.010} & \textbf{0.005} \\
    \hline 
\textbf{21} & 1.721 &2.080 &2.518 &2.831 \\
\hline 
\textbf{22} & 1.717 &2.074 &2.508 &2.819 \\
\hline 
\textbf{23} & 1.714 &2.069 &2.500 &2.807 \\
\hline 
\textbf{24} & 1.711 &2.064 &2.492 &2.797 \\
\hline 
\textbf{25} & 1.708 &2.060 &2.485 &2.787 \\
\hline 
\textbf{26} & 1.706 &2.056 &2.479 &2.779 \\
\hline 
\textbf{27} & 1.703 &2.052 &2.473 &2.771 \\
\hline 
\textbf{28} & 1.701 &2.048 &2.467 &2.763 \\
\hline 
\textbf{29} & 1.699 &2.045 &2.462 &2.756 \\
\hline 
\textbf{30} & 1.697 &2.042 &2.457 &2.750 \\
\hline 
\textbf{40} & 1.684 &2.021 &2.423 &2.704 \\
\hline 
\textbf{50} & 1.676 &2.009 &2.403 &2.678 \\
\hline 
\textbf{60} & 1.671 &2.000 &2.390 &2.660 \\
\hline 
\textbf{70} & 1.667 &1.994 &2.381 &2.648 \\
\hline 
\textbf{70} & 1.667 &1.994 &2.381 &2.648 \\
\hline 
\textbf{90} & 1.662 &1.987 &2.368 &2.632 \\
\hline 
\textbf{100} & 1.660 &1.984 &2.364 &2.626 \\
\hline 
\textbf{120} & 1.658 &1.980 &2.358 &2.617 \\
\hline 
\textbf{150} & 1.655 &1.976 &2.351 &2.609 \\
\hline 
$\pmb{\infty}$& 1.645 &1.960 &2.326 &2.576 \\
\hline 
\end{tabular}
    \end{tabular}
\end{center}