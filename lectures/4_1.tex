% !TEX root = ../main.tex
\section{Характеристичні функції випадкових величин}
\subsection{Поняття характеристичної функції}
\begin{definition}
    \emph{Характеристичною функцією} випадкової величини $\xi$
    називається комплекснозначна функція $\chi_\xi(t) = Ee^{it\xi}, t\in \mathbb{R}$.
    \begin{equation}\label{eq:char_func}
        \chi_\xi(t) = \int\limits_{-\infty}^{+\infty} e^{itx} dF_\xi(x) = \begin{cases}
            \sum\limits_{k=1}^{n(\infty)} e^{itx_k} P\left\{\xi = x_k\right\}, & \xi \text{ --- ДВВ} \\
            \int\limits_{-\infty}^{+\infty} e^{itx} f_\xi(x)dx, & \xi \text{ --- НВВ}
        \end{cases}
    \end{equation}
\end{definition}
Інтеграл $\int\limits_{-\infty}^{+\infty} e^{itx} f_\xi(x)dx$ у курсі
гармонічного аналізу називається \emph{перетворенням Фур'є} функції $f_\xi(x)$.
Отже, у випадку НВВ $\xi$ $\chi_\xi(t)$ --- перетворення Фур'є щільності.
Також з курсу гармонічного аналізу відомо, що за допомогою \emph{оберненого перетворення Фур'є}
можна відновити $f_\xi(x)$ за $\chi_\xi(t)$: $f_\xi(x) = \frac{1}{2\pi}\int\limits_{-\infty}^{+\infty} e^{-itx} \chi_\xi(t)dt$.

\subsection{Властивості характеристичної функції}
\begin{enumerate}
    \item $\chi_\xi(0) = 1$, оскільки $\int\limits_{-\infty}^{+\infty} f_\xi(x)dx = 1$.
    
    $\left|\chi_\xi(t)\right| \leq 1$, оскільки $\left|\int\limits_{-\infty}^{+\infty} e^{itx} dF_\xi(x)\right| \leq \int\limits_{-\infty}^{+\infty} \left|e^{itx}\right| dF_\xi(x) = \int\limits_{-\infty}^{+\infty} dF_\xi(x) = 1$.
    \item Характеристична функція є рівномірно неперервною.
    \begin{proof}
        $\left| \chi_\xi(t+h) - \chi_\xi(t) \right| = 
        \left| \int\limits_{-\infty}^{+\infty} \left( e^{i(t+h)x} - e^{itx}\right) dF_\xi(x) \right| \leq \\
        \leq \int\limits_{-\infty}^{+\infty} \left|e^{i(t+h)x} - e^{itx}\right| dF_\xi(x) = 
        \int\limits_{-\infty}^{+\infty} \left|e^{itx}\right|\left| e^{ihx} - 1\right| dF_\xi(x) = 
        \int\limits_{-\infty}^{+\infty} \left|e^{ihx} - 1\right| dF_\xi(x)$.

        $e^{ihx} - 1 = (\cos(hx) - 1) + i\sin(hx) = -2\sin^2(\frac{hx}{2}) + 2\sin(\frac{hx}{2})\cos(\frac{hx}{2})$, 
        
        $\left| e^{ihx} - 1\right| = 
        \sqrt{4\sin^4(\frac{hx}{2}) + 4\sin^2(\frac{hx}{2})\cos^2(\frac{hx}{2})} = 2\left| \sin(\frac{hx}{2})\right|$.

        $\int\limits_{-\infty}^{+\infty} \left|e^{ihx} - 1\right| dF_\xi(x) = 2 \int\limits_{-\infty}^{+\infty} \left| \sin(\frac{hx}{2})\right| dF_\xi(x) =
        \\\overset{A>0}{=} 2\cdot \left( \int\limits_{-\infty}^{-A}\left| \sin(\frac{hx}{2})\right| dF_\xi(x) + 
        \int\limits_{-A}^{A}\left| \sin(\frac{hx}{2})\right| dF_\xi(x) + 
        \int\limits_{A}^{+\infty}\left| \sin(\frac{hx}{2})\right| dF_\xi(x)\right)
        $.

        Інтеграл $\int\limits_{-A}^{A}\left| \sin(\frac{hx}{2})\right| dF_\xi(x)$
        можна зробити як завгодно малим за рахунок вибору $h$,
        а інші два --- за рахунок вибору $A$.
    \end{proof}
    \item $\eta = a\xi + b$ --- афінне перетворення $\xi$, $a, b \in \mathbb{R}$.

    $\chi_\eta(t) = Ee^{it\eta} = Ee^{it(a\xi + b)} = e^{itb}\cdot Ee^{ita\xi} = e^{itb}\cdot \chi_\xi(at)$.
    \item Якщо $\xi_1, ..., \xi_n$ --- незалежні у сукупності, то
    $\chi_{\sum_{k=1}^n {\xi_k}} (t) = \prod\limits_{k=1}^n \chi_{\xi_k}(t)$.
    \begin{proof}
        $\chi_{\sum_{k=1}^n \xi_k}(t) = E\left( e^{it\sum_{k=1}^n \xi_k}\right) =
        E\left( \prod\limits_{k=1}^n e^{it\xi_k}\right) = \\
        = \left[e^{it\xi_k} \text{ --- теж незалежні у сукупності}\right] = 
        \prod\limits_{k=1}^n Ee^{it\xi_k} = \prod\limits_{k=1}^n \chi_{\xi_k}(t)$.
    \end{proof}
    \item За допомогою характеристичної функції можна знайти початкові моменти будь-якого порядку: $\chi_\xi^{(k)}(t) = \int\limits_{-\infty}^{+\infty} (ix)^k e^{itx} dF_\xi(x)$,
    причому $\chi_\xi^{(k)}(0) = \int\limits_{-\infty}^{+\infty} (ix)^k dF_\xi(x) = i^k \cdot E\xi^k$.
    
    Отже, $E\xi = -i \chi_\xi'(0)$, $D\xi = E\xi^2 - \left( E\xi\right)^2 = - \chi_\xi''(0) + \left( \chi_\xi'(0) \right)^2$.
    \item Зв'язок характеристичної функції та генератриси ДВВ.

    $\varphi_\xi(z) = \sum\limits_{k=0}^{\infty} P\left\{\xi = k\right\} z^k$,
    $\chi_\xi(t) = \sum\limits_{k=0}^{\infty} e^{itk} P\left\{\xi = x_k\right\}$. Отже, $\chi_\xi(t) = \varphi_\xi(e^{it})$.
    \item $\overline{\chi_\xi(t)} = E e^{-it\xi} = \chi_\xi(-t) = \chi_{-\xi} (t)$. 
    \item Для того, щоб характеристична функція була дійснозначною,
    необхідно і достатньо, щоб розподіл ВВ був симетричним відносно 0.
    \begin{proof}
        Розглянемо випадок НВВ. Нехай розподіл є симетричним відносно 0, тоді $f_\xi(x)$ --- парна.
        $\chi_\xi(t) = \int\limits_{-\infty}^{+\infty} e^{itx} f_\xi(x)dx = 
        \int\limits_{-\infty}^{+\infty} \cos(tx) f_\xi(x)dx +
        i\int\limits_{-\infty}^{+\infty} \sin(tx) f_\xi(x)dx = 
        \int\limits_{-\infty}^{+\infty} \cos(tx) f_\xi(x)dx = {Re}\chi_\xi(t)$, 
        інтеграл з синусом рівний 0, бо інтегрується непарна функція по симетричному проміжку.

        Нехай $\chi_\xi(t)$ --- дійснозначна, тоді $\chi_\xi(t) = \overline{\chi_\xi(t)} = \chi_\xi(-t)$.
        Отже, $\chi_\xi(t)$ --- парна, тоді з оберненого перетворення Фур'є $f_\xi(t)$ --- теж парна.
    \end{proof}
\end{enumerate}

\subsection{Необхідні умови того, що функція є характеристичною}
Нехай $\chi(t)$ --- деяка комплекснозначна функція дійсного аргументу.
Якщо вона є характеристичною функцією деякої ВВ, то для неї має виконуватися:
\begin{enumerate}
    \item $\chi(0) = 1$, $\left| \chi(t)\right| \leq 1$.
    \item $\chi(t)$ --- рівномірно неперервна.
    \item $\overline{\chi(t)} = \chi(-t)$.
\end{enumerate}
\begin{example}
    \begin{enumerate}
        \item $\chi(t) = \cos(t)$ --- може бути характеристичною.
        Оскільки $\cos(t) = \frac{1}{2}e^{it} + \frac{1}{2}e^{-it}$, то відповідна ВВ --- дискретна: 
        \begin{tabular}{|c|c|c|}
            \hline
            $\xi$ & $-1$ & $1$ \\
            \hline
            $p$ & $1/2$ & $1/2$ \\
            \hline
        \end{tabular}
        \item $\chi(t) = \sin(t)$ --- не може бути характеристичною, бо $\chi(0) = 0 \neq 1$.
        \item $\chi(t) = \cos^2(t)$ --- може бути характеристичною.
        $\cos(t) = \frac{1}{2} + \frac{1}{2}\cos(2t) = \frac{1}{2} + \frac{1}{4}e^{2it} + \frac{1}{4}e^{-2it}$, тому
        відповідна ВВ --- дискретна:
        \begin{tabular}{|c|c|c|c|}
            \hline
            $\xi$ & $-1$ & $0$ & $1$ \\
            \hline
            $p$ & $1/4$ & $1/2$ & $1/4$ \\
            \hline
        \end{tabular}
        \item $\chi(t) = \frac{\alpha^2}{\alpha^2 + t^2}$ --- може бути характеристичною. 
        Знайдемо щільність розподілу відповідної НВВ за формулою $f_\xi(x) = \frac{1}{2\pi}\int\limits_{-\infty}^{+\infty} \frac{\alpha^2 \cdot e^{-itx}}{\alpha^2 + t^2} dt$.

        При $x<0$: $\int\limits_{-\infty}^{+\infty} \frac{e^{-itx}}{\alpha^2 + t^2} dt = 2\pi i \cdot \underset{z=\alpha i}{\mathrm{Res}} \frac{e^{-izx}}{\alpha^2 + z^2} = 2\pi i \cdot \frac{e^{-i\cdot \alpha i\cdot x}}{2\alpha i} = \pi \cdot \frac{e^{\alpha x}}{\alpha}$.

        При $x>0$ аналогічно отримуємо $\pi \cdot \frac{e^{-\alpha x}}{\alpha}$.
        Остаточно $f_\xi(x) = \frac{\alpha}{2}e^{-\alpha |x|}$ --- це щільність розподілу Лапласа.
    \end{enumerate}
\end{example}

\begin{exercise}
    Нехай $\chi(t)$ --- характеристична функція деякого розподілу. Чи можуть бути характеристичними функції $\overline{\chi}$, $\chi^2$, $|\chi|^2$, ${Re}\chi$, $|\chi|$, ${Im}\chi$? 
\end{exercise}

Необхідні та достатні умови того, що функція є характеристичною, дає
\emph{теорема Бохнера-Хінчина}: $\chi(t)$ має бути рівномірно неперервною, $\chi(0) = 1$ та 
невід'ємно визначеною: $$
    \forall \; t_1, t_2, ..., t_n \in \mathbb{R}, c_1, c_2, ..., c_n \in \mathbb{C}: 
    \sum\limits_{k=1}^n {\sum\limits_{m=1}^n \chi(t_k - t_m) c_k \overline{c_m}} \geq 0
$$

\subsection{Характеристичні функції деяких розподілів}
\begin{enumerate}
    \item $\xi \sim \mathrm{Bin}(n,p)$ --- біноміальний розподіл.
    $\varphi_\xi(z) = (pz+q)^n \Rightarrow \chi_\xi(t) = \left( p e^{it} + q\right)^n$.
    \item $\xi \sim \mathrm{Geom}(p)$ --- геометричний розподіл.
    $\varphi_\xi(z) = \frac{pz}{1-qz} \Rightarrow \chi_\xi(t) = \frac{p e^{it}}{1-q e^{it}}$.
    \item $\xi \sim \mathrm{Poiss}(a)$ --- розподіл Пуассона.
    $\varphi_\xi(z) = e^{a(z-1)} \Rightarrow \chi_\xi(t) = e^{a(e^{it}-1)}$.
    \item $\xi \sim \mathrm{U}\left< a; b\right>$ --- рівномірний розподіл.

    $\chi_\xi(t) = \frac{1}{b-a} \int\limits_a^b e^{itx} dx = \frac{e^{itb} - e^{ita}}{it(b-a)}$.
    Зокрема, при $\xi \sim {U}\left< -a; a\right>$ $\chi_\xi(t) = \begin{cases}
        \frac{\sin(at)}{at}, & t \neq 0 \\
        1, & t = 0
    \end{cases}$.
    \item $\xi \sim \mathrm{Exp}(\lambda)$ --- експоненційний розподіл.
    
    $\chi_\xi(t) = \lambda \int\limits_0^{+\infty} e^{itx} e^{-\lambda x} dx = \lambda \int\limits_0^{+\infty} e^{-(\lambda-it)x} dx = \frac{\lambda}{\lambda - it}$.
    \item $\xi \sim \mathrm{N}(a, \sigma^2)$ --- нормальний розподіл. Розглянемо стандартний розподіл $\eta \sim \mathrm{N}(0, 1)$,
    тоді $\xi = a + \sigma\cdot\eta$.

    $\chi_\eta(t) = \frac{1}{\sqrt{2\pi}} \int\limits_{-\infty}^{+\infty} e^{itx} e^{-x^2/2} dx = 
    \frac{1}{\sqrt{2\pi}} \int\limits_{-\infty}^{+\infty} e^{-x^2/2 + itx} dx = 
    \frac{1}{\sqrt{2\pi}} \int\limits_{-\infty}^{+\infty} e^{-\frac{1}{2}(x^2 - 2itx + i^2t^2) + \frac{1}{2}i^2t^2} dx = 
    \frac{1}{\sqrt{2\pi}} e^{-\frac{t^2}{2}} \int\limits_{-\infty}^{+\infty} e^{-\frac{(x-it)^2}{2}} dx = 
    \left[ x+it = u, dx = du\right] = e^{-\frac{t^2}{2}}$.
    Отже, $\chi_\xi(t) = e^{iat - \frac{\sigma^2 t^2}{2}}$.
    \item $\xi \sim \Gamma(\alpha, \beta)$ --- гамма-розподіл.
    
    $\chi_\xi(t) = \frac{\beta^\alpha}{\Gamma(\alpha)} \int\limits_0^{+\infty} e^{itx} x^{\alpha-1} e^{-\beta x} dx =
    \frac{\beta^\alpha}{\Gamma(\alpha)} \int\limits_0^{+\infty} x^{\alpha-1} e^{-(\beta-it) x} dx = \left[ (\beta-it) x = u, dx = \frac{du}{\beta-it}\right] = 
    \frac{\beta^\alpha}{\Gamma(\alpha)} \cdot \frac{1}{(\beta-it)^\alpha} \int\limits_0^{+\infty} u^{\alpha-1} e^{-u} du = 
    \frac{\beta^\alpha}{\Gamma(\alpha)} \cdot \frac{1}{(\beta-it)^\alpha} \cdot \Gamma(\alpha) = (1-\frac{it}{\beta})^{-\alpha}$.
\end{enumerate}

\begin{definition}
    \emph{Композицією законів розподілу} називається складання закону розподілу суми незалежних випадкових величин.
    
    Закон розподілу називається \emph{стійким по відношенню до операції додавання},
    якщо закон розподілу суми випадкових величин, розподілених за цим законом (в загальному випадку з різними параметрами),
    є таким самим.
\end{definition}
\begin{example}
    \begin{enumerate}
        \item Перевірити стійкість нормального розподілу. Нехай $\xi_1 \sim \mathrm{N}(a_1, \sigma_1), ..., \xi_n \sim \mathrm{N}(a_n, \sigma_n)$ ---
        незалежні у сукупності. 
    
        Знайдемо характеристичну функцію суми: $\chi_{\xi_1 + ... + \xi_n}(t) = \prod\limits_{k=1}^n \chi_{\xi_k}(t)=
        \prod\limits_{k=1}^n e^{ia_k t - \frac{\sigma_k^2 t^2}{2}} = 
        \exp\left\{i\cdot \left( \sum\limits_{k=1}^n a_k\right)\cdot t - \frac{t^2}{2}\cdot\left( \sum\limits_{k=1}^n \sigma_k^2\right)\right\}$ ---
        це характеристична функція нормального розподілу. Отже, $\sum\limits_{k=1}^n = \xi_1 + ... + \xi_n \sim \mathrm{N}\left(\sum\limits_{k=1}^n a_k,  \sqrt{\sum\limits_{k=1}^n \sigma_k^2}\right)$.
        \item Перевірити стійкість біноміального розподілу. Нехай $\xi_1 \sim \mathrm{Bin}(n_1, p_1)$, $\xi_2 \sim \mathrm{Bin}(n_2, p_2)$ --- незалежні.
        $\chi_{\xi_1 + \xi_2} (t) = \left( p_1 e^{it} + q_1\right)^{n_1} \cdot \left( p_2 e^{it} + q_2\right)^{n_2} \overset{?}{=} \left(P e^{it} + Q\right)^N$.

        Взагалі кажучи, біноміальний розподіл не є стійким по відношенню до  операції додавання.
        Його називають \emph{умовно стійким} при $p_1 = p_2$, тоді
        $\chi_{\xi_1 + \xi_2} (t) = \left( p e^{it} + q\right)^{n_1 + n_2}$ і $\xi_1 + \xi_2 \sim \mathrm{Bin}(n_1+n_2, p)$.
    \end{enumerate}
\end{example}

\begin{exercise}
    Перевірити стійкість гамма-розподілу $\Gamma(\alpha, \beta)$.
\end{exercise}

\section{Характеристичні функції випадкових векторів}
\subsection{Поняття характеристичної функції випадкового вектору}
\begin{definition}
    \emph{Характеристичною функцією} випадкового вектору $\vec{\xi} = \left( \xi_1, \xi_2, ..., \xi_n\right)^T$ 
    називається комплекснозначна функція $n$ дійсних аргументів $\chi_{\vec{\xi}}(\vec{t}) = Ee^{i(\vec{\xi}, \vec{t})}$, $\vec{t} \in \mathbb{R}^n$.
    \begin{equation}\label{eq:char_func_vect}
        \chi_{\vec{\xi}}(\vec{t}) = \int\limits_{\mathbb{R}^n} e^{i(\vec{x}, \vec{t})} dF_\xi(x) = \begin{cases}
            \sum\limits_{k_1,k_2, ..., k_n} e^{i \sum\limits_{j=1}^n x_{k_j}t_j} P\left\{\xi_1 = x_{k_1}, ..., \xi_n = x_{k_n}\right\}, & \xi \text{ --- ДВВ} \\
            \int\limits_{\mathbb{R}^n} e^{i(\vec{x}, \vec{t})} f_{\vec{\xi}}(x_1, x_2, ..., x_n)dx_1 dx_2 ... dx_n, & \xi \text{ --- НВВ}
        \end{cases}
    \end{equation}
\end{definition}

\begin{example} Обчислення характеристичної функції для дискретного та неперервного випадкових векторів
    \begin{enumerate}
        \item \begin{tabular}{c c}
            \begin{tabular}{|c|c|c|}
                \hline
                \diagbox{$\xi_2$}{$\xi_1$} & $1$ & $2$ \\
                \hline
                $0$ & $0.1$ & $0.3$ \\
                \hline
                $1$ & $0.2$ & $0.4$ \\
                \hline
            \end{tabular} 
        \end{tabular}

            $\chi_{\vec{\xi}}(t_1, t_2) = 0.1 e^{i(1\cdot t_1 + 0\cdot t_2)} +
            0.3 e^{i(2\cdot t_1 + 0\cdot t_2)} + 0.2 e^{i(1\cdot t_1 + 2\cdot t_2)} +
            0.4 e^{i(2\cdot t_1 + 0\cdot t_2)}$.
        \item $\vec{\xi} \sim {U}(D)$, $D = \left\{(x;y) \in \mathbb{R}^2:|x|\leq 1, |y|\leq 1\right\}$.
        
        $\chi_{\vec{\xi}}(t_1, t_2) = \frac{1}{4} \iint\limits_{D} e^{i(t_1x + t_2y)} dx dy = \frac{1}{4}\cdot\left( \int\limits_{-1}^1 e^{it_1x} dx\right) 
        \cdot\left( \int\limits_{-1}^1 e^{it_2y} dy\right) = \frac{e^{it_1}-e^{-it_1}}{2it_1}\cdot\frac{e^{it_2}-e^{-it_2}}{2it_2} = \begin{cases}
            \frac{\sin(t_1)}{t_1} \cdot \frac{\sin(t_2)}{t_2}, & t_1 \neq 0, t_2 \neq 0 \\
            \frac{\sin(t_1)}{t_1}, & t_2 = 0 \\
            \frac{\sin(t_2)}{t_2}, & t_1 = 0 \\
            1, & t_1 = t_2 = 0
        \end{cases}$
    \end{enumerate}
\end{example}

\subsection{Властивості характеристичних функцій векторів}
\begin{enumerate}
    \item Характеристична функція існує для будь-якого розподілу випадкового вектора, бо \\
    $\left| \chi_{\vec{\xi}}(\vec{t})\right| \leq 1$, $\chi_{\vec{\xi}}(\vec{0}) = 1$.
    \item Характеристична функція рівномірно неперервна по кожному з аргументів.
    \item За характеристичною функцією вектора можна знайти 
    характеристичну функцію будь-якої підсистеми цього вектора.
    $\vec{\xi} = (\xi_1, \xi_2, ..., \xi_n)$, $\vec{\eta} = (\xi_1, \xi_2, ..., \xi_k)$, $k<n$.

    $\chi_{\vec{\eta}}(t_1, t_2, ..., t_k) = \chi_{\vec{\xi}}(t_1, t_2, ..., t_k, 0, ..., 0)$.
    \item Якщо координати $\vec{\xi}$ незалежні у сукупності, то $\chi_{\vec{\xi}}(\vec{t}) = \prod\limits_{k=1}^n \chi_{\xi_k}(t_k)$.
    \begin{proof}
        $\chi_{\vec{\xi}}(\vec{t}) = E\left( e^{i(\xi_1 t_1 + ... + \xi_n t_n)}\right) = 
        E\left( e^{i\xi_1 t_1} ... e^{i\xi_n t_n}\right) = \left[ \text{незалежні у сукупності}\right] =
        E\left( e^{i\xi_1 t_1}\right) \cdot ... \cdot E\left( e^{i\xi_n t_n}\right) = \prod\limits_{k=1}^n \chi_{\xi_k}(t_k)$.
    \end{proof}
    \item $\vec{\eta} = A\vec{\xi} + \vec{b}$ --- афінне перетворення $\vec{\xi}$.

    $\chi_{\vec{\eta}}(\vec{t}) = E e^{i(\vec{\eta}, \vec{t})} = E e^{i(A\vec{\xi} + \vec{b}, \vec{t})}=
    E e^{i(\vec{b}, \vec{t})} \cdot E e^{(A\vec{\xi}, \vec{t})} = e^{i(\vec{b}, \vec{t})} \cdot E e^{(\vec{\xi}, A^{*}\vec{t})} = e^{i(\vec{b}, \vec{t})} \cdot \chi_{\vec{\xi}}(A^{*}\vec{t})$.
    \item Характеристична функція вектора дозволяє знайти початкові моменти будь-якого порядку:
    \begin{gather*}
        E\xi_1^{k_1} \xi_2^{k_2} ... \xi_n^{k_n} = \frac{1}{i^{k_1 + k_2 + ... + k_n}} \cdot \frac{\partial^{k_1 + k_2 + ... + k_n}\chi_{\vec{\xi}}(t)}{\partial t_1^{k_1} \partial t_2^{k_2} ... \partial t_n^{k_n}} \Biggr \vert_{\vec{t} = \vec{0}}, 
        \; E\xi_j = \frac{1}{i} \cdot\frac{\partial \chi_{\vec{\xi}}(t)}{\partial t_j} \Biggr \vert_{\vec{t} = \vec{0}} \\
        {cov}(\xi_k, \xi_j) = E\xi_k\xi_j - E\xi_k\cdot E\xi_j = \frac{1}{i^2} \frac{\partial^2 \chi_{\vec{\xi}}(t)}{\partial t_k \partial t_j} \Biggr \vert_{\vec{t} = \vec{0}} - 
        \left( \frac{1}{i} \cdot\frac{\partial \chi_{\vec{\xi}}(t)}{\partial t_k} \Biggr \vert_{\vec{t} = \vec{0}}\right) \cdot \left( \frac{1}{i} \cdot\frac{\partial \chi_{\vec{\xi}}(t)}{\partial t_j} \Biggr \vert_{\vec{t} = \vec{0}}\right) = \\
        -\frac{\partial^2 \chi_{\vec{\xi}}(t)}{\partial t_k \partial t_j} \Biggr \vert_{\vec{t} = \vec{0}} + 
        \left(\frac{\partial \chi_{\vec{\xi}}(t)}{\partial t_k} \Biggr \vert_{\vec{t} = \vec{0}}\right) \cdot \left(\frac{\partial \chi_{\vec{\xi}}(t)}{\partial t_j} \Biggr \vert_{\vec{t} = \vec{0}}\right)\end{gather*}
\end{enumerate}

Необхідні та достатні умови того, що комплекснозначна функція $n$ дійсних змінних є характеристичною, дає
\emph{теорема Бохнера-Хінчина}: $\chi(\vec{t})$ має бути рівномірно неперервною по кожній змінній, $\chi(\vec{0}) = 1$ та 
невід'ємно визначеною: $$
    \forall \; \vec{t^1}, \vec{t^2}, ..., \vec{t^n} \in \mathbb{R}^n, c_1, c_2, ..., c_n \in \mathbb{C}: 
    \sum\limits_{k=1}^n {\sum\limits_{m=1}^n \chi(\vec{t^k} - \vec{t^m}) c_k \overline{c_m}} \geq 0$$