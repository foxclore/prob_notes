% !TEX root = ../main.tex

\section{Поняття випадкової величини та її задання}
\begin{definition}
    \emph{Випадковою величиною} називається дійснозначна вимірна функція, що здійснює 
    відображення з $\Omega$ в $\mathbb{R}$, тобто $\xi = \xi(\omega): \Omega 
    \rightarrow \mathbb{R}$.
\end{definition}
\begin{remark}
    Вимірність функції $\xi(\omega)$ означає, що 
    \begin{equation}
        \forall x \in \mathbb{R}: 
        \left\{ \omega \in \Omega\; |\; \xi(\omega) < x\right\} \in \mathcal{F}
    \end{equation} В курсі 
    функціонального аналізу доводиться, що якщо виконується (1), то
    \begin{equation}
        \left.\begin{aligned}
            \{\omega\;|&\;\xi(\omega) > x\}\\
            \{\omega\;|&\;\xi(\omega) \leq x\}\\
            \{\omega\;|&\;\xi(\omega) \geq x\}\\
            \{\omega\;|&\;a < \xi(\omega) < b\}
        \end{aligned}\right\} \in \mathcal{F}
    \end{equation}
\end{remark}

Всього існують два види випадкових величин --- дискретні та неперервні.

\subsection{Дискретні випадкові величини та їх задання}
\begin{definition}
    Випадкова величина $\xi = \xi(\omega)$ називається 
    \emph{дискретною випадковою величиною} (ДВВ), якщо вона набуває скінченну або зліченну 
    кількість значень.
\end{definition}
Для задання ДВВ крім знання значень випадкової величини необхідно знати ймовірності, 
з якими ці значення приймаються.
$$p_i = P\left\{\omega: \xi(\omega) = x_i\right\} = P(\xi = x_i)$$
$$\sum_{i=1}^\infty p_i = 1$$
\begin{definition}
    \emph{Законом розподілу (імовірностей)} ДВВ називається співвідношення, яке вказує, 
    які ця випадкова величина приймає значення і з якими імовірностями.

    Закон розподілу ДВВ записується у вигляді \emph{ряду розподілу}:

    \begin{tabular}{c|c|c|c|c|c}
        $\xi$ & $x_1$ & $x_2$ & ... & $x_n$ & ... \\
        \hline
        $p$ & $p_1$ & $p_2$ & ... & $p_n$ & ...
    \end{tabular}
    \hspace{40pt}
    $x_1 < x_2 < ... < x_n < ...,\;\; \sum\limits_i p_i = 1$
\end{definition}
\begin{example}
    $\xi$ задає кількість влучень при чотирьох пострілах з імовірністю влучення 
    $p = \frac{1}{2}$. Скласти закон розподілу цієї випадкової величини.

    $P(\xi = k) = C_4^k \left(\frac{1}{2}\right)^k \left(\frac{1}{2}\right)^{4-k} = 
    C_4^k \left(\frac{1}{2}\right)^4 = \frac{C_4^k}{2^4}$

    \begin{tabular}{c|c|c|c|c|c}
        $\xi$ & 0 & 1 & 2 & 3 & 4 \\
        \hline
        $p$ & $\frac{1}{2^4}$ & $\frac{4}{2^4}$ & $\frac{6}{2^4}$ & $\frac{5}{2^4}$ & 
        $\frac{1}{2^4}$
    \end{tabular}
\end{example}
\newpage
\begin{definition}
    Графічне зображення закону розподілу ДВВ називається 
    \emph{полігоном розподілу імовірностей}.
\end{definition}

\hbox to \hsize{\hfil{
    \begin{tikzpicture}[yscale = 0.8]
        %\node [below] at (2, 0) {$np-q$};
        %\node [below] at (6, 0) {$np+p$};
        %\draw [fill] (4, 0) circle [radius = 0.08];
        %\draw [fill] (2, 0) circle [radius = 0.05];
        %\draw [fill] (6, 0) circle [radius = 0.05];
        %\node [below] at (4, 0) {$m_0$};
        %\draw [-{Straight Barb}] [thick] (0, 0) -- (8, 0);
        %\draw [thick] (2, 0) to (2, 0.5);
        %\draw [thick] (6, 0) to (6, 0.5);
        %\draw [{Straight Barb}-{Straight Barb}] [thick] (2, 0.5) to (6, 0.5);
        %\node [above] at (4, 0.5) {1};
        \draw [->] (0,0) -- (5,0);
        \draw [->] (0,0) -- (0,6);
        \draw [fill] (1, 0) circle [radius = 0.05];
        \node [below] at (1, 0) {$x_1$};
        \node [below] at (2, 0) {$x_2$};
        \node [below] at (3, 0) {$x_3$};
        \node [below] at (3.75, 0) {$...$};
        \node [below] at (4.5, 0) {$x_n$};
        \node [left] at (0, 1) {$p_1$};
        \node [left] at (0, 2) {$p_2$};
        \node [left] at (0, 3) {$p_3$};
        \node [left] at (0, 3.75) {$...$};
        \node [left] at (0, 4.5) {$p_n$};
        \draw [fill] (2, 0) circle [radius = 0.05];
        \draw [fill] (3, 0) circle [radius = 0.05];
        \draw [fill] (4.5, 0) circle [radius = 0.05];
        \draw [fill] (0, 1) circle [radius = 0.05];
        \draw [fill] (0, 2) circle [radius = 0.05];
        \draw [fill] (0, 3) circle [radius = 0.05];
        \draw [fill] (0, 4.5) circle [radius = 0.05];
        \draw [fill] (0, 5.5) circle [radius = 0.05];
        \node [left] at (0, 5.5) {$1$};
        \node [right] at (5, 0) {$x$};
        \node [above] at (0, 6) {$p$};
        \draw [dashed] (0, 5.5) -- (5, 5.5);
        \draw [fill] (1, 1) circle [radius = 0.05];
        \draw [fill] (2, 3) circle [radius = 0.05];
        \draw [fill] (3, 2) circle [radius = 0.05];
        \draw [fill] (4.5, 4.5) circle [radius = 0.05];
        \draw (1, 1) -- (2, 3) -- (3, 2) -- (4.5, 4.5);
    \end{tikzpicture}
}\hfil}