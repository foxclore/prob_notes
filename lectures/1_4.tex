% !TEX root = ../main.tex

\section{Поліноміальна модель ймовірності. Схема Бернуллі.}
Припустимо, що події $A_1, A_2, \dots, A_k$ утворюють повну групу подій деякого 
СЕ, причому відомі ймовірності $$P(A_i) = p_i,\;\;\;\; \sum_{i=1}^k p_i = 1$$
Експеримент проводиться n разів, в кожному з яких може відбутись одна з 
подій $A_i$. 
\begin{definition}
    Випробування, що проводяться, називаються \emph{незалежними}, якщо 
    $$P(A_{i_1}^{(1)} \cap A_{i_2}^{(2)} \cap \dots \cap A_{i_n}^{(n)}) = 
    P(A_{i_1}^{(1)}) P(A_{i_2}^{(2)}) \dots P(A_{i_n}^{(n)}),\;\;\;
    i=\overline{1,k}$$
\end{definition}

Основна задача - знайти ймовірність того, що подія $A_1$ відбудеться $m_1$ разів, 
подія $A_2$ відбудеться $m_2$ разів і т.д, подія $A_k$ відбудеться $m_k$ разів, при 
чому $\sum_{j=1}^k m_j = n$. Такі ймовірності будемо позначати як $P_n(m_1, m_2, \dots, m_n)$.
$$P(\underbrace{A_1 \cap \dots \cap A_1}_{m_1 \text{разів}} 
\cap \underbrace{ A_2 \cap \dots \cap A_2}_{m_1 \text{разів}}
\cap \dots \cap  \underbrace{A_k \cap \dots \cap A_k}_{m_1 \text{разів}})
= p_1^{m_1} p_2^{m_2} \dots p_k^{m_k}$$

Це є лише один із можливих варіантів появи подій в серії випробувань. Всього таких варіантів 
рівно $C_n^{m_1, m_2, \dots, m_k} = \frac{n!}{m_1!m_2! \dots m_k!}$. Таким чином отримуємо:
$$P_n(m_1, \dots, m_k) = \frac{n!}{m_1!m_2! \dots m_k!} p_1^{m_1} p_2^{m_2} \dots p_k^{m_k}$$
\begin{definition}
    Імовірності, що обчислюємо за даною формулою, називаються \emph{поліноміальними}, а 
    сама схема - \emph{поліноміальною схемою ймовірності}. 
\end{definition}
\begin{remark}
    $$1 = (p_1 + p_2 + \dots + p_k)^n = \sum_{m_1 + \dots + m_k = n} P_n(m_1, \dots, m_k)$$
\end{remark}
\begin{example}
    Студент ІПСА за рівнем підготовки з ймовірністю $0,3$ вважається слабким студентом, 
    з ймовірністю $0,5$ вважається середнім студентом та 
    з ймовірністю $0,2$ - середнім студентом. Яка ймовірність того, що з 6 навмання 
    обраних студентів кількість слабких та сильних буде однаковою?
\end{example}