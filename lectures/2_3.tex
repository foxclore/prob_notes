% !TEX root = ../main.tex
\section{Деякі закони розподілу випадкових величин}

\subsection{Біноміальний закон}
\noindent\textbf{Означення:}
    ДВВ $\xi$ розподілена за \emph{біноміальним законом}, 
    якщо набуває значень $0,1,...,n$ з ймовірностями \begin{equation}
        P\left\{\xi = k\right\} = C_n^k p^k q^{n-k}, q = 1 - p
    \end{equation}
\textbf{Коротке позначення:} $\xi \sim {Bin}(n, p)$.
    $n$ і $p$ --- параметри закону, $n\in \mathbb{N}$, $p\in (0;1)$.

Окремим важливим випадком біноміального закону є \emph{розподіл Бернуллі}: $\xi \sim {Bin}(1, p)$.
За законом Бернуллі розподілені випадкові події-індикатори $I_A = \begin{cases}
    0, & \text{подія A не відбулась}\\ 1, & \text{подія А відбулась}
\end{cases}$ при $P(A) = p$.

\noindent\textbf{Основне застосування:} якщо проводиться $n$ незалежних випробувань з ймовірністю успіху $p$, 
то $\xi \sim {Bin}(n, p)$ задає кількість успіхів.

\noindent\textbf{Ряд розподілу:}

\begin{tabular}{c|c|c|c|c|c}
    $\xi$ & 0 & 1 & 2 & ... & $n$ \\
    \hline
    $p$ & $q^n$ & $npq^{n-1}$ & $C_n^2 p^2 q^{n-2}$ & ... & $p^n$
\end{tabular}

\noindent\textbf{Функція розподілу:}

\begin{tabular}{c c}
    $
        F_\xi(x) = \begin{cases}
            0, & x \leq 0 \\
            q^n, & 0 < x \leq 1 \\
            q^n + npq^{n-1}, & 1 < x \leq 2 \\
            \dots \\
            1, & x > n
        \end{cases}
    $ &
    \begin{tikzpicture}[baseline={(current bounding box.center)}, yscale=2.5]
        \pgfmathsetmacro{\p}{0.6};
        \pgfmathsetmacro{\q}{1-\p};
        \pgfmathsetmacro{\n}{5};
        \draw [->] (-1,0) -- (\n+1, 0);
        \draw [->] (0, -0.1) -- (0, 1.2);
        \draw [ultra thick] (-1, 0) -- (0,0);
        \draw [ultra thick] [<-] (0,\q^\n) -- (1, \q^\n);
        \draw [ultra thick] [<-] (1, {\q^\n + \n*\p*\q^(\n-1)}) -- (2, {\q^\n + \n*\p*\q^(\n-1)});
        \draw [ultra thick] [<-] (2, {\q^\n + \n*\p*\q^(\n-1) + (\n*(\n-1)/2)*\p^2*\q^(\n-2)}) -- (3, {\q^\n + \n*\p*\q^(\n-1) + (\n*(\n-1)/2)*\p^2*\q^(\n-2)});
        \draw [ultra thick] [<-] (3, {1 - (\q^\n + \n*\p*\q^(\n-1) + (\n*(\n-1)/2)*\p^2*\q^(\n-2))}) -- (4, {1 - (\q^\n + \n*\p*\q^(\n-1) + (\n*(\n-1)/2)*\p^2*\q^(\n-2))});
        \draw [ultra thick] [<-] (4, {1 - (\q^\n + \n*\p*\q^(\n-1))}) -- (5, {1 - (\q^\n + \n*\p*\q^(\n-1))});
        \draw [ultra thick] [<-] (5, 1) -- (6, 1);
        \node [below left] at (0, 0) {0};
        \foreach \k in {1,...,\n}:
            \node [below] at (\k, 0) {\k};
        \draw [dashed] (0, 1) -- (\n, 1);
        \node [left] at (0, 1) {1};
        \node [right] [align=center] at (3.5, 0.2) {Приклад для \\ $n = \n, p = \p$};
        \node [below] at (\n+1, 0) {$x$};
        \node [left] at (0, 1.2) {$F_\xi(x)$};
    \end{tikzpicture}
\end{tabular}

Для дослідження числових характеристик скористаємося генератрисою розподілу:

$\varphi_\xi(z) = \sum\limits_{k=0}^{n} P\left\{\xi = k\right\} z^k = \sum\limits_{k=0}^{n} C_n^k p^k q^{n-k} z^k = (pz+q)^n$.
$\varphi'_\xi(z) = np(pz+q)^{n-1}$, $\varphi''_\xi(z) = n(n-1)p^2(pz+q)^{n-2}$.

$\varphi'_\xi(1) = np(p+q)^{n-1} = np$, $\varphi''_\xi(1) = n(n-1)p^2 = n^2p^2 - np^2$, $\varphi''_\xi(1) + \varphi'_\xi(1) - \left( \varphi'_\xi(1)\right)^2 = n^2p^2 - np^2 + np - n^2p^2 = np(1-p) = npq$.
Отже, знайдено значення математичного сподівання та дисперсії.

\noindent\textbf{Числові характеристики:}
\begin{enumerate}
    \item $E\xi = np$.
    \item $D\xi = npq$, $\sigma_\xi = \sqrt{npq}$.
    \item ${Mo}\xi = \begin{cases}
        \left[np+p\right], & \text{якщо } np+p \text{ не ціле}\\
        np+p, np-q, & \text{якщо } np+p \text{ ціле}
    \end{cases}$ --- як найбільш ймовірна кількість успіхів у схемі Бернуллі.
    \item ${Me}\xi$ --- одне зі значень $\left[np\right] - 1$, $\left[np\right]$, $\left[np\right] + 1$.
\end{enumerate}

\subsection{Геометричний розподіл}
\noindent\textbf{Означення:}
    ДВВ $\xi$ розподілена за \emph{геометричним законом}, 
    якщо набуває значень $1,2,3,..$ з ймовірностями \begin{equation}
        P\left\{\xi = k\right\} = pq^{k-1}, q = 1 - p
    \end{equation}
    \textbf{Коротке позначення:} $\xi \sim {Geom}(p)$.
    $p$ --- параметр закону, $p\in (0;1)$.

\noindent\textbf{Основне застосування:} якщо проводяться незалежні випробування з ймовірністю успіху $p$ до першого успішного,
то $\xi$, що задає кількість проведених випробувань, має розподіл ${Geom}(p)$.

\noindent\textbf{Ряд розподілу:}

\begin{tabular}{c|c|c|c|c|c|c}
    $\xi$ & 1 & 2 & 3 & ... & $k$ & ... \\
    \hline
    $p$ & $p$ & $pq$ & $pq^2$ & ... & $pq^{k-1}$ & ...
\end{tabular}

\noindent\textbf{Функція розподілу:}

\begin{tabular}{c c}
    $
        F_\xi(x) = \begin{cases}
            0, & x \leq 1 \\
            p, & 1 < x \leq 2 \\
            p+pq, & 2 < x \leq 3 \\
            \dots \\
            \sum\limits_{m=0}^{k-1}pq^m = 1-q^k, & k < x \leq k+1 \\
            \dots
        \end{cases}
    $ &
    \begin{tikzpicture}[baseline={(current bounding box.center)}, yscale=2.5, xscale=0.88]
        \pgfmathsetmacro{\p}{0.5};
        \pgfmathsetmacro{\q}{1-\p};
        \pgfmathsetmacro{\n}{5};
        \draw [->] (-0.3,0) -- (\n+1, 0);
        \draw [->] (0, -0.1) -- (0, 1.2);
        \draw [ultra thick] (-0.3, 0) -- (1,0);
        \foreach \k in {1,...,\n}:
            \draw [ultra thick] [<-] (\k, 1-\q^\k) -- (\k+1, 1-\q^\k);
        \node [below left] at (0, 0) {0};
        \foreach \k in {1,...,\n}:
            \node [below] at (\k, 0) {\k};
        \draw [dashed] (0, 1) -- (\n+1, 1);
        \node [left] at (0, 1) {1};
        \node [right] [align=center] at (3.2, 0.2) {Приклад для \\ $p = \p$};
        \node [below] at (\n+1, 0) {$x$};
        \node [left] at (0, 1.2) {$F_\xi(x)$};
    \end{tikzpicture}
\end{tabular}

Для дослідження числових характеристик скористаємося генератрисою розподілу:

$\varphi_\xi(z) = \sum\limits_{k=1}^{\infty} P\left\{\xi = k\right\} z^k = \sum\limits_{k=1}^{\infty} pq^{k-1} z^k = \frac{pq}{1-qz}$, якщо $\left| qz\right|<1$.
$\varphi'_\xi(z) = \frac{p}{(1-qz)^2}$, $\varphi''_\xi(z) = \frac{2pq}{(1-qz)^3}$.
$\varphi'_\xi(1) = \frac{1}{p}$, $\varphi''_\xi(1) = \frac{2q}{p^2}$, $\varphi''_\xi(1) + \varphi'_\xi(1) - \left( \varphi'_\xi(1)\right)^2 = \frac{2q}{p^2} + \frac{1}{p} - \frac{1}{p^2} = \frac{2q+p-1}{p^2} = \frac{q}{p^2}$.
Отже, знайдено значення математичного сподівання та дисперсії.

\noindent\textbf{Числові характеристики:}
\begin{enumerate}
    \item $E\xi = \frac{1}{p}$.
    \item $D\xi = \frac{q}{p^2}$, $\sigma_\xi = \frac{\sqrt{q}}{p}$.
    \item ${Mo}\xi = 1$.
    \item ${Me}\xi = \left[ \frac{-1}{\log_2(1-p)}\right]$.
\end{enumerate}

\begin{exercise}
    Записати ряд розподілу, функцію розподілу, генетратрису та обчислити
    числові характеристики для іншого означення геометричного розподілу, 
    де $\xi$ приймає значення $0,1,2,...$ з ймовірностями $P\left\{\xi = k\right\} = pq^k$.
\end{exercise}
Геометричний розподіл \textbf{<<не має пам'яті>>}: $P\left\{\xi>n+m / \xi>n\right\} = P\left\{\xi>m\right\}$.
Це означає, що кількість минулих <<невдач>> не впливає на кількість майбутніх <<невдач>>.
\begin{exercise}
    Довести цю властивість і те, що геометричний розподіл --- 
    єдиний дискретний розподіл, що <<не має пам'яті>>.
\end{exercise}
До геометричного закону можна звести \emph{закон розподілу Паскаля} $\xi \sim {Pas}(a)$,
що задається $P\left\{\xi = k\right\} = \frac{a^k}{(1+a)^{k+1}}, k = 0,1,2,..., a>0$,
заміною $p=\frac{1}{a+1}$. Для нього $E\xi = a$, $D\xi = a + a^2$.

\subsection{Розподіл Пуассона}
\noindent\textbf{Означення:}
    ДВВ $\xi$ розподілена за \emph{законом Пуассона}, 
    якщо набуває значень $0,1,2,..$ з ймовірностями \begin{equation}
        P\left\{\xi = k\right\} = \frac{a^k}{k!}e^{-a}
    \end{equation}
    \textbf{Коротке позначення:} $\xi \sim {Poiss}(a)$.
    $a$ --- параметр закону, $a > 0$.

\noindent\textbf{Ряд розподілу:}

\begin{tabular}{c|c|c|c|c|c|c}
    $\xi$ & 0 & 1 & 2 & ... & $k$ & ... \\
    \hline
    $p$ & $e^{-a}$ & $ae^{-a}$ & $\frac{a^2}{2}e^{-a}$ & ... & $\frac{a^k}{k!}e^{-a}$ & ...
\end{tabular}

\noindent\textbf{Функція розподілу:}

\begin{tabular}{c c}
    $
        F_\xi(x) = \begin{cases}
            0, & x \leq 0 \\
            e^{-a}, & 0 < x \leq 1 \\
            e^{-a}+ae^{-a}, & 1 < x \leq 2 \\
            \dots \\
            \sum\limits_{m=0}^{k-1}\frac{a^m}{m!}e^{-m}& k-1 < x \leq k \\
            \dots
        \end{cases}
    $ &
    \begin{tikzpicture}[baseline={(current bounding box.center)}, yscale=2.5, xscale=0.88]
        \pgfmathsetmacro{\a}{2};
        \pgfmathsetmacro{\n}{5};
        \draw [->] (-1,0) -- (\n+1, 0);
        \draw [->] (0, -0.1) -- (0, 1.2);
        \draw [ultra thick] (-1, 0) -- (0,0);
        \draw [ultra thick] [<-] (0, {e^(-\a)}) -- (1, {e^(-\a)});
        \draw [ultra thick] [<-] (1, {e^(-\a)*(1 + \a)}) -- (2, {e^(-\a)*(1 + \a)});
        \draw [ultra thick] [<-] (2, {e^(-\a)*(1 + \a + \a^2/2)}) -- (3, {e^(-\a)*(1 + \a + \a^2/2)});
        \draw [ultra thick] [<-] (3, {e^(-\a)*(1 + \a + \a^2/2 + \a^3/6)}) -- (4, {e^(-\a)*(1 + \a + \a^2/2 + \a^3/6)});
        \draw [ultra thick] [<-] (4, {e^(-\a)*(1 + \a + \a^2/2 + \a^3/6 + \a^4/24)}) -- (5, {e^(-\a)*(1 + \a + \a^2/2 + \a^3/6 + \a^4/24)});
        \draw [ultra thick] [<-] (5, {e^(-\a)*(1 + \a + \a^2/2 + \a^3/6 + \a^4/24 + \a^5/120)}) -- (6, {e^(-\a)*(1 + \a + \a^2/2 + \a^3/6 + \a^4/24 + \a^5/120)});
        \node [below left] at (0, 0) {0};
        \foreach \k in {1,...,\n}:
            \node [below] at (\k, 0) {\k};
        \draw [dashed] (0, 1) -- (\n+1, 1);
        \node [left] at (0, 1) {1};
        \node [right] [align=center] at (3.2, 0.2) {Приклад для \\ $a = \a$};
        \node [below] at (\n+1, 0) {$x$};
        \node [left] at (0, 1.2) {$F_\xi(x)$};
    \end{tikzpicture}
\end{tabular}

Для дослідження числових характеристик скористаємося генератрисою розподілу:

$\varphi_\xi(z) = \sum\limits_{k=0}^{\infty} P\left\{\xi = k\right\} z^k = e^{-a} \sum\limits_{k=0}^{\infty} \frac{a^k}{k!}z^k = e^{-a(1-z)}$.
$\varphi'_\xi(z) = ae^{az-a}$, $\varphi''_\xi(z) = a^2e^{az-a}$.
$\varphi'_\xi(1) = a$, $\varphi''_\xi(1) = a^2$, $\varphi''_\xi(1) + \varphi'_\xi(1) - \left( \varphi'_\xi(1)\right)^2 = a^2 + a - a^2 = a$.
Отже, знайдено значення математичного сподівання та дисперсії.

\noindent\textbf{Числові характеристики:}
\begin{enumerate}
    \item $E\xi = a$.
    \item $D\xi = a$, $\sigma_\xi = \sqrt{a}$.
    \item ${Mo}\xi = \left[a\right]$.
    \item ${Me}\xi \approx \left[a + 1/3 - 0.02/a\right]$.
\end{enumerate}

\subsection{Потік Пуассона}
\emph{Потоком} називають послідовність подій, які наступають в певні моменти часу одна за одною.
Потік характеризується випадковою величиною $\xi(t)$ --- 
кількістю подій, що наступили протягом проміжку часу $\left[ 0; t\right)$.
Ймовірності $P\left\{\xi(t) = m\right\}$ позначаються $p_m(t)$.

Накладемо на потік такі вимоги:
\begin{enumerate}
    \item \emph{Стаціонарність (однорідність)} --- кількість подій, що наступають
    за певний проміжок часу, залежить лише від довжини проміжку і не залежить
    від того, де цей проміжок розташований на часовій осі.
    \item \emph{Відсутність післядії} --- якщо проміжки часу не перетинаються, то кількості подій,
    які за ці проміжки відбулися, є незалежними подіями.
    \item \emph{Ординарність} --- події наступають поодинці, тобто
    $P\left\{\xi(t+\Delta t) - \xi(t) = 1\right\} = \lambda\cdot\Delta t + o(\Delta t)$
    ($\lambda > 0$ --- параметр інтенсивності) і 
    $P\left\{\xi(t+\Delta t) - \xi(t) \geq 2\right\} = o(\Delta t)$.
\end{enumerate}
\begin{definition}
    Потік, що має перелічені властивості, називається \emph{потоком Пуассона}.
\end{definition}
Отримаємо явний вираз для ймовірностей $p_m(t)$ у потоці Пуассона.

Позначимо $\tilde{p}_k(t+\Delta t) = P\left\{\xi(t+\Delta t) - \xi(t) = k\right\}$.
Тоді $p_m(t+\Delta t) = p_m(t)\cdot \tilde{p}_0(t+\Delta t) + p_{m-1}(t)\cdot \tilde{p}_1(t+\Delta t) +
p_{m-2}(t)\cdot \tilde{p}_2(t+\Delta t) + ... + p_0(t)\cdot \tilde{p}_m(t+\Delta t)$ для $m \geq 1$,
для $m=0$ $p_0(t+\Delta t) = \tilde{p}_0(t+\Delta t)$. Застосуємо ординарність:

$p_0(t+\Delta t) = p_0(t) \cdot (1 - \lambda \cdot \Delta t + o(\Delta t))$

$p_m(t+\Delta t) = p_m(t) \cdot (1 - \lambda \cdot \Delta t + o(\Delta t)) + p_{m-1}(t) \cdot (\lambda \cdot \Delta t + o(\Delta t)) + o(\Delta t)$

\noindent Розкриємо дужки:

$p_0(t+\Delta t) - p_0(t) = -\lambda \cdot \Delta t \cdot p_0(t) + o(\Delta t)$
\nopagebreak

$p_m(t+\Delta t) - p_m(t) = -\lambda \cdot \Delta t \cdot p_m(t) + \lambda\cdot\Delta t \cdot p_{m-1}(t) + o(\Delta t)$

\noindent Поділимо на $\Delta t$ та перейдемо до границі при $\Delta t \rightarrow 0$.
Отримаємо систему диференціальних рівнянь:

$\begin{cases}
    p'_0(t) = - \lambda p_0(t) \\
    p'_m(t) = - \lambda p_m(t) + \lambda p_{m-1}(t), \; m \geq 1
\end{cases}$

\noindent Щоб розв'язати цю систему, введемо генератрису ймовірностей $p_m(t)$:

$\varphi_\xi(z, t) = \sum\limits_{m=0}^{\infty} P\left\{\xi(t) = m\right\} z^m = \sum\limits_{m=0}^{\infty} p_m(t) z^m$ 

\noindent Помножимо всі рівняння отриманої системи на $z^m$ та складемо:

$\sum\limits_{m=0}^{\infty} p'_m(t) z^m = -\lambda \sum\limits_{m=0}^{\infty} p_m(t) z^m + \lambda \sum\limits_{m=1}^{\infty} p_{m-1}(t) z^m = 
-\lambda \sum\limits_{m=0}^{\infty} p_m(t) z^m + \lambda z \sum\limits_{m=0}^{\infty} p_{m}(t) z^m$

\noindent Отримали диференціальне рівняння для генератриси:

$\frac{\partial \varphi_\xi}{\partial t} = - \lambda \varphi_\xi + \lambda z \varphi_\xi = -\lambda(1-z) \varphi_\xi$

\noindent Його розв'язком буде $\varphi_\xi(z, t) = C(z) \cdot e^{-\lambda(1-z)t}$.
Оскільки $\varphi_\xi(z, 0) = 1$, то $C(z) = 1$.

Отже, генератриса рівна $\varphi_\xi(z, t) = e^{-\lambda(1-z)t}$ --- це генератриса закону Пуассона з параметром $a=\lambda t$.

Таким чином, $p_m(t) = \frac{(\lambda t)^m}{m!}e^{-\lambda t}$. 
Тепер зрозумілою стає назва параметру $\lambda$ (<<інтенсивність>>): $E\xi = \lambda t$, $\lambda = \frac{E\xi}{t}$ --- середня кількість подій за одиницю часу.