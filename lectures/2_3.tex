% !TEX root = ../main.tex
\section{Деякі закони розподілу випадкових величин}

\subsection{Біноміальний закон}
\begin{definition*}
    ДВВ $\xi$ розподілена за \emph{біноміальним законом}, 
    якщо набуває значень $0,1,...,n$ з ймовірностями \begin{equation}
        P\left\{\xi = k\right\} = C_n^k p^k q^{n-k}, q = 1 - p
    \end{equation}
    \emph{Коротке позначення:} $\xi \sim {Bin}(n, p)$.
    $n$ і $p$ --- параметри закону, $n\in \mathbb{N}$, $p\in (0;1)$.
\end{definition*}
Окремим важливим випадком біноміального закону є \emph{розподіл Бернуллі}: $\xi \sim {Bin}(1, p)$.
За законом Бернуллі розподілені випадкові події-індикатори $I_A = \begin{cases}
    0, & \text{подія A не відбулась}\\ 1, & \text{подія А відбулась}
\end{cases}$ при $P(A) = p$.

\emph{Основне застосування:} якщо проводиться $n$ незалежних випробувань з ймовірністю успіху $p$, 
то $\xi \sim {Bin}(n, p)$ задає кількість успіхів.

\emph{Числові характеристики:}
\begin{enumerate}
    \item 
\end{enumerate}