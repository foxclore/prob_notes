% !TEX root = ../main.tex
\section{Послідовності випадкових величин}
\subsection{Нерівності Маркова та Чебишова}
\begin{theorem*}[нерівність Маркова]
    Нехай модуль випадкової величини $\xi$ має скінченне математичне сподівання: $E|\xi| < +\infty$.
    Тоді 
    \begin{gather}\label{Markov_ineq}
        \forall \; \varepsilon >0 : P\left\{ |\xi| \geq \varepsilon\right\} \leq \frac{E|\xi|}{\varepsilon}
    \end{gather}
\end{theorem*}
\begin{proof}
    Запишемо випадкову величину $|\xi|$ через події-індикатори: 
    $|\xi| = |\xi|\cdot I\left\{|\xi| \geq \varepsilon\right\} + |\xi|\cdot I\left\{|\xi| < \varepsilon\right\} \geq
    \varepsilon\cdot I\left\{|\xi| \geq \varepsilon\right\}$. Звідси $E|\xi| \geq E \left( \varepsilon\cdot I\left\{|\xi| \geq \varepsilon\right\}\right) =
    \varepsilon \cdot P\left\{ |\xi| \geq \varepsilon\right\}$.
\end{proof}
\begin{remark}
    Еквівалентною нерівністю є $P\left\{ |\xi| < \varepsilon\right\} \geq 1 - \frac{E|\xi|}{\varepsilon}$.
\end{remark}

\begin{example}
    Багатьма спостереженнями з'ясовано, що середня кількість
    сонячних дні у Києві складає 220. Оцінити ймовірність того, що
    сонячних днів за рік буде не менше 300.

    \noindentПозначимо $\xi$ кількість сонячних днів. За умовою $\xi$ невід'ємна та $E\xi = 220$,
    тому за нерівністю Маркова $P\left\{ \xi \geq 300\right\} \leq \frac{220}{300} = \frac{11}{15}$.
\end{example}

\begin{theorem*}[нерівність Чебишова]
    Нехай випадкова величина $\xi$ має скінченні математичне сподівання та дисперсію.
Тоді
\begin{gather}\label{Cheb_ineq}
    \forall \; \varepsilon >0 : P\left\{ \left|\xi - E\xi\right| \geq \varepsilon\right\} \leq \frac{D\xi}{\varepsilon^2}
\end{gather}
\end{theorem*} 
\begin{proof}
    $P\left\{ |\xi - E\xi| \geq \varepsilon\right\} = P\left\{ (\xi - E\xi)^2 \geq \varepsilon^2\right\}$.
    Застосуємо нерівність Маркова:
    
    \noindent $P\left\{ (\xi - E\xi)^2 \geq \varepsilon^2\right\} \leq \frac{E(\xi - E\xi)^2}{\varepsilon^2} = \frac{D\xi}{\varepsilon^2}$.
\end{proof}
\begin{remark}
    Еквівалентною нерівністю є $P\left\{ \left|\xi - E\xi\right| < \varepsilon\right\} \geq 1 - \frac{D\xi}{\varepsilon^2}$.
\end{remark}
\begin{example}
    Отримаємо <<правило $3 \sigma$>> для довільної випадкової величини зі скінченними математичним сподівання та дисперсією.
    $P\left\{ |\xi - E\xi| < 3 \sigma\right\} \geq 1 - \frac{D\xi}{9 \sigma^2} = 1 - \frac{\sigma^2}{9 \sigma^2} = \frac{8}{9}$.
\end{example}
Розглянемо застосування \emph{нерівності Чебишова в схемі Бернуллі}. Нехай $\xi \sim \mathrm{Bin}(n, p)$, $E\xi = np$, $D\xi = npq$.
Відношення $\frac{\xi}{n}$ називається відносною частотою появи успіху або частістю. $E\left( \frac{\xi}{n}\right) = p$, 
$D\left( \frac{\xi}{n}\right) = \frac{pq}{n}$. З нерівності Чебишова 
$P\left\{ \left|\frac{\xi}{n} - p\right| \geq \varepsilon\right\} \leq \frac{pq}{n \epsilon^2}$.

\subsection{Послідовності випадкових величин}
Розглядаємо фіксований ймовірнісний простір $\left\{ \Omega, \mathcal{F}, P\right\}$ та
послідовність випадкових величин $\left\{ \xi_n (\omega)\right\}_{n=1}^{\infty}$.
Якщо для кожного $n \in \mathbb{N}$ події $\xi_1, \xi_2, ..., \xi_n$ \emph{незалежні у сукупності},
то послідовність $\left\{ \xi_n (\omega)\right\}_{n=1}^{\infty}$ називається \emph{послідовністю незалежних випадкових величин}.

Послідовності випадкових величин можна задавати різними способами:
\begin{enumerate}
    \item Нехай $\xi$ --- деяка випадкова величина, можна задати $\xi_n = f_n(\xi)$, де $f_n$ --- деяка числова функція. 
    Наприклад: $\xi_n = \xi^n$, $\xi_n = \cos (n\xi)$.
    \item $n$ може входити як параметр закону розподілу $\xi_n$. Наприклад, $\xi_n \sim \mathrm{Exp}(n)$, $\xi_n \sim \mathrm{N}(0, \frac{1}{n})$.
    \item Для послідовностей ДВВ $n$ може входити як в значення, що приймає $\xi_n$, так і у відповідні ймовірності. Наприклад:
    \begin{tabular}{|c|c|c|c|}
        \hline
        $\xi_n$ & $-\sqrt{n}$ & $0$ & $\sqrt{n}$ \\
        \hline
        $p$ & $1/n$ & $1 - 2/n$ & $1/n$ \\
        \hline
    \end{tabular}
\end{enumerate}
В курсі функціонального аналізу вводяться різні види збіжності послідовності вимірних функцій та
зв'язок між цими видами збіжності.

\subsection{Види збіжності послідовності випадкових величин}
Нагадаємо класичне означення границі числової послідовності. Число $a$ називають границею послідовності 
$\left\{ a_n\right\}_{n=1}^{\infty}$, якщо
$$ \forall \; \varepsilon > 0 \; \exists \; N(\varepsilon) \in \mathbb{N}: \forall \; n \geq N(\varepsilon) \; \left| a_n - a\right| < \varepsilon$$
Оскільки послідовність випадкових величин $\left\{ \xi_n\right\}_{n=1}^{\infty}$ є послідовністю функцій з $\Omega$ в $\mathbb{R}$, то це означення
не є застосовним, бо $\left| \xi_n - \xi\right| < \varepsilon$ є випадковою подією, що виконується, взагалі кажучи, не для всіх елементарних подій
$\omega \in \Omega$. Тому ми маємо ввести інше означення границі (та збіжності) послідовності
випадкових величин. Виявляється, що таких означень можна запропонувати декілька, причому вони
не є еквівалентними одне одному.
\vspace{1em}
\noindent\begin{enumerate}
    \item \textbf{Збіжність майже напевно (сильна збіжність, збіжність з ймовірністю 1).}
    \noindent$\left\{ \xi_n\right\}_{n=1}^{\infty}$ збігається до $\xi$ \emph{майже напевно}, якщо
    $P\left\{ \omega: \underset{n\to\infty}{\lim} \xi_n(\omega) = \xi(\omega)\right\} = 1$. 
    Це еквівалентно умові $P\left\{ \omega: \underset{n\to\infty}{\lim} \xi_n(\omega) \neq \xi(\omega)\right\} = 0$.

    Позначення: $\xi_n \overset{\mathrm{P1}}{\longrightarrow} \xi, n \to \infty$ або $\xi_n \overset{\text{м.н.}}{\longrightarrow} \xi, n \to \infty$.
    \begin{exercise}
        Довести ще одне еквівалентне означення цієї збіжності: $\xi_n \overset{\mathrm{P1}}{\longrightarrow} \xi, n \to \infty$, якщо
        $\forall \; \varepsilon > 0: P\left( \bigcap\limits_{n=1}^{\infty} \bigcup\limits_{k = n}^{\infty}
        \left\{ \omega : \left| \xi_k(\omega) - \xi(\omega)\right| > \varepsilon\right\}\right) = 0$.
    \end{exercise}

    Це найбільш природний з інтуїтивної точки зору вид збіжності. Зауважимо, однак, що він має доволі дивні властивості. Наприклад, можна навести приклад
    послідовності $\xi_n$, що не збігається до деякої $\xi$, але будь-яка її підпослідовність $\xi_{n_k}$ містить свою підпослідовність
    $\xi_{n_{k_l}}$, яка все ж таки збігається до $\xi$.
    
    \begin{proposition*}[лема Бореля-Кантеллі]
        Якщо для послідовності подій $\left\{A_n \right\}_{n=1}^{\infty}$ ряд
    $\sum\limits_{n=1}^{\infty} P(A_n)$ збігається, то $P\left( \bigcap\limits_{n=1}^{\infty} \bigcup\limits_{k = n}^{\infty} A_k\right) = 0$.
    Це означає, що ймовірність того, що відбудеться нескінченна кількість цих подій, є нульовою.
    \end{proposition*}
    \begin{proof}
        Послідовність подій $B_n = \bigcup\limits_{k = n}^{\infty} A_k$ монотонно спадна, тому за теоремою неперервності \ref{th:2}
        $P\left( \bigcap\limits_{n=1}^{\infty} B_n\right) = \underset{n\to\infty}{\lim} P(B_n)$.
        $P(B_n) = P\left( \bigcup\limits_{k = n}^{\infty} A_k\right) \leq \sum\limits_{k=n}^{\infty}P(A_n) \to 0$ при $n\to \infty$ 
        зі збіжності ряду. Тому $\underset{n\to\infty}{\lim} P(B_n) = P\left( \bigcap\limits_{n=1}^{\infty} \bigcup\limits_{k = n}^{\infty} A_k\right) = 0$.
    \end{proof}
    Застосуванням цієї леми до послідовності подій $A_n = \left\{ \omega : \left| \xi_n(\omega) - \xi(\omega)\right| > \varepsilon\right\}$ отримаємо зручну для використання
    ознаку збіжності майже напевно: якщо для будь-якого $\varepsilon >0$ ряд
    $\sum\limits_{n=1}^{\infty} P\left\{\left| \xi_n - \xi\right| > \varepsilon\right\}$ збігається, то 
    $\xi_n \overset{\mathrm{P1}}{\longrightarrow} \xi, n \to \infty$.
    \begin{example}
        Довести, що послідовність $\xi_n \sim \mathrm{U}(-\frac{1}{n}, \frac{1}{n})$ при $n\to\infty$ збігається до 0 майже напевно.
        
        Візьмемо довільне $\varepsilon > 0$ та знайдемо $P\left\{ |\xi_n| > \varepsilon\right\}$. Для $\varepsilon \geq 1$
        ця ймовірність, очевидно, рівна 0. В іншому випадку, для кожного $\varepsilon \in (0; 1)$ можна знайти такий номер $N$,
        для якого $\varepsilon$ буде більше за $\frac{1}{n}$ при $n \geq N$. Тому для будь-якого $\varepsilon >0$ ймовірності
        $P\left\{ |\xi_n| > \varepsilon\right\}$ рівні 0, починаючи з якогось $n$. Отже, для будь-якого $\varepsilon >0$ ряд 
        $\sum\limits_{n=1}^{\infty} P\left\{\left| \xi_n \right| > \varepsilon\right\}$ збігається і 
        $\xi_n \overset{\mathrm{P1}}{\longrightarrow} 0, n \to \infty$.
    \end{example}
    \item \textbf{Збіжність за ймовірністю.}
    \noindent$\left\{ \xi_n\right\}_{n=1}^{\infty}$ збігається до $\xi$ \emph{за ймовірністю}, якщо 
    $\forall \; \varepsilon > 0: \underset{n \to \infty}{\lim} P\left\{|\xi_n - \xi| \geq \varepsilon\right\}= 0$.
    Це еквівалентно умові $\forall \; \varepsilon > 0: \underset{n \to \infty}{\lim} P\left\{|\xi_n - \xi| \leq \varepsilon\right\}= 1$.
    Позначення: $\xi_n \overset{\mathrm{P}}{\longrightarrow} \xi, n \to \infty$.
    \begin{example}
        Нехай $\xi_n$ --- послідовність ДВВ: 
        \begin{tabular}{|c|c|c|}
            \hline
            $\xi_n$ & $0$ & $n^7$ \\
            \hline
            $p$ & $1-1/n$ & $1/n$ \\
            \hline
        \end{tabular}.
        Перевірити збіжність $\xi_n \overset{\mathrm{P}}{\longrightarrow} 0, n \to \infty$.
        
        \noindentДля $\varepsilon >0$ $P\left\{|\xi_n -0| \geq \varepsilon\right\} = P\left\{ \xi_n \geq \varepsilon\right\}$.
        $\forall \; \varepsilon >0 \; \exists \; N: \forall n\geq N:n^7 > \varepsilon$, тому з якогось номера
        $P\left\{ \xi_n \geq \varepsilon\right\} = P\left\{ \xi_n = n^7 \right\} = \frac{1}{n} \to 0, n\to\infty$, тому $\xi_n \overset{\mathrm{P}}{\longrightarrow} 0, n \to \infty$.
    \end{example}
    \item \textbf{Збіжність в середньому.}
    \noindent$\left\{ \xi_n\right\}_{n=1}^{\infty}$ збігається до $\xi$ \emph{в середньому},
    якщо $\underset{n \to \infty}{\lim} E|\xi_n - \xi| = 0$.
    Позначення: $\xi_n \overset{\text{С}}{\longrightarrow} \xi, n \to \infty$.
    \item \textbf{Збіжність в середньому квадратичному.}
    \noindent$\left\{ \xi_n\right\}_{n=1}^{\infty}$ збігається до $\xi$ \emph{в середньому квадратичному},
    якщо $\underset{n \to \infty}{\lim} E(\xi_n - \xi)^2 = 0$.
    
    \noindentПозначення: $\xi_n \overset{\text{СК}}{\longrightarrow} \xi, n \to \infty$.
    \item \textbf{Збіжність за розподілом (слабка збіжність, збіжність в основному).}

    \noindent$\left\{ \xi_n\right\}_{n=1}^{\infty}$ збігається до $\xi$ \emph{за розподілом}, якщо функціональна послідовність
    $\left\{ F_{\xi_n} (x)\right\}_{n=1}^{\infty}$ збігається до $F_{\xi}(x)$ в точках її неперервності.

    Позначення: $\xi_n \overset{\mathrm{F}}{\longrightarrow} \xi, n \to \infty$.

    Ця збіжність за характером відрізняється від інших тим, що не враховує залежність або незалежність $\xi_n$.
    Є також \emph{еквівалентне означення збіжності} за розподілом: якщо для будь-якої обмеженої неперервної функції $\varphi : \mathbb{R} \to \mathbb{R}$
    $\underset{n \to \infty}{\lim} E \varphi(\xi_n) = E\varphi(\xi)$.
    Доведення еквівалентності цих двох означень виходить за рамки курсу: його ідея полягає в тому, що $F_{\xi}(x) = P\left\{ \xi < x\right\} = E 1_{(-\infty, x)}(\xi)$,
    де $1_{(-\infty, x)}$ --- індикатор множини $(-\infty, x)$, і такі функції-індикатори можна наблизити з будь-якою заданою точністю неперервними функціями та навпаки.
    \begin{example}
        Нехай $\xi_n \sim \mathrm{Exp}(\frac{1}{n})$. Перевірити $\xi_n \overset{\mathrm{F}}{\to} 0, n \to \infty$.
        Тут під 0 розуміється ДВВ, що приймає значення 0 з ймовірністю 1.

        $F_{\xi_n}(x) = \begin{cases}
            1 - e^{\frac{x}{n}}, & x > 0 \\
            0, & x \leq 0
        \end{cases}$. Видно, що при $n \to \infty$ $F_{\xi_n}(x) \to F_0(x) = \begin{cases}
            1, & x > 0 \\
            0, & x \leq 0
        \end{cases}
        $ --- функція розподілу 0.
    \end{example}
\end{enumerate}

\begin{remark}
    Збіжності в середньому та середньому квадратичному є частковими випадками \emph{збіжності порядку $k$}, для якої
$\underset{n \to \infty}{\lim} E|\xi_n - \xi|^k = 0$. Оскільки 
для $0 < s < t$ має місце
$( E \left| \xi\right|^s)^{\frac{1}{s}} \leq ( E \left| \xi\right|^t)^{\frac{1}{t}}$
то збіжність порядку $k$ гарантує збіжність порядків менше $k$.
\end{remark}
\begin{proposition*}
     Нехай послідовність $\left\{ \xi_n\right\}_{n=1}^{\infty}$ збігається за ймовірністю до двох випадкових
    величин $\xi$ та $\eta$. Тоді $P\left\{ \xi = \eta\right\} = 1$.
\end{proposition*}
\begin{proof}
    Для будь-якого $\varepsilon > 0$ : $\left\{ |\xi - \eta| > \varepsilon\right\} \subset 
    \left\{ |\xi - \xi_n| > \frac{\varepsilon}{2}\right\} \cup \left\{ |\eta - \xi_n| > \frac{\varepsilon}{2}\right\}$, тому \\
    $P\left\{ |\xi - \eta| > \varepsilon\right\} \leq P\left\{ |\xi - \xi_n| > \frac{\varepsilon}{2}\right\} + 
    P\left\{ |\eta - \xi_n| > \frac{\varepsilon}{2}\right\} \to 0, n\to\infty$.
    Отже, $P\left\{ |\xi - \eta| > \varepsilon\right\} = 0$, і за довільністю $\varepsilon > 0$ маємо $P\left\{ \xi \neq \eta\right\} = 0$,
    або ж $P\left\{ \xi = \eta\right\} = 1$.
\end{proof}
\begin{exercise}
    Довести, що це твердження виконується для збіжностей з ймовірністю 1, в середньому та середньому квадратичному, але не виконується
    для збіжності за розподілом.
\end{exercise}
Можна довести, що усім збіжностям, крім збіжності за розподілом, притаманні відомі арифметичні властивості: збіжність суми $\xi_n + \eta_n$ та добутку
$\xi_n \eta_n$ до $\xi + \eta$ та $\xi \eta$ відповідно за умови збіжностей $\xi_n$ до $\xi$ та $\eta_n$ до $\eta$. Також, $\varphi(\xi_n)$ збігається
до $\varphi(\xi)$ за умови неперервності $\varphi$.

В курсі функціонального аналізу встановлюється зв'язок між розглянутими видами збіжності:
\vspace{5 pt}
\begin{center}
\begin{tikzpicture}
\label{scheme}
\node[block] (a) {Збіжність з ймовірністю 1 \\ ($\mathrm{P1}$)};
\node[block, above right = 0cm and 2cm of a] (b) {Збіжність в середньому\\ квадратичному ($\text{СК}$)};
\node[block, below right = 0cm and 2cm of a]   (c) {Збіжність в середньому \\ ($\text{С}$)};
\node[block, below right = 2cm and -1.5cm of a]   (d) {Збіжність за ймовірністю \\ ($\mathrm{P}$)};
\node[block, below = 1cm of d]   (e){Збіжність за розподілом \\ ($\mathrm{F}$)};
\node[below left = 1cm and -2.5 cm of a]   (1){\circled 1};
\node[below right = 0.25cm and -1.6 cm of d]   (2){\circled 2};

\draw[line] (a.south) -- ([xshift=-0.5cm]d.north);
\draw[line] (b.south) -- (c.north);
\draw[line] (c.south) -- ([xshift=0.5cm]d.north);
\draw[line] ([xshift=-0.5cm]d.south) -- ([xshift=-0.5cm]e.north);
\draw[line,dashed] ([xshift=-1.5cm]d.north) -- ([xshift=-1cm]a.south);
\draw[line,dashdotted] ([xshift=0.5cm]e.north) -- ([xshift=0.5cm]d.south);
\end{tikzpicture}
\end{center}

Штрихова лінія \circled 1 означає, що у кожній послідовності $\left\{ \xi_n\right\}_{n=1}^{\infty}$, що збігається за ймовірністю, 
міститься підпослідовність $\left\{ \xi_{n_k}\right\}_{k=1}^{\infty}$, що збігається з імовірністю 1. 
Штрих-пунктирна лінія \circled 2 показує, що відповідний перехід справедливий, коли гранична випадкова величина $\xi$ є константою.
Доведемо деякі з цих тверджень.

\begin{proposition*} 
    Зі збіжності в середньому квадратичному випливає збіжність в середньому.
\end{proposition*}
\begin{proof}
    $E|\xi_n -\xi| \leq \sqrt{E(\xi_n - \xi)^2}$, як було сказано вище, тому з $E(\xi_n - \xi)^2 \to 0$
    випливає $E|\xi_n - \xi| \to 0, n\to\infty$.
\end{proof}
\begin{proposition*}
    Зі збіжності в середньому випливає збіжність за ймовірністю.
\end{proposition*}
\begin{proof}
    З нерівності Маркова (\ref{Markov_ineq})
    $\forall \; \varepsilon > 0: P\left\{|\xi_n - \xi| \geq \varepsilon\right\} \leq \frac{E|\xi_n - \xi|}{\varepsilon}$, 
    тому з $E|\xi_n - \xi| \to 0$ маємо $P\left\{|\xi_n - \xi| \geq \varepsilon\right\} \to 0, n\to\infty$.
\end{proof}
\begin{proposition*}
     Зі збіжності за розподілом \emph{до константи} випливає збіжність за ймовірністю.
\end{proposition*}
\begin{proof}
    Нехай $\xi_n \overset{\mathrm{F}}{\longrightarrow} c$, $c \in \mathbb{R}$. Візьмемо $\varepsilon > 0$.
    $P\left\{ |\xi_n - c| \leq \varepsilon\right\} = P\left\{ c - \varepsilon \leq \xi_n \leq c + \varepsilon \right\} \geq
    P\left\{ c - \varepsilon \leq \xi_n < c + \varepsilon \right\} = F_{\xi_n}(c+\varepsilon) - F_{\xi_n}(c-\varepsilon)$, тому
    $\underset{n \to \infty}{\lim} P\left\{ |\xi_n - c| \leq \varepsilon\right\} \geq \underset{n \to \infty}{\lim}  F_{\xi_n}(c+\varepsilon) -
    \underset{n \to \infty}{\lim}  F_{\xi_n}(c-\varepsilon) = F_c(c+\varepsilon) - F_c(c-\varepsilon) = 1$, звідки $\xi_n \overset{\mathrm{P}}{\longrightarrow} c$.
\end{proof}
\begin{example}
    Нехай $\xi_n$ --- послідовність ДВВ: 
        \begin{tabular}{|c|c|c|c|}
            \hline
            $\xi_n$ & $-n$ & $0$ & $2n$ \\
            \hline
            $p$ & $1/{2n}$ & $1 - 1/n$ & $1/{2n}$ \\
            \hline
        \end{tabular}.

    Зі збільшенням $n$ $\xi_n$ все з більшою ймовірністю набувають значення 1. 
    Водночас, інші два можливі значення ($-n$ та $2n$) розбігаються до нескінченності. 
    Ці два процеси спричиняють протилежні ефекти --- перший наближає $\xi_n$ до 1, а другий --- віддаляє. 
    Тому з точки зору одних видів збіжності, більш чутливих до <<аномальних викидів>>, послідовність $\xi_n$ не буде мати границі, 
    а з точки зору інших, менш чутливих, границя дорівнюватиме 1.
    Дійсно: 
    \begin{gather*}
        E\left|\xi_n -1 \right| = \frac{1}{2n} |-n-1|  + \left( 1 - \frac{1}{n}\right)|1-1| + \frac{1}{2n}|2n-1| \to \frac{3}{2} \\
        E\left(\xi_n -1 \right)^2 = \frac{1}{2n} (-n-1)^2  + \left( 1 - \frac{1}{n}\right)(1-1)^2 + \frac{1}{2n}(2n-1)^2 \to \infty
    \end{gather*}
    Отже, збіжності до 1 ні в середньому, ні в середньому квадратичному немає. Зауважимо, що якби <<викиди>> прямували до нескінченності повільніше або відповідні
    ймовірності прямували до нуля швидше, то ці збіжності могли б бути: наприклад, якщо замість $n$ та $-2n$ $\xi_n$ набували значення $\sqrt{n}$ та $2\sqrt{n}$
    з ймовірностями $\frac{1}{2n^2}$, то послідовність збігалася б до 1 і в середньому, і в середньому квадратичному.
    З іншого боку, для будь-якого $\varepsilon>0$ значення $-n$ та $2n$ рано чи пізно вийдуть за межі відрізку $[1-\varepsilon;1+\varepsilon]$, і тому
    $\xi_n \overset{\mathrm{P}}{\longrightarrow} 1$. Тепер стає зрозуміло, чому перевіряли збіжність в середньому та середньому квадратичному лише до 1: якби
    якась з цих границь існувала, то вона була б і границею за ймовірністю.
\end{example}
\begin{remark}Для збіжності $\xi_n \overset{\text{СК}}{\longrightarrow} C$ ($C$ --- стала) необхідно і достатньо, щоб
$\underset{n \to \infty}{\lim} E\xi_n = C$ та $\underset{n \to \infty}{\lim} D\xi_n = 0$.
$\underset{n \to \infty}{\lim} E\xi_n = C \Leftrightarrow \underset{n \to \infty}{\lim} (E\xi_n - EC) = 0$,
$E(\xi_n - C)^2 = D\xi_n + (E(\xi_n -C))^2$,
звідки отримуємо твердження при $n \to \infty$, оскільки $D\xi_n \geq 0$ та $(E(\xi_n -C))^2 \geq 0$.
Зауважимо, що ці дві умови є достатніми для збіжності $\xi_n \overset{\mathrm{P}}{\longrightarrow} C$.
\end{remark}
\begin{example}
    Нехай $\xi_n \sim \mathrm{N}\left(5 + \frac{1}{n}, \sigma = \frac{1}{n}\right)$. Перевірити $\xi_n \overset{\mathrm{P}}{\longrightarrow} 5$.
    
    \noindent Оскільки $E\xi_n = 5+\frac{1}{n} \to 5$, $D\xi_n = \frac{1}{n^2} \to 0$ при $n \to \infty$,
    то $\xi_n \overset{\text{СК}}{\longrightarrow} 5 \Rightarrow \xi_n \overset{\mathrm{P}}{\longrightarrow} 5$.
\end{example}

Наведемо без доведення важливу теорему, що стосується збіжності за розподілом.

\begin{theorem*}[теорема неперевності Леві]
    Збіжність за розподілом послідовності випадкових величин $\left\{ \xi_n\right\}_{n=1}^{\infty}$ еквівалентна поточковій збіжності їх
    характеристичних функцій:
    \begin{gather}\label{levi_theor}
        \xi_n \overset{\mathrm{F}}{\longrightarrow} \xi \Longleftrightarrow \chi_{\xi_n}(t) \to \chi_\xi (t) \; \forall \; t\in \mathbb{R}
    \end{gather}
\end{theorem*}