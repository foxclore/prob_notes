% !TEX root = ../main.tex
\section{Дискретні випадкові вектори}
\begin{definition}
    Вимірна функція $\Omega \rightarrow \mathbb{R}^n$, задана на ймовірнісному 
    просторі $\left\{\Omega, \mathcal{F}, P \right\}$, 
    називається \emph{випадковим вектором}.
\end{definition}

\noindent \textbf{Позначення: }
$\vec{\xi} = \vec{\xi}(\omega) = 
\left(\xi_1(\omega), \xi_2(\omega), ... , \xi_n(\omega)\right)^{T}$

\begin{remark}
    Під вимірністю мається на увазі те, що 
    \begin{equation*}
        \forall \vec{x} \in \mathbb{R}^n: 
        A=\left\{\omega \in \Omega: \xi_1(\omega) < x_1, 
                                    \xi_2(\omega) < x_2,
                                    ... ,
                                    \xi_n(\omega) < x_n\right\}
        \in \mathcal{F}
    \end{equation*}
\end{remark}

\begin{definition}
    Випадковий вектор називається \emph{дискретним}, якщо всі його координати - 
    дискретні випадкові величини.
\end{definition}
\begin{remark}
    Випадковий вектор можна трактувати як випадкову точку в $\mathbb{R}^n$.

    При $n = 2$: $\vec{\xi} = (\xi_1, \xi_2)^T$ --- випадкова точка на площині. 
\end{remark}

Закон розподілу двовимірного дискретного випадкового вектора задається 
\emph{таблицею розподілу}.


\hbox to \hsize{\hfil{
    \begin{tabular}{|c|c|c|c|c|}
        \hline
        \backslashbox{$\xi_2$}{$\xi_1$} & $x_1$ & $x_2$ & ... & $x_n$ \\
        \hline
        $y_1$ & $p_{11}$ & $p_{21}$ & ... & $p_{n1}$ \\
        \hline
        $y_2$ & $p_{12}$ & $p_{22}$ & ... & $p_{n2}$ \\
        \hline
        ... & ... & ... & $p_{ij}$ & ... \\
        \hline
        $y_m$ & $p_{1m}$ & $p_{2m}$ & ... & $p_{nm}$ \\
        \hline
    \end{tabular}
}\hfil}