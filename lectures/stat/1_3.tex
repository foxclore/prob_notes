% !TEX root = ../../main.tex
\section{Інтервальне оцінювання}
Нехай $\xi$ --- ГС, $\theta$ --- якийсь параметр її
розподілу.
Задача \emph{інтервального оцінювання} $\theta$ --- це пошук за заданим \emph{рівнем надійності} 
$\gamma$ статистик
$\theta^*_1(\vec{\xi})$, $\theta^*_2(\vec{\xi})$ таких, що:
\begin{gather*}
    \P\left\{\theta \in (\theta^*_1, \theta^*_2)\right\} = \gamma \Leftrightarrow
    \P\left\{ \theta^*_1 < \theta < \theta^*_2\right\} = \gamma
\end{gather*}
Іноді в цих співвідношеннях пишуть $\geq \gamma$, оскільки у випадку дискретного розподілу цих статистик рівності можуть не мати сенсу.

\begin{definition}
    $(\theta^*_1, \theta^*_2)$ називається \emph{довірчим інтервалом з рівнем надійності $\gamma$}.
    Більш точно --- маємо справу з послідовністю таких інтервалів, що залежать від обсягу вибірки.
\end{definition}
\begin{remark}
    Оскільки межі інтервалу є випадковими, то запис $\theta \in (\theta^*_1, \theta^*_2)$ правильно читати не як
    <<$\theta$ потрапляє в інтервал $(\theta^*_1, \theta^*_2)$>>, а як
    <<інтервал $(\theta^*_1, \theta^*_2)$ накриває $\theta$>>.
\end{remark}
Зручно шукати довірчі інтервали конкретного вигляду, наприклад,
\emph{симетричні} відносно $\theta$ з умови $\P\left\{|\theta - \theta^*| < \varepsilon\right\} = \gamma$, де $\varepsilon$ --- \emph{точність} 
довірчого інтервалу, або \emph{однобічні} з умов $\P\{\theta > \theta^*\} = \gamma$ чи $\P\{\theta < \theta^*\} = \gamma$.
У випадку симетричного довірчого інтервалу зазвичай одразу обирається сама статистика $\theta^*$ (наприклад, якась <<хороша>> точкова оцінка).
В такому разі пошук $\varepsilon$ --- це пошук ширини довірчого інтервалу, що забезпечує заданий рівень надійності.

Якщо розглядається конкретна реалізація вибірки, то можна обчислити межі знайденого інтервалу як значення відповідних статистик.

\subsection{Побудова довірчих інтервалів для гаусcівської ГС}
Нехай ГС $\xi \sim \mathrm{N}(a, \sigma^2)$. Розглянемо чотири випадки побудови довірчих інтервалів
для параметрів $a$ та $\sigma^2$ (нагадаємо, що це математичне сподівання та дисперсія).

\noindent\textbf{Довірчий інтервал для математичного сподівання при відомій дисперсії.}

Шукатимемо симетричний довірчий інтервал $\P\left\{|a - a^*| < \varepsilon\right\} = \gamma$, де в якості
оцінки $a^*$ буде вибіркове середнє $\overline{\xi}$. З властивостей незалежних гаусcівських ВВ
маємо $\sum\limits_{k=1}^n \xi_k \sim \mathrm{N}(na, n\sigma^2)$, тому $\overline{\xi} \sim \mathrm{N}\left(a, \frac{\sigma^2}{n}\right)$.
Отже,
$
    \P\{|a - a^*| < \varepsilon\} = \P \{a - \varepsilon < \overline{\xi} < a + \varepsilon\} = 
    2\Phi\left(\frac{\varepsilon\sqrt{n}}{\sigma}\right) = \gamma
$, де $\Phi(x)$ --- функція Лапласа. З таблиці її значень знайдемо відповідне значення $\varepsilon$ і отримаємо шуканий
довірчий інтервал $\left(\overline{\xi}-\varepsilon, \overline{\xi}+\varepsilon \right)$.

\noindent\textbf{Довірчий інтервал для математичного сподівання при невідомій дисперсії.}

Шукатимемо симетричний довірчий інтервал $\P\left\{|a - a^*| < \varepsilon\right\} = \gamma$. В умовах попереднього прикладу
$\sqrt{n} \cdot \frac{\overline{\xi} - a}{\sigma} \sim \mathrm{N}(0, 1)$, але $\sigma$ тепер невідоме. Розглянемо статистику
$\D^* \xi = \frac{1}{n}\sum\limits_{k=1}^n (\xi_k - \overline{\xi})^2 = \frac{1}{n} \sum\limits_{k=1}^n (\xi_k - a)^2 - (\overline{\xi} - a)^2$.
$\frac{n \D^* \xi}{\sigma^2} = \sum\limits_{k=1}^n \left(\frac{\xi_k - a}{\sigma} \right)^2 - 
\left(\sqrt{n} \cdot \frac{\overline{\xi} - a}{\sigma} \right)^2$. В теоремі нижче буде доведено незалежність доданків,
що дасть право сказати, що $\frac{n \D^* \xi}{\sigma^2} \sim \chi_{n-1}^2$. Як наслідок,
$$ 
\frac{\sqrt{n}\cdot\frac{\overline{\xi} - a}{\sigma}}{\sqrt{\frac{1}{n-1} \cdot \frac{n \D^* \xi}{\sigma^2} }} = 
\frac{\sqrt{n}\cdot(\overline{\xi} - a)}{\sqrt{\D^{**}\xi}} \sim \mathrm{St}_{n-1}
$$

Тепер з рівності $\P\left\{ \frac{\sqrt{n}\cdot|\overline{\xi} - a|}{\sqrt{\D^{**}\xi}} < t_{\gamma}\right\} = \gamma$ знайдемо
значення $t_{\gamma}$. Шуканим значенням $\varepsilon$ буде $\varepsilon = \frac{t_{\gamma}}{\sqrt{n}} \cdot \sqrt{(\D^{**}\xi)_{\text{зн}}}$.
\begin{theorem*}[теорема Фішера]
    Статистики стандартної гаусcівської ГС $\overline{\xi}$ та $\D^*\xi$ --- незалежні випадкові величини.
\end{theorem*}
\begin{proof}
    $\vec{\xi}$ --- випадкова вибірка, $\vec{\xi} \sim \mathrm{N}(\vec{0}, I)$. Нехай $C$ --- деяка ортогональна матриця,
    тоді $\vec{\eta} = C \vec{\xi}$ теж має розподіл $\mathrm{N}(\vec{0}, I)$, причому 
    $\Vert \vec{\eta} \Vert = \Vert \vec{\xi} \Vert$. Розглянемо матрицю 
    $$
    C = \begin{pmatrix}
        c_{1,1} & c_{1,2} & \ldots & c_{1,n} \\
        c_{2,1} & c_{2,2} & \ldots & c_{2,n} \\
        \vdots & \vdots & \ddots & \vdots \\
        c_{n-1,1} & c_{n-1,2} & \ldots & c_{n-1,n} \\
        1/\sqrt{n} & 1/\sqrt{n} & \ldots & 1/\sqrt{n}
    \end{pmatrix}
    $$
    перші $n-1$ рядків якої --- це елементи ортонормованого базису $\text{л.о.}\left\{
    \begin{pmatrix}
        \frac{1}{\sqrt{n}} & ... & \frac{1}{\sqrt{n}}
    \end{pmatrix}^{T}\right\}^{\perp}$.
    $\vec{\eta} = C\vec{\xi}$, $\eta_n = \frac{1}{\sqrt{n}} \left(\xi_1 + \xi_2 + ... + \xi_n \right) = \sqrt{n} \cdot \overline{\xi}$.
    Вище у прикладі побудови довірчого інтервалу було показано, що
    $\D^* \xi = \frac{1}{n} \sum\limits_{k=1}^n (\xi_k - a)^2 - (\overline{\xi} - a)^2$. В умовах теореми ця рівність спрощується до
    $\D^* \xi = \frac{1}{n} \sum\limits_{k=1}^n \xi_k^2 - \overline{\xi}^2 = \frac{1}{n} \Vert \vec{\xi} \Vert^2 - \frac{1}{n}\eta_n^2 = 
    \frac{1}{n} \Vert \vec{\eta} \Vert^2 - \frac{1}{n}\eta_n^2 = \frac{1}{n} \sum\limits_{k=1}^n \eta_k^2 - \frac{1}{n}\eta_n^2 = 
    \frac{1}{n} \sum\limits_{k=1}^{n-1} \eta_k^2$. Таким чином, $\D^* \xi$ залежить від перших $n-1$ координат $\vec{\eta}$, а отже ---
    не залежить від $\overline{\xi} = \frac{1}{\sqrt{n}} \eta_n$.
\end{proof}
\begin{example}
    Побудувати 95\% довірчий інтервал для математичного сподівання гаусcівської ГС, якщо $\overline{x} = 2$, $n=25$,
    а дисперсія відома і рівна $6$, а потім --- якщо невідома і $(\D^{**}\xi)_{\text{зн}} = 5.78$.
    \begin{enumerate}
        \item $\D\xi = 6$. За умовою $\gamma = 0.95$, тому $\varepsilon$ шукаємо з 
        $2\Phi\left(\frac{\varepsilon\sqrt{n}}{\sigma}\right) = \gamma$. $\Phi\left(\varepsilon\cdot\frac{5}{\sqrt{6}}\right) = 0.475$, звідки
        $\varepsilon = \frac{\sqrt{6}}{5} \cdot 1.96 \approx 0.96$. Отже, шуканий довірчий інтервал --- $(1.04, 2.96)$.
        \item $(\D^{**}\xi)_{\text{зн}} = 5.78$. Спочатку знайдемо $t_{\gamma} = 2.064$, звідки $\varepsilon = \frac{2.064}{\sqrt{25}}\cdot \sqrt{5.78} \approx 2.386$.
        Отже, шуканий довірчий інтервал --- $(-0.064, 4.064)$.
    \end{enumerate}
\end{example}

\label{normal_variance_conf_interv}
\noindent\textbf{Довірчий інтервал для дисперсії при відомому математичному сподіванні.}

В якості точкової оцінки дисперсії візьмемо $\D^* \xi = \frac{1}{n} \sum\limits_{k=1}^n (\xi_k - a)^2$, тоді
$\frac{n \D^* \xi}{\sigma^2} = \sum\limits_{k=1}^n \left(\frac{\xi_k - a}{\sigma} \right)^2 \sim \chi^2_n$.
Шукатимемо довірчий інтервал з умови $\P\left\{ t_1 < \frac{n \D^* \xi}{\sigma^2} < t_2\right\} = \gamma$, де $t_1$ та 
$t_2$ задовольняють $\P\left\{\frac{n \D^* \xi}{\sigma^2} > t_1 \right\} = \frac{1 + \gamma}{2}$ та
$\P\left\{\frac{n \D^* \xi}{\sigma^2}\geq t_2 \right\} = \frac{1 - \gamma}{2}$. Шуканим довірчим інтервалом буде
$\left( \frac{n}{t_2}(\D^{*}\xi)_{\text{зн}}, \frac{n}{t_1}(\D^{*}\xi)_{\text{зн}} \right)$.

\noindent\textbf{Довірчий інтервал для дисперсії при невідомому математичному сподіванні.}

Як було показано раніше, для $\D^* \xi = \frac{1}{n} \sum\limits_{k=1}^n (\xi_k - \overline{\xi})^2$ статистика 
$\frac{n \D^* \xi}{\sigma^2}$ має розподіл $\chi^2_{n-1}$, тому $\frac{(n-1) \D^{**} \xi}{\sigma^2} \sim \chi^2_{n-1}$.
Як і в попередньому випадку, шукаємо $t_1$ та $t_2$ з умов 
$\P\left\{\frac{(n-1) \D^{**} \xi}{\sigma^2} > t_1 \right\} = \frac{1 + \gamma}{2}$ та
$\P\left\{\frac{(n-1) \D^{**} \xi}{\sigma^2}\geq t_2 \right\} = \frac{1 - \gamma}{2}$.
Шуканий довірчий інтервал ---
$\left( \frac{n-1}{t_2}(\D^{**}\xi)_{\text{зн}}, \frac{n-1}{t_1}(\D^{**}\xi)_{\text{зн}} \right)$.

\subsection{Наближені довірчі інтервали}
Якщо статистика $\theta^*$, що оцінює невідомий параметр, є асимптотично нормальною, то розподіл 
$\frac{\theta^* - \theta}{\sqrt{\D\theta^*}}$ можна вважати приблизно рівним $\mathrm{N}(0, 1)$ (звісно, при достатньо великих $n$).
В такому випадку довірчий інтервал можна будувати з рівності
$\P\left\{\frac{|\theta^* - \theta|}{\sqrt{\D\theta^*}} < t_{\gamma}\right\} = \gamma$, де $t_{\gamma}$
знаходимо з таблиці значень функції Лапласа. Щоб тепер отримати довірчий інтервал, треба розв'язати відносно
$\theta$ нерівність $\frac{|\theta^* - \theta|}{\sqrt{\D\theta^*}} < t_{\gamma}$.
\begin{example}
    Побудуємо наближений довірчий інтервал для $a$ у випадку $\xi \sim \mathrm{Poiss}(a)$. Вважатимемо, що значення $t_\gamma$
    вже знайдемо з рівності 
    $\P\left\{\frac{|a^* - a|}{\sqrt{\D a^*}} < t_{\gamma}\right\} = \gamma$, де $a^* = \overline{\xi}$. Розв'яжемо нерівність:
    \begin{gather*}
        \frac{|a^* - a|}{\sqrt{\D a^*}} < t_{\gamma} \Leftrightarrow
        \frac{(a^* - a)^2}{\D a^*} < t_{\gamma}^2, \; \D a^* = \D \overline{\xi} = \frac{1}{n}\D\xi = \frac{1}{n}{a} \\
        (a^* - a)^2 < a \cdot \frac{t_{\gamma}^2}{n} \Leftrightarrow (a^*)^2 - 2a^* a + a^2 < a \cdot \frac{t_{\gamma}^2}{n}
        \Leftrightarrow a^2 - 2a \left( a^* + \frac{t_{\gamma}^2}{2n}\right) + (a^*)^2 < 0
    \end{gather*}
    Позначимо $a^* + \frac{t_{\gamma}^2}{2n} = b$:
    \begin{gather*}
        a^2 - 2ab + (a^*)^2 < 0, \; D/4 = b^2 - a^2 (a^*)^2 \Rightarrow
        a \in \left(b - \sqrt{D/4}, b + \sqrt{D/4} \right) \\
        b \pm \sqrt{D/4} =  a^* + \frac{t_{\gamma}^2}{2n} \pm 
        \sqrt{\left( a^* + \frac{t_{\gamma}^2}{2n}\right)^2  - (a^*)^2} = 
        a^* + \frac{t_{\gamma}^2}{2n} \pm 
        \sqrt{a^* \cdot\frac{t_{\gamma}^2}{n} + \frac{t_{\gamma}^4}{4n^2}}
    \end{gather*}
    Отже, шуканий довірчий інтервал:
    $$ 
    \left(\overline{\xi} + \frac{t_{\gamma}^2}{2n} -
    \sqrt{\overline{\xi} \cdot\frac{t_{\gamma}^2}{n} + \frac{t_{\gamma}^4}{4n^2}},
    \overline{\xi} + \frac{t_{\gamma}^2}{2n} +
    \sqrt{\overline{\xi} \cdot\frac{t_{\gamma}^2}{n} + \frac{t_{\gamma}^4}{4n^2}}
    \right)
    $$
\end{example}

\subsection{Довірчий інтервал для ймовірності появи події}
За точкову оцінку ймовірності $p$ появи події $A$
в схемі Бернуллі беруть частість $p^* = \frac{m}{n}$, де $n$ ---
загальна кількість незалежних випробувань,
$m$ --- кількість появ події $A$ в цих випробуваннях
Задамо рівень надійності $\gamma$
і знайдемо такі величини $p_1$ та $p_2$, щоб
виконувалось співвідношення $\P\left\{ p_1 < p < p_2\right\} = \gamma$.
Інтервал $(p_1, p_2)$ буде шуканим довірчим інтервалом. Розглянемо два випадки.

\noindent\textbf{Кількість випробувань досить велика.} 

В цьому випадку розподіл величини $m$ в силу граничної теореми Муавра-Лапласа 
можна наближено замінити $\mathrm{N}(np, \sqrt{npq})$, тому
розподіл $p^* = \frac{m}{n}$ приблизно рівний $\mathrm{N}\left(p, \sqrt{\frac{pq}{n}} \right)$.
Таким чином, статистика $\frac{(p^* - p)\sqrt{n}}{\sqrt{pq}}$ має наближено розподіл $\mathrm{N}(0, 1)$.
Користуючись таблицею значень функції Лапласа, для заданої довірчої ймовірності $\gamma$ знайдемо таке $t_{\gamma}$,
при якому 
$\P\left\{\frac{|p^* - p|\sqrt{n}}{\sqrt{pq}} < t_{\gamma}\right\} = \gamma$. Розв'яжемо відносно $p$ нерівність під знаком ймовірності:
\begin{gather*}
    \frac{|p^* - p|\sqrt{n}}{\sqrt{pq}} < t_{\gamma} \Leftrightarrow
    \frac{(p^* - p)^2 n}{p(1-p)} < t^2_{\gamma} \Leftrightarrow
    (p^*)^2 - 2p^* p + p^2 < \left(p - p^2\right)\cdot \frac{t^2_{\gamma}}{n} \Leftrightarrow \\
    \Leftrightarrow
    \left(1 + \frac{t^2_{\gamma}}{n}\right)p^2 - \left(2p^* + \frac{t^2_{\gamma}}{n}\right)p + (p^*)^2 < 0 \Leftrightarrow \\
    \Leftrightarrow
    p \in (p_1, p_2), \; p_{1, 2} = \frac{\left(p^* + \frac{t^2_{\gamma}}{2n}\right) \pm t_{\gamma} \cdot \sqrt{\frac{p^*(1-p^*)}{n} + \frac{t^2_{\gamma}}{4n^2}}}{1 + \frac{t^2_{\gamma}}{n}}
\end{gather*}

Цей результат має геометричну інтерпретацію. Розглянемо систему координат, по осі абсцис якої відкладаємо частість $p^*$, а по осі ординат ---
ймовірність $p$. Повернемося до нерівності, яку вже розв'язали:
\begin{gather*}
    \left(1 + \frac{t^2_{\gamma}}{n}\right)p^2 - 2p^* p + (p^*)^2 - \frac{t^2_{\gamma}}{n}p < 0
\end{gather*}
\begin{tabular}{c p{10.4cm}}
    \begin{tikzpicture}[xscale=2.5, yscale=2.5, baseline={(current bounding box.north)}]
        \fill [gray!20] (0.47, 0.475) circle [x radius=0.7, y radius=0.2, rotate=40];
        \draw [thick] (0.47, 0.475) circle [x radius=0.7, y radius=0.2, rotate=40];
        \draw [->] (0, 0) -- (1.2, 0);
        \draw [->] (0, 0) -- (0, 1.2);
        \draw [dashed] (0, 0.95) -- (0.95, 0.95);
        \draw [dashed] (0.95, 0) -- (0.95, 0.95);
        \node [below] at (1, 0) {$1$};
        \node [left] at (0, 1) {$1$};
        \node [below] at (1.2, 0) {$p^*$};
        \node [left] at (0, 1.2) {$p$};
        \node [below left] at (0, 0) {$0$};
        \draw [dashed] (0.6, 0) -- (0.6, 0.82);
        \draw [dashed] (0.6, 0.82) -- (0, 0.82);
        \draw [dashed] (0.6, 0.325) -- (0, 0.325);
        \node [left] at (0, 0.325) {$p_1$};
        \node [left] at (0, 0.82) {$p_2$};
        \node [below] at (0.6, 0) {$p^*_{\text{зн.}}$};
    \end{tikzpicture} &
    Ця нерівність задає внутрішню частину деякого еліпса.
    Таким чином, довірчим інтервалом для $p$ при відомому значенні частості $p^*_{\text{зн.}}$ буде множина точок всередині цього еліпса з абсцисою,
    що дорівнює $p^*_{\text{зн.}}$. Важливо зауважити, що навіть якщо $p^*_{\text{зн.}} = 1$, то немає підстав казати, що справжнє значення $p$ дорівнює 1.
    В цьому випадку отримаємо довірчий інтервал
    $\left(\frac{1}{1+t_{\gamma}^2/n}, 1\right)$. Аналогічно, при $p^*_{\text{зн.}} = 0$ отримаємо довірчий інтервал
    $\left(0, \frac{t_{\gamma}^2/n}{1+t_{\gamma}^2/n}\right)$.
\end{tabular}
Якщо обсяг вибірки $n$ доволі великий, то величиною $\frac{t_{\gamma}^2}{n}$ можна знехтувати. 
Тоді межі довірчого інтервалу набувають наближених значень 
$$p_{1, 2} \approx p^* \pm t_{\gamma} \sqrt{\frac{p^*(1-p^*)}{n}}$$
\begin{example}
    Деяка подія в серії з $n=100$ незалежних випробувань відбулась $m=78$ разів. Побудувати довірчий інтервал для
    ймовірності $p$ появи цієї події з надійністю $\gamma = 0.9$.
    
    За умовою $p^*_{\text{зн.}} = 0.78$, а відповідне значення $t_{\gamma}$ з таблиці функції Лапласа дорівнює $1.65$.
    Отже,
    \begin{gather*}
        p_{1, 2} = \frac{0.78 + \frac{1.65^2}{200} \pm 1.65 \sqrt{\frac{0.78\cdot 0.22}{100} + \frac{1.65^2}{4\cdot 100^2}}}{1+\frac{1.65^2}{100}} \Rightarrow
        p_1 \approx 0.7047, \; p_2 \approx 0.8404
    \end{gather*}
    Якщо обчислити межі довірчого інтервалу за наближеними формулами для великих $n$, отримаємо
    \begin{gather*}
        p_{1, 2} \approx 0.78 \pm 1.65\sqrt{\frac{0.78\cdot 0.22}{100}} \Rightarrow p_1 \approx 0.7116, \; p_2 \approx 0.8483
    \end{gather*}
\end{example}

\noindent\textbf{Кількість випробувань мала.} 
В цьому випадку граничними теоремами скористатися не вийде. Згадаємо формулу для ймовірності появи події $A$ в схемі Бернуллі
з $n$ іспитами $k$ разів: 
$\P\left\{ \xi = k\right\} = C_n^k p^k (1-p)^{n-k}$, де $\xi\sim \mathrm{Bin}(n, p)$. 
Задамо рівень надійності $\gamma$ та знайдемо такі $p_1$ та $p_2$, що $\P\left\{p_1 < p < p_2 \right\} = \gamma$.
Приймемо без доведення, що $p_1$ --- розв'язок рівняння 
$\sum\limits_{k=0}^{m-1} C_n^k p_1^k (1-p_1)^{n-k} = \frac{1+\gamma}{2}$, а $p_2$ --- розв'язок рівняння
$\sum\limits_{k=0}^{m} C_n^k p_2^k (1-p_2)^{n-k} = \frac{1-\gamma}{2}$.
В цих формулах $m$ є конкретним числом (а не випадковою величиною), кількістю випробувань, в яких сталася подія $A$.
Існують спеціальні таблиці для знаходження значень $p_1$ та $p_2$, що задовольняють цим рівнянням.

\subsection{Довірчі інтервали для параметрів рівномірної ГС}
Як відомо,  асимптотично незміщеними та конзистентними оцінками параметрів $a$ та $b$ рівномірної ГС є, відповідно,
$a^* = \underset{1\leq k \leq n}{\min}\xi_k$ та $b^* = \underset{1\leq k \leq n}{\max}\xi_k$. З огляду на характер цих параметрів,
шукатимемо довірчі інтервали з рівностей
$\P\left\{a\leq a^* \leq a+\varepsilon_1 \right\} \geq \gamma$ та $\P\left\{b-\varepsilon_2\leq b^* \leq b\right\} \geq \gamma$,
де $\gamma$ --- заданий рівень надійності.
Користуючись щільностями розподілу $a^*$ та $b^*$, обчислимо ці ймовірності:
\begin{gather*}
    \P\left\{a\leq a^* \leq a+\varepsilon_1 \right\} = \int_{a}^{a+\varepsilon_1} \frac{n}{(b-a)^n} (b-x)^{n-1} dx = 1 - \left(1 - \frac{\varepsilon_1}{b-a}\right)^n \\
    \P\left\{b-\varepsilon_2\leq b^* \leq b\right\} = \int_{b-\varepsilon_2}^b \frac{n}{(b-a)^n} (x-a)^{n-1} dx = 1 - \left(1 - \frac{\varepsilon_2}{b-a}\right)^n
\end{gather*}
Для різниці $b-a$ з умов $a^* \leq a+\varepsilon_1$ та $b^* \geq b-\varepsilon_2$ маємо оцінку
$b-a \leq \underset{1\leq k \leq n}{\max}\xi_k + \varepsilon_2 - \underset{1\leq k \leq n}{\min}\xi_k + \varepsilon_1$.
Далі для зручності позначатимемо $\underset{1\leq k \leq n}{\max}\xi_k = M$, $\underset{1\leq k \leq n}{\min}\xi_k = m$.
Отримуємо систему:
\begin{gather*}
    \begin{cases}
        1 - \left(1 - \frac{\varepsilon_1}{b-a}\right)^n \geq 1 - \left(1 - \frac{\varepsilon_1}{M - m + \varepsilon_2 + \varepsilon_1}\right)^n = \gamma \\
        1 - \left(1 - \frac{\varepsilon_2}{b-a}\right)^n \geq 1 - \left(1 - \frac{\varepsilon_2}{M - m + \varepsilon_2 + \varepsilon_1}\right)^n = \gamma
    \end{cases}
\end{gather*}
Розв'яжемо її відносно $\varepsilon_1$ та $\varepsilon_2$.
\begin{gather*}
    \begin{cases}
        \left(1 - \frac{\varepsilon_1}{M - m + \varepsilon_2 + \varepsilon_1}\right)^n = 1 - \gamma \\
        \left(1 - \frac{\varepsilon_2}{M - m + \varepsilon_2 + \varepsilon_1}\right)^n = 1 - \gamma
    \end{cases} \Leftrightarrow
    \begin{cases}
        1 - \frac{\varepsilon_1}{M - m + \varepsilon_2 + \varepsilon_1} = \sqrt[\leftroot{-3}\uproot{3}n]{1-\gamma} \\
        1 - \frac{\varepsilon_2}{M - m + \varepsilon_2 + \varepsilon_1} = \sqrt[\leftroot{-3}\uproot{3}n]{1-\gamma}
    \end{cases} \Leftrightarrow \\
    \Leftrightarrow
    \begin{cases}
        \varepsilon_1 = \left(M - m + \varepsilon_2 + \varepsilon_1\right)\left(1 - \sqrt[\leftroot{-3}\uproot{3}n]{1-\gamma}\right) \\
        \varepsilon_2 = \left(M - m + \varepsilon_2 + \varepsilon_1\right)\left(1 - \sqrt[\leftroot{-3}\uproot{3}n]{1-\gamma}\right)
    \end{cases}
\end{gather*}
Отже, $\varepsilon_1 = \varepsilon_2$, тому далі розв'язуємо лише одне рівняння:
\begin{gather*}
    \varepsilon_1 = \left(M - m + 2\varepsilon_1\right) \left(1 - \sqrt[\leftroot{-3}\uproot{3}n]{1-\gamma}\right) \Leftrightarrow
    \left( 2\sqrt[\leftroot{-3}\uproot{3}n]{1-\gamma} - 2 + 1\right)\varepsilon_1 = \left(M - m\right)\left(1 - \sqrt[\leftroot{-3}\uproot{3}n]{1-\gamma}\right)
\end{gather*}
Таким чином, $\varepsilon_1 = \varepsilon_2 = \frac{\left(M - m\right)\left(1 - \sqrt[\leftroot{-3}\uproot{3}n]{1-\gamma}\right)}{2\sqrt[\leftroot{-3}\uproot{3}n]{1-\gamma} - 1}$.
Оскільки $\varepsilon_1, \varepsilon_2 > 0$, то на $\gamma$ треба накладати вимогу $\gamma < 1 - 2^{-n}$.
Шукані довірчі інтервали матимуть вигляд $\left(a^* - \varepsilon_1, a^*\right)$ та $\left(b^*, b^* + \varepsilon_2\right)$.