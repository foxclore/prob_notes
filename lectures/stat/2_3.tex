% !TEX root = ../../main.tex
\subsection{Перевірка складних параметричних гіпотез}
Розглянемо задачу перевірки \emph{простої} основної параметричної гіпотези 
$H_0 : \theta = \theta_0$ проти \emph{складної} альтернативної гіпотези, яка записується у вигляді 
$H_1 : \theta > \theta_0$ або $H_1: \theta < \theta_0$.
В першому випадку задачу називають \emph{перевіркою простої 
основної гіпотези проти правосторонньої альтернативи}, а в другому випадку --- 
\emph{перевіркою простої основної гіпотези проти лівосторонньої альтернативи}.
Має місце теорема, яку іноді називають теоремою Неймана-Пірсона.
\begin{theorem*}
    Побудований на основі відношення правдоподібності критерій перевірки простої параметричної
    гіпотези (за лемою Неймана-Пірсона) буде оптимальним критерієм і для перевірки
    простої параметричної гіпотези проти наведених вище складних 
    альтернативних параметричних гіпотез.
\end{theorem*}

Зробимо пояснення щодо справедливості цієї теореми. Розглянемо просту 
основну параметричну гіпотезу $H_0: \theta = \theta_0$ проти складної альтернативної гіпотези
$H_1: \theta > \theta_0$. Візьмемо довільне $\theta_1 > \theta_0$ та розглянемо альтернативну гіпотезу
$\widetilde{H}_1 : \theta = \theta_1$, яка вже є простою. Для неї за лемою Неймана-Пірсона можна побудувати
найкращу критичну область $W_{C_\alpha}$ для заданого рівня значущості $\alpha$. Оскільки $\theta_1 > \theta_0$,
критична область є правосторонньою. Як було показано в прикладах, межа критичної області від $\theta_1$ не залежить,
тому логічно припустити, що знайдена критична область буде найкращою і для інших значень $\theta_1$ ---
це і є твердженням теореми.
\begin{remark}
    На практиці іноді користуються альтернативною гіпотезою у вигляді $H_1: \theta \neq \theta_0$ ---
    так звану двосторонню альтернативу. Оптимального критерію в такому випадку не існує, а для пошуку лівої та правої
    межі критичної області за заданим рівнем значущості $\alpha$ беруть рівні $\alpha_1 = \alpha_2 = \frac{\alpha}{2}$.
\end{remark}

\subsection{Перевірка гіпотез про рівність параметрів двох ГС}
На практиці часто виникає потреба порівняти результат однієї серії 
випробувань із результатом іншої серії. Виникає питання: чи можна 
розбіжність, що виникла пояснити випадковою помилкою експерименту, чи ця 
розбіжність невипадкова?

\subsubsection*{Перевірка гіпотези про рівність математичних сподівань двох гауссівських 
ГС при відомих дисперсіях}
Маємо дві реалізації обсягами $n_1$ та $n_2$ відповідно двох гауссівських ГС
$\xi \sim \mathrm{N}\left(\E \xi, \sigma_1^2\right)$ та $\eta \sim \mathrm{N}\left(\E \eta, \sigma_2^2\right)$.
Дисперсії цих ГС відомі, а самі ГС --- незалежні. На рівні значущості $\alpha$ треба перевірити
просту основну гіпотезу $H_0: \E \xi = \E\eta$ проти однієї з альтернативних:
$H_1 : \E\xi > \E\eta$ або $H_2 : \E\xi < \E\eta$. Статистикою критерію є
$\zeta = \frac{\overline{\xi} - \overline{\eta}}{\sqrt{\frac{\sigma_1^2}{n_1} + \frac{\sigma_2^2}{n_2}}}$. Вона має
гауссівський розподіл з $\D\zeta = 1$, причому, якщо $H_0$ справджується, то $\E\zeta = 0$.
Критична область є правосторонньою у випадку $H_1: \E\xi > \E\eta$ та лівосторонньою у випадку $H_2: \E\xi < \E\eta$,
її межа знаходиться з таблиці значень функції Лапласа як квантиль $\mathrm{N}(0, 1)$ рівня $\alpha$ для лівосторонньої
критичної області та $1-\alpha$ для правосторонньої.
\begin{example}
    За вибіркою обсягом $n_1 = 14$ знайдемо середній розмір $\overline{x} = 182$ мм діаметра деталей,
    які виготовлено першим верстатом, а за вибіркою $n_2 = 9$ --- середній розмір
    $\overline{y} = 185$ мм деталей, які виготовлено другим верстатом. Розмір діаметра деталі
    має гаусcівський розподіл з дисперсією $\sigma_1^2 = 5$ мм$^2$ для першого верстату
    та $\sigma_2^2 = 7$ мм$^2$ для другого. Чи можна на рівні значущості $0.05$ пояснити різницю вибіркових середніх
    випадковою помилкою?

    Висуваємо основну гіпотезу $H_0 : \E \xi = \E\eta$ проти альтернативної
    $H_1 : \E\xi < \E\eta$, оскільки $\overline{x} < \overline{y}$. Критична область буде лівосторонньою.
    Знайдемо значення статистики критерію:
    $$
    \eta_{\text{зн.}} = \frac{182 - 185}{\sqrt{\frac{5}{14} + \frac{7}{9}}} \approx -2.81
    $$
    Оскільки $t_{\text{кр.}} = -1.64 > \eta_{\text{зн.}}$, то на рівні значущості $0.05$ дані суперечать гіпотезі
    про рівність математичних сподівань двох ГС: тобто, різницю між вибірковими середніми не можна вважати випадковою.
\end{example}

\subsubsection*{Перевірка гіпотези про рівність дисперсій двох гауссівських 
ГС при відомих або невідомих математичних сподіваннях}
Маємо дві реалізації обсягами $n_1$ та $n_2$ відповідно двох гауссівських ГС
$\xi \sim \mathrm{N}\left(\E \xi, \sigma_1^2\right)$ та $\eta \sim \mathrm{N}\left(\E \eta, \sigma_2^2\right)$.
Дисперсії цих ГС невідомі, а самі ГС --- незалежні. На рівні значущості $\alpha$ треба перевірити
просту основну гіпотезу $H_0: \D \xi = \D \eta$ проти однієї з альтернативних:
$H_1 : \D\xi > \D\eta$ або $H_2 : \D\xi < \D\eta$. 

Припустимо, що математичні сподівання $\E\xi$ та $\E\eta$ відомі. Розглянемо дві статистики:
\begin{gather*}
    h_1 = \frac{n_1 \D^* \xi}{\sigma_1^2} \sim \chi^2_{n_1}, \;
    h_2 = \frac{n_2 \D^* \eta}{\sigma_2^2} \sim \chi^2_{n_2}
\end{gather*}
Оскільки $\xi$ та $\eta$ незалежні, то $h_1$ та $h_2$ --- також. Якщо $H_0$ справджується, то
$\gamma = \frac{h_1/n_1}{h_2/n_2} = \frac{\D^*\xi}{\D^*\eta} \sim \mathrm{F}(n_1, n_2)$.

Якщо ж математичні сподівання $\E\xi$ та $\E\eta$ невідомі, то
\begin{gather*}
    h_1 = \frac{(n_1-1) \D^{**} \xi}{\sigma_1^2} \sim \chi^2_{n_1-1}, \;
    h_2 = \frac{(n_2-1) \D^{**} \eta}{\sigma_2^2} \sim \chi^2_{n_2-1}
\end{gather*}
Якщо $H_0$ справджується, то
$\gamma = \frac{h_1/(n_1-1)}{h_2/(n_2-1)} = \frac{\D^{**}\xi}{\D^{**}\eta} \sim \mathrm{F}(n_1-1, n_2-1)$.

В обох випадках критична область є правосторонньою чи лівосторонньою в залежності від альтернативної гіпотези,
як і у попередніх прикладах. Межа критичної області береться з таблиці 
квантилів розподілу Фішера-Снедекора. Нагадаємо, що за властивістю цього розподілу
$\frac{1}{\gamma} \sim \mathrm{F}(n_2, n_1)$ (або $\mathrm{F}(n_2-1, n_1-1)$).

\begin{example}
    На двох токарних верстатах обробляються деталі. Відібрані дві проби: з деталей, 
    що зроблені на першому верстаті, $n_1 = 15$, на другому верстаті --- 
    $n_2 = 18$. По даним цих вибірок обчислені значення виправлених вибіркових 
    дисперсій $(\D^{**} \xi)_\text{зн} = 8.5$ (для першого верстата) для контролюючого розміру 
    деталей та $(\D^{**} \eta)_\text{зн} = 6.3$ (для другого верстата). Контролюючий розмір деталі, 
    що виготовляється верстатами, має гауссівський розподіл. На рівні значущості 
    0.05 з’ясувати, чи можна вважати, що верстати мають різну точність.

    Маємо нульову гіпотезу $H_0$ : $\D \xi = \D \eta$, тобто дисперсії 
    контролюючого розміру деталі на різних верстатах є рівними. 
    Альтернативну гіпотезу варто брати $H_1$ : $\D \xi > \D \eta$, бо значення 
    виправленої вибіркової дисперсії для першого верстата більше. 
    Статистикою критерію беруть випадкову величину:
    $$\gamma = \frac{h_1 / (n_1 - 1)}{h_2 / (n_2 - 1)} = \frac{\D^{**}\xi}
    {\D^{**} \eta} \sim F(n_1 - 1, n_2 - 1)$$

    В чисельник записують більшу з оцінок виправленої вибіркової дисперсії (завжди!)
    , бо це дає можливість, враховуючи властивості F --- розподілу, 
    скоротити обсяг його табличних значень у два рази. 
    Значення статистики критерія $\gamma_\text{зн} \sim 1.37$. 
    За таблицею значень F - розподілу на рівні значущості 0,05 при кількості 
    ступенів вільності $n_1 - 1 = 14$ та $n_2 - 1 = 17$ знаходимо $t_\text{кр} = 2.33$. 
    Згідно 
    конкуруючий гіпотезі, критична область знаходиться справа від $t_\text{кр}$, 
    тому робимо висновок, оскільки $\gamma_\text{зн} < t_\text{кр}$, що дослідні 
    дані на рівні значущості 
    0.05 не протирічать основній гіпотезі. А це означає, що дослідні дані 
    не дозволяють вважати, що верстати мають різну точність.
\end{example}

\subsubsection*{Перевірка гіпотези про рівність математичних сподівань двох 
гауссівських ГС при при невідомих, але рівних дисперсій}

Маємо дві реалізації обсягами $n_1$ та $n_2$ відповідно двох гауссівських 
генеральних сукупностей $\xi \sim N(\E \xi, \frac{\sigma^2}{n_1})$ та 
$\eta \sim N(\E \eta, \frac{\sigma^2}{n_2})$. Дисперсії цих генеральних 
сукупностей невідомі, але рівні. На рівні значущості $\alpha$ потрібно перевірити 
просту основну гіпотезу $H_0$ : $\E \xi = \E \eta$ проти однієї з конкуруючих 
гіпотез $H_1$ : $\E \xi > \E \eta$ або $H_2$ : $\E \xi < \E \eta$.

Оскільки дисперсії невідомі, то ми можемо побудувати тільки їх точкові оцінки:
$$h_1 = \frac{(n_1 - 1)\D^{**}\xi}{\sigma_1^2} \sim \chi^2_{n_1 - 1} 
\text{ та } h_2 = \frac{(n_2 - 1)\D^{**}\xi}{\sigma_2^2} \sim \chi^2_{n_2 - 1}$$

Тоді $h = h_1 + h_2 \sim \chi_{n_1 - 1}^2$ за властивостями розподілу хі-квадрат.

Статистика критерію в цьому випадку буде такою:
$$\gamma = \frac{\overline{\xi} - \overline{\eta}}
{\sqrt{\frac{n_1 + n_2}{n_1 n_2}}
\sqrt{\frac{(n_1 - 1)\D^{**}\xi + (n_2 - 1)\D^{**}\eta}{n_1 + n_2 - 2}}} 
\sim {St}_{n_1 + n_2 - 2}$$

Тобто маємо розподіл Стьюдента з $n_1 + n_2 - 2$ ступенями вільності. Критична 
область береться правостороння чи лівостороння, в залежності від конкуруючої 
гіпотези. Межа критичної області береться з таблиці значень розподілу Стьюдента. 
Гіпотезу про рівність дисперсій слід перевіряти заздалегідь, як правило, 
на тому ж самому рівні значущості.

\begin{example}
    Витрати сировини на одиницю продукції за старою технологією склав:
    \begin{center}
        \begin{tabular}{|l|l|l|l|}
            \hline
            Витрати сировини & $304$ & $307$ & $308$ \\
            \hline
            Кількість виробів & $1$ & $4$ & $4$ \\
            \hline
        \end{tabular}
    \end{center}
    За новою технологією:
    \begin{center}
        \begin{tabular}{|l|l|l|l|l|}
            \hline
            Витрати сировини & $303$ & $304$ & $306$ & $308$ \\
            \hline
            Кількість виробів & $2$ & $6$ & $4$ & $1$ \\
            \hline
        \end{tabular}
    \end{center}
    Припустив, що відповідні генеральні сукупності мають гауссівський розподіл, 
    на рівні значущості 0.05 відповісти на питання: чи дає нова технологія 
    економію середньої витрати сировини на один виріб?
    
\end{example}