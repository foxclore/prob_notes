% !TEX root = ../../main.tex
\subsection{Перевірка складних параметричних гіпотез}
Розглянемо задачу перевірки \emph{простої} основної параметричної гіпотези 
$H_0 : \theta = \theta_0$ проти \emph{складної} альтернативної гіпотези, яка записується у вигляді 
$H_1 : \theta > \theta_0$ або $H_1: \theta < \theta_0$.
В першому випадку задачу називають \emph{перевіркою простої 
основної гіпотези проти правосторонньої альтернативи}, а в другому випадку --- 
\emph{перевіркою простої основної гіпотези проти лівосторонньої альтернативи}.
Має місце теорема, яку іноді називають теоремою Неймана-Пірсона.
\begin{theorem*}
    Побудований на основі відношення правдоподібності критерій перевірки простої параметричної
    гіпотези (за лемою Неймана-Пірсона) буде оптимальним критерієм і для перевірки
    простої параметричної гіпотези проти наведених вище складних 
    альтернативних параметричних гіпотез.
\end{theorem*}

Зробимо пояснення щодо справедливості цієї теореми. Розглянемо просту 
основну параметричну гіпотезу $H_0: \theta = \theta_0$ проти складної альтернативної гіпотези
$H_1: \theta > \theta_0$. Візьмемо довільне $\theta_1 > \theta_0$ та розглянемо альтернативну гіпотезу
$\widetilde{H}_1 : \theta = \theta_1$, яка вже є простою. Для неї за лемою Неймана-Пірсона можна побудувати
найкращу критичну область $W_{C_\alpha}$ для заданого рівня значущості $\alpha$. Оскільки $\theta_1 > \theta_0$,
критична область є правосторонньою. Як було показано в прикладах, межа критичної області від $\theta_1$ не залежить,
тому логічно припустити, що знайдена критична область буде найкращою і для інших значень $\theta_1$ ---
це і є твердженням теореми.
\begin{remark}
    На практиці іноді користуються альтернативною гіпотезою у вигляді $H_1: \theta \neq \theta_0$ ---
    так звану двосторонню альтернативу. Оптимального критерію в такому випадку не існує, а для пошуку лівої та правої
    межі критичної області за заданим рівнем значущості $\alpha$ беруть рівні $\alpha_1 = \alpha_2 = \frac{\alpha}{2}$.
\end{remark}

\subsection{Перевірка гіпотез про рівність параметрів двох ГС}
На практиці часто виникає потреба порівняти результат однієї серії 
випробувань із результатом іншої серії. Виникає питання: чи можна 
розбіжність, що виникла пояснити випадковою помилкою експерименту, чи ця 
розбіжність невипадкова?

\subsubsection*{Перевірка гіпотези про рівність математичних сподівань двох гауссівських 
ГС при відомих дисперсіях}
Дано дві реалізації обсягами $n_1$ та $n_2$ відповідно двох гауссівських ГС
$\xi \sim \mathrm{N}\left(\E \xi, \sigma_1^2\right)$ та $\eta \sim \mathrm{N}\left(\E \eta, \sigma_2^2\right)$.
Дисперсії цих ГС відомі, а самі ГС --- незалежні. На рівні значущості $\alpha$ треба перевірити
просту основну гіпотезу $H_0: \E \xi = \E\eta$ проти однієї з альтернативних:
$H_1 : \E\xi > \E\eta$ або $H_2 : \E\xi < \E\eta$. Статистикою критерію є
$\zeta = \frac{\overline{\xi} - \overline{\eta}}{\sqrt{\frac{\sigma_1^2}{n_1} + \frac{\sigma_2^2}{n_2}}}$. Вона має
гауссівський розподіл з $\D\zeta = 1$, причому, якщо $H_0$ справджується, то $\E\zeta = 0$.
Критична область є правосторонньою у випадку $H_1: \E\xi > \E\eta$ та лівосторонньою у випадку $H_2: \E\xi < \E\eta$,
її межа знаходиться з таблиці значень функції Лапласа як квантиль $\mathrm{N}(0, 1)$ рівня $\alpha$ для лівосторонньої
критичної області та $1-\alpha$ для правосторонньої.
\begin{example}
    За вибіркою обсягом $n_1 = 14$ знайдемо середній розмір $\overline{x} = 182$ мм діаметра деталей,
    які виготовлено першим верстатом, а за вибіркою $n_2 = 9$ --- середній розмір
    $\overline{y} = 185$ мм деталей, які виготовлено другим верстатом. Розмір діаметра деталі
    має гаусcівський розподіл з дисперсією $\sigma_1^2 = 5$ мм$^2$ для першого верстату
    та $\sigma_2^2 = 7$ мм$^2$ для другого. Чи можна на рівні значущості $0.05$ пояснити різницю вибіркових середніх
    випадковою помилкою?

    Висуваємо основну гіпотезу $H_0 : \E \xi = \E\eta$ проти альтернативної
    $H_1 : \E\xi < \E\eta$, оскільки $\overline{x} < \overline{y}$. Критична область буде лівосторонньою.
    Знайдемо значення статистики критерію:
    $$
    \eta_{\text{зн}} = \frac{182 - 185}{\sqrt{\frac{5}{14} + \frac{7}{9}}} \approx -2.81
    $$
    Оскільки $t_{\text{кр}} = -1.64 > \eta_{\text{зн}}$, то на рівні значущості $0.05$ дані суперечать гіпотезі
    про рівність математичних сподівань двох ГС: тобто, різницю між вибірковими середніми не можна вважати випадковою.
\end{example}

\subsubsection*{Перевірка гіпотези про рівність дисперсій двох гауссівських 
ГС при відомих або невідомих математичних сподіваннях}
Дано дві реалізації обсягами $n_1$ та $n_2$ відповідно двох гауссівських ГС
$\xi \sim \mathrm{N}\left(\E \xi, \sigma_1^2\right)$ та $\eta \sim \mathrm{N}\left(\E \eta, \sigma_2^2\right)$.
Дисперсії цих ГС невідомі, а самі ГС --- незалежні. На рівні значущості $\alpha$ треба перевірити
просту основну гіпотезу $H_0: \D \xi = \D \eta$ проти однієї з альтернативних:
$H_1 : \D\xi > \D\eta$ або $H_2 : \D\xi < \D\eta$. 

Припустимо, що математичні сподівання $\E\xi$ та $\E\eta$ відомі. Розглянемо дві статистики:
\begin{gather*}
    h_1 = \frac{n_1 \D^* \xi}{\sigma_1^2} \sim \chi^2_{n_1}, \;
    h_2 = \frac{n_2 \D^* \eta}{\sigma_2^2} \sim \chi^2_{n_2}
\end{gather*}
Оскільки $\xi$ та $\eta$ незалежні, то $h_1$ та $h_2$ --- також. Якщо $H_0$ справджується, то
$\gamma = \frac{h_1/n_1}{h_2/n_2} = \frac{\D^*\xi}{\D^*\eta} \sim \mathrm{F}(n_1, n_2)$.

Якщо ж математичні сподівання $\E\xi$ та $\E\eta$ невідомі, то
\begin{gather*}
    h_1 = \frac{(n_1-1) \D^{**} \xi}{\sigma_1^2} \sim \chi^2_{n_1-1}, \;
    h_2 = \frac{(n_2-1) \D^{**} \eta}{\sigma_2^2} \sim \chi^2_{n_2-1}
\end{gather*}
Якщо $H_0$ справджується, то
$\gamma = \frac{h_1/(n_1-1)}{h_2/(n_2-1)} = \frac{\D^{**}\xi}{\D^{**}\eta} \sim \mathrm{F}(n_1-1, n_2-1)$.

В обох випадках критична область є правосторонньою чи лівосторонньою в залежності від альтернативної гіпотези,
як і у попередніх прикладах. Межа критичної області береться з таблиці 
квантилів розподілу Фішера-Снедекора. Нагадаємо, що за властивістю цього розподілу
$\frac{1}{\gamma} \sim \mathrm{F}(n_2, n_1)$ (або, відповідно, $\mathrm{F}(n_2-1, n_1-1)$).

\begin{example}
    На двох токарних верстатах обробляються деталі. Відібрані дві проби: з деталей, 
    що зроблені на першому верстаті, $n_1 = 15$, на другому верстаті --- 
    $n_2 = 18$. За обома вибірками обчислені значення виправлених вибіркових 
    дисперсій розміру деталей: $(\D^{**} \xi)_\text{зн} = 8.5$ мм$^2$ для першого верстата 
    та $(\D^{**} \eta)_\text{зн} = 6.3$ мм$^2$ для другого. Розмір деталі, 
    що виготовляється верстатами, має гауссівський розподіл. На рівні значущості 
    $0.05$ з'ясувати, чи можна вважати, що верстати мають різну точність.

    Маємо нульову гіпотезу $H_0: \D \xi = \D \eta$ --- дисперсії 
    розміру деталі на різних верстатах є рівними. 
    В якості альтернативної гіпотези варто брати $H_1: \D \xi > \D \eta$, бо значення 
    виправленої вибіркової дисперсії для першого верстата більше. 
    Статистика критерію $\gamma$, якщо $H_0$ справджується, має розподіл $\mathrm{F}(n_1-1, n_2-1) = \mathrm{F}(14, 17)$. 
    Значення статистики $\gamma_\text{зн} = \frac{(\D^{**} \xi)_\text{зн}}{(\D^{**} \eta)_\text{зн}} \approx 1.37$. 
    За таблицею квантилів розподілу Фішера-Снедекора на рівні значущості $0.05$ при кількості 
    ступенів вільності $n_1 - 1 = 14$ та $n_2 - 1 = 17$ знаходимо $t_\text{кр} = 2.33$. 
    Згідно з
    альтернативною гіпотезою, критична область знаходиться справа від $t_\text{кр}$
    (тобто, є правосторонньою),тому робимо висновок, що дослідні 
    дані на рівні значущості $0.05$ не суперечать основній гіпотезі,
    оскільки $\gamma_\text{зн} < t_\text{кр}$. Це означає, що дослідні дані 
    не дозволяють вважати, що верстати мають різну точність.
\end{example}

\subsubsection*{Перевірка гіпотези про рівність математичних сподівань двох 
гауссівських ГС при невідомих, але рівних дисперсіях}

Дано дві реалізації обсягами $n_1$ та $n_2$ відповідно двох гауссівських ГС
$\xi \sim \mathrm{N}\left(\E \xi, \sigma_1^2\right)$ та $\eta \sim \mathrm{N}\left(\E \eta, \sigma_2^2\right)$.
Дисперсії цих ГС невідомі, але рівні ($\sigma_1 = \sigma_2 = \sigma$), а самі ГС --- незалежні. На рівні значущості $\alpha$ потрібно перевірити 
просту основну гіпотезу $H_0 : \E \xi = \E \eta$ проти однієї з альтернативних 
гіпотез $H_1 : \E \xi > \E \eta$ або $H_2 : \E \xi < \E \eta$.

Оскільки дисперсії невідомі, то ми можемо побудувати тільки їх точкові оцінки $\D^{**}\xi$ та
$\D^{**}\eta$, тоді
$$h_1 = \frac{(n_1 - 1)\D^{**}\xi}{\sigma^2} \sim \chi^2_{n_1 - 1}, \; h_2 = \frac{(n_2 - 1)\D^{**}\eta}{\sigma^2} \sim \chi^2_{n_2 - 1}$$

Тоді $h = h_1 + h_2 = \frac{1}{\sigma^2}\left((n_1-1)\D^{**}\xi + (n_2-1)\D^{**}\eta\right) \sim \chi_{n_1 + n_2 - 2}^2$ за властивостями розподілу хі-квадрат.
Якщо $H_0: \E\xi = \E\eta$ справджується, то 
$\zeta = \frac{\overline{\xi} - \overline{\eta}}{\sqrt{\frac{\sigma^2}{n_1} + \frac{\sigma^2}{n_2}}} =
\frac{\overline{\xi} - \overline{\eta}}{\sigma\sqrt{\frac{n_1 + n_2}{n_1 n_2}}} \sim \mathrm{N}(0, 1)$.

В якості статистики критерію береться
$$\gamma = \frac{\zeta}{\sqrt{h/(n_1 + n_2 - 2)}} = 
\frac{\overline{\xi} - \overline{\eta}}
{\sqrt{\frac{n_1 + n_2}{n_1 n_2}}
\sqrt{\frac{(n_1 - 1)\D^{**}\xi + (n_2 - 1)\D^{**}\eta}{n_1 + n_2 - 2}}} 
\sim {St}_{n_1 + n_2 - 2}$$

Критична область є правосторонньою чи лівосторонньою в залежності від альтернативної гіпотези.
Межа критичної області береться з таблиці квантилів розподілу Стьюдента. 
\begin{remark}
    Якщо про рівність дисперсій заздалегідь нічого не відомо, треба перевіряти гіпотезу про їх рівність
    на тому ж самому рівні значущості.
\end{remark}

\begin{example}
    Є дані про витрати сировини на одиницю продукції за старою технологією та за новою:
    \begin{center}
        \begin{tabular}{|c|c|c|c|c|c|c|c|}
            \hline
            {} & \multicolumn{3}{c|}{стара} & \multicolumn{3}{c|}{нова} \\
            \hline
            Витрати сировини & $304$ & $307$ & $308$ & $303$ & $304$ & $306$ \\
            \hline
            Кількість виробів & $1$ & $4$ & $4$ & $2$ & $6$ & $4$ \\
            \hline
        \end{tabular}
    \end{center}
    Припустивши, що відповідні генеральні сукупності мають гауссівський розподіл, 
    на рівні значущості $0.05$ відповісти на питання: чи дає нова технологія 
    економію середньої витрати сировини на один виріб?

    Потрібно перевірити гіпотезу про рівність математичних сподівань. Оскільки дисперсії невідомі, перевірити
    також треба гіпотезу про їх рівність. Знайдемо значення вибіркових середніх та виправлених вибіркових дисперсій,
    позначивши $\xi$ ГС, що відповідає старій технології, а $\eta$ --- новій:
    \begin{gather*}
        \left(\overline{\xi}\right)_\text{зн} = \overline{x} = \frac{1}{9}\left(304\cdot 1 + 307\cdot 4 + 308\cdot 4\right) \approx 307.11 \\
        (\D^{**} \xi)_\text{зн} = \frac{1}{9-1}\left((304 - 307.11)^2 \cdot 1 + (307 - 307.11)^2 \cdot 4 + (308 - 307.11)^2 \cdot 4\right) \approx 1.611 \\
        \left(\overline{\eta}\right)_\text{зн} = \overline{y} \approx 304.5, \; (\D^{**} \eta)_\text{зн} \approx 1.364
    \end{gather*} 
    Розглянемо основну гіпотезу $H_0 : \D\xi = \D\eta$ та альтернативну $H_1 : \D\xi > \D\eta$, бо
    $(\D^{**} \xi)_\text{зн} > (\D^{**} \eta)_\text{зн}$. Критична область правостороння, за таблицею для $\mathrm{F}(8, 11)$
    знаходимо її межу --- $t_\text{кр} = 2.95$, тобто, критична область --- $(2.95; +\infty)$. 
    Значення статистики критерію для перевірки гіпотези про рівність дисперсій
    $\frac{(\D^{**} \xi)_\text{зн}}{(\D^{**} \eta)_\text{зн}} = \frac{1.611}{1.364} \approx 1.181 < t_\text{кр}$.
    Отже, гіпотеза про рівність дисперсій не відхиляється на рівні значущості $0.05$.
    \begin{exercise}
        Перевірити, що якби у вибірці, що відповідає новій технології, був один виріб з витратою сировини 308, то гіпотезу про 
        рівність дисперсій довелося б відхилити.
    \end{exercise}
    Перейдемо до перевірки гіпотези про рівність математичних сподівань $H_0: \E\xi = \E\eta$ проти 
    альтернативної $H_1 : \E\xi > \E\eta$, бо $\overline{x} > \overline{y}$. Статистикою для перевірки буде
    \begin{gather*}
        \gamma = \frac{\overline{\xi} - \overline{\eta}}
        {\sqrt{\frac{n_1 + n_2}{n_1 n_2}}
        \sqrt{\frac{(n_1 - 1)\D^{**}\xi + (n_2 - 1)\D^{**}\eta}{n_1 + n_2 - 2}}} 
         = \frac{\overline{\xi} - \overline{\eta}}
        {\sqrt{\frac{9 + 12}{9\cdot 12}}
        \sqrt{\frac{(9 - 1)\D^{**}\xi + (12 - 1)\D^{**}\eta}{9 + 12 - 2}}} 
        \sim {St}_{n_1 + n_2 - 2} = {St}_{19}
    \end{gather*}
    $\gamma_\text{зн} \approx 4.885$. За таблицею квантилів розподілу Стьюдента
    знайдемо $t_\text{кр} = 1.729$, причому критична область є правосторонньою.
    Отже, на рівні значущості $0.05$ гіпотеза про рівність математичних сподівань відхиляється,
    тому можна вважати, що нова технологія дійсно дає економію середньої кількості сировини 
    для виготовлення одного виробу.
\end{example}

\begin{remark}
    Якщо дисперсії двох ГС невідомі і немає підстав вважати їх рівними, має місце
    \emph{проблема Беренса-Фішера}. В такому випадку також застосовується статистика, що має
    розподіл Стьюдента, але з кількістю ступенів вільності (з округленням до найближчого цілого)
    \begin{gather*}
        k = (n_1 + n_2 - 2)\cdot\left(\frac{1}{2} + \frac{
            (\D^{**} \xi)_\text{зн} (\D^{**} \eta)_\text{зн}
        }{
            (\D^{**} \xi)_\text{зн}^2 + (\D^{**} \eta)_\text{зн}^2
        }\right) 
    \end{gather*}
    Розглянемо другий множник. Оскільки
    $\frac{
        (\D^{**} \xi)_\text{зн} (\D^{**} \eta)_\text{зн}
    }{
        (\D^{**} \xi)_\text{зн}^2 + (\D^{**} \eta)_\text{зн}^2
    } \leq \frac{1}{2}$ та рівність досягається при $(\D^{**} \xi)_\text{зн} = (\D^{**} \eta)_\text{зн}$,
    то зі збільшенням різниці між значенням виправлених вибіркових дисперсій цей доданок зменшується.
    Якщо ці значення розрізняються досить суттєво, то кількість степенів свободи
    зменшується майже вдвічі порівняно з $n_1 + n_2 - 2$.
    Це, в свою чергу, звужує критичну область (та, відповідно, ймовірність потрапити туди, оскільки
    вона розміщена на <<хвостах>> розподілу). До того ж, розкид можливих значень
    $\overline{x} - \overline{y}$ буде визначатися в основному тією ГС, яка має більшу дисперсію,
    тому дані з ГС з меншою дисперсією <<губляться>>. Ці два фактори разом призводять
    до більшої недовіри висновку щодо основної гіпотези, ніж у випадку відомих дисперсій.
\end{remark}

\subsubsection*{Перевірка гіпотези про рівність ймовірностей біномних генеральних 
сукупностей при великому обсязі вибірки}
Дано дві реалізації обсягами $n_1$ та $n_2$ відповідно двох незалежних ГС $\xi$ та $\eta$, що мають розподіл Бернуллі, тобто
$\xi \sim \mathrm{Bin}(1, p_1)$, $\eta \sim \mathrm{Bin}(1, p_2)$,
причому обсяги цих вибірок достатньо великі, щоб можна було застосувати асимптотичну
нормальність оцінки ймовірності <<успіху>> в одному випробуванні. На рівні значущості
$\alpha$ потрібно перевірити основну гіпотезу $H_0: p_1 = p_2$ проти альтернативної
$H_1: p_1 > p_2$ або $H_2: p_1 < p_2$. Нехай $m_1$ та $m_2$ --- кількості успішних випробувань
у кожній з реалізацій. Тоді за ЦГТ
$\frac{m_1}{n_1} = p_1^*$ та $\frac{m_2}{n_2} = p_2^*$ наближено мають розподіли
$\mathrm{N}\left(p_1, \frac{p_1 q_1}{n_1}\right)$ та $\mathrm{N}\left(p_2, \frac{p_2 q_2}{n_2}\right)$ відповідно,
а $p_1^* - p_2^*$, якщо $H_0$ справджується --- розподіл $\mathrm{N}\left(0, \frac{p_1 q_1}{n_1} + \frac{p_2 q_2}{n_2}\right)$.
Також, якщо $H_0$ справджується, то можна переписати $\frac{p_1 q_1}{n_1} + \frac{p_2 q_2}{n_2}$
як $p q \left(\frac{1}{n_1} + \frac{1}{n_2}\right)$, тому в якості статистики для перевірки основної гіпотези беруть
\begin{gather*}
    \gamma = \frac{p_1^* - p_2^*}{\sqrt{p^* (1 - p^*) \left(\frac{1}{n_1} + \frac{1}{n_2}\right)}},
    \; \text{де } p^* = \frac{m_1 + m_2}{n_1 + n_2}
\end{gather*}
Оскільки вона має наближено розподіл $\mathrm{N}(0, 1)$, то критичне значення знаходять
за таблицею значень функції Лапласа як квантиль рівня $\alpha$ для лівосторонньої критичної області
та рівня $1-\alpha$ для правосторонньої. Критична область береться правосторонньою чи лівосторонньою
в залежності від альтернативної гіпотези.

\begin{example}
    \begin{enumerate}
        \item Порівняти частку браку на рівні значущості $0.1$ для двох партій виробів:
        
        \begin{center}
            \begin{tabular}{|c|c|c|c|}
                \hline
                номер партії & обсяг вибірки & к-сть бракованих & частка бракованих \\
                \hline
                1 & $n_1 = 200$ & $m_1 = 5$ & $(p_1^*)_\text{зн} = \frac{m_1}{n_1} = \frac{1}{40}$ \\ 
                \hline
                2 & $n_2 = 300$ & $m_2 = 10$ & $(p_2^*)_\text{зн} = \frac{m_2}{n_2} = \frac{1}{30}$ \\
                \hline 
            \end{tabular}
        \end{center}

        Потрібно перевірити гіпотезу $H_0: p_1 = p_2$ проти альтернативної
        $H_1: p_1 < p_2$. Обсяги обох вибірок великі, тому розподіли $p_1^*$ та
        $p_2^*$ можна наближувати нормальним. $(p^*)_\text{зн} = \frac{5 + 10}{200 + 300} = \frac{3}{100}$, тому
        $\gamma_\text{зн} = \frac{\frac{1}{40} - \frac{1}{30}}{
            \sqrt{\frac{3}{100} \frac{97}{100} \left(\frac{1}{200} + \frac{1}{300}\right)}
            } \approx -0.535$. Критичне значення --- $t_\text{кр} = -1.28$.
            Оскільки критична область є лівосторонньою, то на рівні значущості $0.1$ гіпотеза
            про однакову частку бракованих виробів не відхиляється.
            \item Контрольну роботу з теорії ймовірностей виконували студенти двох груп другого курсу ІПСА.
            Першій групі викладач запропонував 105 задач, з яких правильно було розв'язано 60, а другій --- 140 задач, з яких правильно
            розв'язаних вийшло 69. На рівні значущості $0.02$ перевірити гіпотезу про відсутність різниці в засвоєнні дисципліни 
            <<Теорія ймовірностей>> студентами різних груп.

            Нехай $p_1$ та $p_2$ --- ймовірності розв'язання задачі студентами першої та другої групи відповідно. Потрібно перевірити
            гіпотезу $H_0: p_1 = p_2$ проти альтернативної $H_1: p_1 > p_2$, оскільки
            $(p_1^*)_\text{зн} = \frac{m_1}{n_1} = \frac{60}{105} \approx 0.571$,
            $(p_2^*)_\text{зн} = \frac{m_2}{n_2} = \frac{69}{140} \approx 0.493$.
            Також, $(p^*)_\text{зн} = \frac{60 + 69}{105 + 140} \approx 0.527$.
            Тому 
            $\gamma_\text{зн} = \frac{0.571 - 0.493}{\sqrt{0.527\cdot(1-0.527)\cdot\left(\frac{1}{105} + \frac{1}{140}\right)}} \approx 1.21$.
            Критична область є правосторонньою, її межа $t_\text{кр} = 2.05 > \gamma_\text{зн}$.
            Отже, на рівні значущості $0.02$ дослідні дані не суперечать гіпотезі про те, що різниці в засвоєнні
            дисципліни студентами немає.
    \end{enumerate}
\end{example}