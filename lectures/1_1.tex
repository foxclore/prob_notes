% !TEX root = ../main.tex

\section{Основні поняття теорії ймовірності}
\subsection*{Стохастичний експеримент (СЕ)}
\begin{definition}
    Стохастичним експериментом (випробуванням) називається експеримент, 
    який можна повторювати неодноразово зберігаючи певні умови і результат цього 
    експерименту заздалегідь передбачити неможливо.
\end{definition}
\begin{definition}
    Будь-який результат СЕ називається подією.
\end{definition}
\begin{example}
    СЕ - кидання кубика, подія - випало 6 очок.
\end{example}

\textbf{Події бувають:}
\begin{enumerate}
    \item Випадкові (можуть відбутися чи не відбутися при проведенні СЕ)
    \item Неможливі (ніколи не відбуваються при проведенні СЕ)
    \item Вірогідні (завжди відбуваються при проведенні СЕ)
\end{enumerate}
\subsection*{Теоретико-множинний підхід до основних понять ТЙ}

%qw