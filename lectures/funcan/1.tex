% !TEX root = ../main.tex

Для подальшої зрозумілості матеріалу необхідне базові знання функціонального аналізу і, 
в особливості, теорії міри. 

\subsection*{Аксіоматика}

Нехай вже маємо довільний певним чином побудований ймовірнісний простір 
$\{\Omega, \mathcal{F}, \P \}$ деякого стохастичного експерименту, що задовольняє 
аксіоматиці Колмогорова. Схожого вигляду структури використовуються в іншому розділі 
математики - теорії міри. Апарат теорії міри дозволяє побудувати більш концептуально 
логічну та зрозумілу теорію ймовірностей, яку в подальшому більш зручно використовувати в 
науковій математичній діяльності. Почнемо із вправи.
\begin{exercise}
    Довести, що ймовірнісний простір $\{\Omega, \mathcal{F}, \P \}$ є простором з 
    мірою (тобто $\mathcal{F}$ --- алгебра або $\sigma$-алгебра підмножин в 
    $\Omega$, $\P$ : $\Omega \rightarrow \mathbb{R}$ --- міра).
\end{exercise}
Таким чином до ймовірнісного простору можна застосовувати апарат теорії міри для 
отримання нових для нас результатів та знаходження неочікуваних 
взаємозв'язків між вже існуючими поняттями при цьому перетворюючи апарат теорії 
ймовірностей до більш зручного вигляду.

\subsection*{Теореми про збіжність}

