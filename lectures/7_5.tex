% !TEX root = ../main.tex
\section{Збіжність та граничні теореми для послідовностей випадкових векторів}
В цьому розділі наведемо без доведення деякі означення та факти, що стосуються збіжності
послідовностей, закону великих чисел та центральної граничної теореми для випадкових векторів.
Розглядаємо послідовності випадкових векторів $\left\{ \vec{\xi}_n (\omega)\right\}_{n=1}^{\infty}$ на фіксованому
ймовірносному просторі.
\subsection{Збіжність випадкових векторів}
\index{збіжність!випадкових векторів}
На послідовності випадкових векторів цілком природно переносяться означення збіжності за ймовірністю, за розподілом та з ймовірністю 1.
\begin{enumerate}
    \item Збіжність з ймовірністю 1: $\left(\vec{\xi}_n \overset{\mathrm{P1}}{\longrightarrow} \vec{\xi}, n \to \infty\right) \Leftrightarrow 
    \left(\P\left\{ \omega: \underset{n\to\infty}{\lim} \vec{\xi}_n(\omega) = \vec{\xi}(\omega)\right\} = 1\right)$.
    \item Збіжність за ймовірністю: $\left(\vec{\xi}_n \overset{\mathrm{P}}{\longrightarrow} \vec{\xi}, n \to\infty\right) \Leftrightarrow
    \left(\forall \; \varepsilon > 0: \underset{n \to \infty}{\lim} \P\left\{\left\Vert\vec{\xi}_n - \vec{\xi}\right\Vert \geq \varepsilon\right\}= 0\right)$.
    \item Збіжність за розподілом: 
    
    $\left(\vec{\xi}_n \overset{\mathrm{F}}{\longrightarrow} \vec{\xi}, n \to\infty\right) \Leftrightarrow
    \left(\underset{n\to\infty}{\lim}F_{\vec{\xi}_n}(\vec{x}) = F_{\vec{\xi}}(\vec{x}) \text{ для точок неперервності } F_{\vec{\xi}}\right)$.
\end{enumerate}
Зв'язки між цими видами збіжності такі ж, як у випадку випадкових величин.

\subsection{Граничні теореми}
Для простоти розглядаємо лише випадок \emph{незалежних однаково розподілених} $\vec{\xi}_n$.
\begin{theorem*}[посилений закон великих чисел]\label{multivar_lln}\index{закон великих чисел!посилений}
    Нехай $\left\{ \vec{\xi}_n\right\}_{n=1}^{\infty}$ --- послідовність незалежних однаково розподілених випадкових векторів, що мають скінченне
    математичне сподівання $\vec{a}$ (тобто $\Vert \vec{a} \Vert < \infty$). Тоді 
    $$
    \frac{1}{n} \sum\limits_{k=1}^n \vec{\xi}_k \overset{\mathrm{P1}}{\longrightarrow} \vec{a}, \; n \to \infty
    $$
\end{theorem*}
\begin{theorem*}[центральна гранична теорема]\index{центральна гранична теорема}
    Нехай $\left\{ \vec{\xi}_n\right\}_{n=1}^{\infty}$ --- послідовність незалежних однаково розподілених випадкових векторів, що мають скінченні
    математичне сподівання $\vec{a}$ та кореляційну матрицю $K$. Тоді
    $$
    \sqrt{n}\left( \frac{1}{n} \sum\limits_{k=1}^n \vec{\xi}_k - \vec{a}\right) = 
    \frac{1}{\sqrt{n}} \sum\limits_{k=1}^n \left( \vec{\xi_k} - \vec{a}\right) \overset{\mathrm{F}}{\longrightarrow} \vec{\eta} \sim \mathrm{N}\left(\vec{0}, K\right), \; n\to\infty
    $$
\end{theorem*}