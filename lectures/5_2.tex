% !TEX root = ../main.tex

\section{Функції кількох випадкових аргументів}
\subsection{Випадок довільної функції}
Нехай $\varphi : \mathbb{R}^n \to \mathbb{R}$ --- задана числова функція.

Якщо $\vec{\xi} = \left(\xi_1, ..., \xi_n\right)$ --- дискретний випадковий вектор, тоді $\eta = \varphi(\vec{\xi})$ --- ДВВ.
Побудову закону розподілу $\eta$ доцільно розглянути на прикладі.
\begin{example}
    $\vec{\xi} = \left( \xi_1, \xi_2\right)$ задано таблицею розподілу:
    \begin{tabular}{|c|c|c|c|}
        \hline
        \diagbox{$\xi_2$}{$\xi_1$} & $0$ & $1$ & $2$ \\
        \hline
        $-1$ & $0.1$ & $0.2$ & $0.3$ \\
        \hline
        $1$ & $0.2$ & $0.1$ & $0.1$ \\
        \hline
    \end{tabular}.

    \noindentЗнайти закони розподілу $\eta_1 = \xi_1 \xi_2$ та $\eta_2 = \xi_1 + \xi_2$.
    Для цього треба визначити значення, які приймають ці величини, та обчислити відповідні ймовірності.
    
    $\eta_1$ приймає значення $-2$ (коли $\xi_1 = 2$, $\xi_2 = -1$), 
    $-1$ (коли $\xi_1 = 1$, $\xi_2 = -1$), 
    $0$ (коли $\xi_1 = 0$, $\xi_2 = -1$ або $\xi_2 = 1$), 
    $1$ (коли $\xi_1 = 1$, $\xi_2 = 1$), 
    $2$ (коли $\xi_1 = 2$, $\xi_2 = 1$).

    $\eta_2$ приймає значення $-1$ (коли $\xi_1 = 0$, $\xi_2 = -1$), 
    $0$ (коли $\xi_1 = 1$, $\xi_2 = -1$), 
    $1$ (коли $\xi_1 = 0$, $\xi_2 = 1$ або $\xi_1 = 2$, $\xi_2 = -1$), 
    $2$ (коли $\xi_1 = 1$, $\xi_2 = 1$), 
    $3$ (коли $\xi_1 = 2$, $\xi_2 = 1$).
    Відповідні сумісні ймовірності отримуємо з таблиці з таблиці розподілу $\vec{\xi}$.

    \begin{tabular}{|c|c|c|c|c|c|}
        \hline
        $\eta_1$ & $-2$ & $-1$ & $0$ & $1$ & $2$ \\
        \hline
        $p$ & $0.3$ & $0.2$ & $0.3$ & $0.1$ & $0.1$ \\
        \hline
    \end{tabular}
    \begin{tabular}{|c|c|c|c|c|c|}
        \hline
        $\eta_2$ & $-1$ & $0$ & $1$ & $2$ & $3$ \\
        \hline
        $p$ & $0.1$ & $0.2$ & $0.5$ & $0.1$ & $0.1$ \\
        \hline
    \end{tabular}
\end{example}

Якщо $\vec{\xi} = \left(\xi_1, ..., \xi_n\right)$ --- неперервний випадковий вектор
із щільністю $f_{\vec{\xi}} (\vec{x})$, то можна знайти функцію розподілу $\eta = \varphi(\vec{\xi})$.
$$F_\eta (y) = P \left\{ \eta < y\right\} = P \left\{ \xi \in D_y\right\} = \underset{D_y}{\int ... \int} f_{\vec{\xi}} (\vec{x}) d \vec{x}, \text{ де }D_y = \left\{\vec{x} \in \mathbb{R}^n : \varphi(\vec{x}) < y\right\}$$

Розглянемо тепер взаємно однозначне гладке перетворення $\vec{\psi} : \mathbb{R}^n \to \mathbb{R}^n$ та
знайдемо щільність розподілу $\vec{\eta} = \vec{\psi} (\vec{\xi})$. Для множини $D \subset \mathbb{R}^n$
$P\left\{ \vec{\psi} (\vec{\xi}) \in D\right\} = P\left\{ \vec{\xi} \in \vec{\psi}^{-1}(D)\right\} = \int_{\vec{\psi}^{-1}(D)} f_{\vec{\xi}} (\vec{x}) d\vec{x} = 
\left[ \text{заміна }\vec{y} = \vec{\psi}(\vec{x})\right] = \int_D f_{\vec{\xi}} (\vec{\psi}^{-1}(\vec{y})) \left| \mathcal{J}^{-1} (\vec{\psi}^{-1}(\vec{y})\right| d\vec{y}$,
де $\mathcal{J} (\vec{x})$ --- якобіан $\vec{\psi}$. Отже,
$$f_{\vec{\eta}} (\vec{y}) = f_{\vec{\xi}} (\vec{\psi}^{-1}(\vec{y})) \left| \mathcal{J}^{-1} (\vec{\psi}^{-1}(\vec{y})\right|$$

\begin{example}
    Нехай $A$ --- невироджена матриця розміру $n \times n$, $\vec{b} \in \mathbb{R}^n$ --- деякий вектор, $\vec{\xi}$ --- неперервний випадковий вектор.
    Знайти щільність розподілу $\vec{\eta} = A \vec{\xi} + \vec{b}$.

    Тут $\vec{y} = \vec{\psi}(\vec{x}) = A \vec{x} + \vec{b}$, тому $\vec{\psi}^{-1} (\vec{y}) = A^{-1} (\vec{y} - \vec{b})$. Якобіан $\vec{\psi}$ рівний $\left| \det A\right|$. 
    Отже,
    $f_{\vec{\eta}}(\vec{y}) = \frac{1}{\left| \det A\right|} f_{\vec{\xi}}\left(A^{-1} (\vec{y} - \vec{b})\right)$
\end{example}

\subsection{Закон розподілу мінімуму та максимуму}

Нехай $\xi_1, ..., \xi_n$ незалежні та розподілені однаково, як деяка випадкова величина $\xi$.
Знайдемо розподіл їх мінімуму та максимуму.
\begin{enumerate}
    \item $\mu_1 = \min\left\{\xi_1, \xi_2, ..., \xi_n\right\}$.

    $F_{\mu_1} (x) = P \left\{ \min\{\xi_1, \xi_2, ..., \xi_n\} < x \right\} =
    1 - P \left\{ \min\{\xi_1, \xi_2, ..., \xi_n\} \geq x \right\} =$

    $ = 1 - P \left\{ \xi_1 \geq x, \xi_2 \geq x, ..., \xi_n \geq x \right\} = 
    1 - P\left\{ \xi_1 \geq x\right\} \cdot P\left\{ \xi_2 \geq x\right\} \cdot ... \cdot P\left\{ \xi_n \geq x\right\} = $
    
    $ = 1 - (P\left\{ \xi \geq x\right\})^n = 1 - (1- F_\xi (x))^n$.

    Якщо $\xi_1, ..., \xi_n$ --- неперервні, то $f_{\mu_1} (x) = \left( F_{\mu_1} (x)\right)^\prime = n (1- F_\xi (x))^{n-1} f_\xi(x)$.
    \item $\mu_2 = \max\left\{\xi_1, \xi_2, ..., \xi_n\right\}$.

    $F_{\mu_2} (x) = P \left\{ \max\{\xi_1, \xi_2, ..., \xi_n\} < x \right\} =
    P \left\{ \xi_1 < x, \xi_2 < x, ..., \xi_n < x \right\} = $

    $ = P\left\{ \xi_1 < x\right\} \cdot P\left\{ \xi_2 < x\right\} \cdot ... \cdot P\left\{ \xi_n < x\right\} =
    (P\left\{ \xi < x\right\})^n = (F_\xi (x))^n$.

    Якщо $\xi_1, ..., \xi_n$ --- неперервні, то $f_{\mu_2} (x) = \left( F_{\mu_2} (x)\right)^\prime = n (F_\xi (x))^{n-1} f_\xi(x)$.
\end{enumerate}

\begin{example}
    Нехай $\xi_1, ..., \xi_n$ незалежні та мають розподіл $\mathrm{Exp}(\lambda)$. Знайти розподіл їх мінімуму.

    \noindent$f_{\min}(x) = n (1-(1-e^{-\lambda x}))^{n-1} \lambda e^{-\lambda x} = n \lambda e^{-n\lambda}$ при $x \geq 0$ та $0$ при $x < 0$.
    
    \noindentОтже, $\min\{\xi_1, ..., \xi_n\} \sim \mathrm{Exp} (n \lambda)$.
\end{example}

\subsection{Закон розподілу добутку двох НВВ}

\subsection{Закон розподілу частки двох НВВ}

\subsection{Закон розподілу суми двох НВВ}