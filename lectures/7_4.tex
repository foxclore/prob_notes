% !TEX root = ../main.tex
\section{Застосування ЦГТ до схеми Бернуллі}
\subsection{Інтегральна теорема Муавра-Лапласа}
\begin{theorem*}
    Нехай $\xi_n$ задає кількість успіхів в схемі Бернуллі з ймовірністю успіху $p$, тобто $\xi_n \sim \mathrm{Bin}(n, p)$,
    $q = 1-p$.
    Тоді рівномірно відносно $x$:
    \begin{gather}
        \lim_{n \rightarrow \infty} P \left\{
            \frac{\xi_n - np}{\sqrt{npq}}
            < x
        \right\} = F_{\mathrm{N}(0, 1)}(x) = \Phi(x) + \frac{1}{2} = 
        \frac{1}{\sqrt{2\pi}} \int\limits_0^x e^{-\frac{t^2}{2}} dt + \frac{1}{2}
    \end{gather}
\end{theorem*}
\begin{proof}
    $\xi_n = \sum\limits_{k=1}^n \eta_k$, де $\eta_k$ --- індикатор появи успіху в $k$-тому випробуванні,
    $\eta_k \sim \mathrm{Bin}(1, p)$. $\eta_k$ --- незалежні та однаково розподілені, $E\eta_k = p$, $D\eta_k = pq$,
    тому до них можна застосувати ЦГТ:
    \begin{gather*}
        \sum\limits_{k=1}^n \left( \eta_k - E\eta_k\right) = \xi_n - np, \; \sum\limits_{k=1}^n D\eta_k = n p q \\
        \frac{\sum\limits_{k=1}^n \left( \eta_k - E\eta_k\right)}{\sqrt{\sum\limits_{k=1}^n D\eta_k}} =
         \frac{\xi_n - np}{\sqrt{n p q}} \overset{\mathrm{F}}{\longrightarrow} \xi \sim \mathrm{N}(0, 1), \; n\to \infty
    \end{gather*}
\end{proof}
\noindent\textbf{Робочі формули.} При великих $n$:
\begin{enumerate}
    \item $P\left\{ \frac{\xi_n - np}{\sqrt{n p q}} < x\right\} \approx \Phi(x) + \frac{1}{2}$.
    \item $\forall \; a, b \in \mathbb{R}, \; a<b : P\left\{a < \frac{\xi_n - np}{\sqrt{n p q}} < b\right\} \approx \Phi(b) - \Phi(a)$.
    \item $\forall \; m_1, m_2 \in \mathbb{R}, m_1<m_2 : P\left\{ m_1 < \xi_n < m_2\right\} = 
    P\left\{\frac{m_1 - np}{\sqrt{n p q}} < \frac{\xi_n - np}{\sqrt{n p q}} < \frac{m_2 - np}{\sqrt{n p q}}\right\} \approx$

    $\approx \Phi\left(\frac{m_2 - np}{\sqrt{n p q}}\right) - \Phi\left(\frac{m_1 - np}{\sqrt{n p q}}\right)$.
    \item $\forall \; \varepsilon > 0 : P\left\{ \left|\frac{\xi_n}{n} - np\right| < \varepsilon\right\} = 
    P\left\{-\varepsilon < \frac{\xi_n - np}{n} < \varepsilon\right\} =
    P\left\{-\varepsilon \sqrt{\frac{n}{pq}} < \frac{\xi_n - np}{\sqrt{n p q}} < \varepsilon \sqrt{\frac{n}{pq}}\right\} \approx$
    $\approx \Phi\left( \varepsilon \sqrt{\frac{n}{pq}}\right) - \Phi\left( -\varepsilon \sqrt{\frac{n}{pq}}\right) = 2\Phi\left( \varepsilon \sqrt{\frac{n}{pq}}\right)$.
\end{enumerate}
\begin{remark}
    В усіх наведених наближених рівностях будь-яку строгу нерівність можна замінити на нестрогу. На практиці формулі 3 іноді застосовують \emph{<<поправку на неперервність>>},
    використовуючи $m_2 + \frac{1}{2}$ та $m_1 - \frac{1}{2}$ замість $m_2$ та $m_1$ відповідно. Емпірично з'ясовано, що це дозволяє підвищити точність наближення.
    Також важливо зауважити, що на практиці ці формули застосовують при $n \geq 50$ та $npq \geq 10$ (як і для наступних формул, 
    різні джерела можуть давати інші межі, наприклад $npq \geq 20$).
\end{remark}
\begin{example}
    Стрілець робить 100 пострілів по мішені з ймовірністю влучення $0.7$, $\xi$ --- кількість влучень. Знайти наближено 
    $P\left\{ 67 \leq \xi \leq 72\right\}$ та порівняти з точним значенням.
    
    \noindent Маємо $np = 70$, $npq = 100 \cdot 0.7 \cdot 0.3 = 21 \geq 10$, $P\left\{ 67 \leq \xi \leq 72\right\} \approx 
    \Phi\left( \frac{2}{\sqrt{21}}\right) - \Phi\left(-\frac{3}{\sqrt{21}}\right) \approx 0.4124$.
    З поправкою на неперервність отримаємо 
    $P\left\{ 67 \leq \xi \leq 72\right\} \approx 
    \Phi\left( \frac{2.5}{\sqrt{21}}\right) - \Phi\left(-\frac{3.5}{\sqrt{21}}\right) \approx 0.4848$.
    Підрахунок за точною формулою $\sum\limits_{k=67}^{72} C_{100}^{k} 0.7^k 0.3^{100-k}$ дає значення $0.4829$.
\end{example}

\subsection{Локальна теорема Муавра-Лапласа}
\begin{theorem*}
    Нехай проводиться $n$ незалежних випробувань з ймовірністю успіху $p$, $q = 1-p$, причому $n$ досить велике.
    Тоді ймовірність отримання $m$ успіхів 
    \begin{gather}
        P_n(m) \approx \frac{1}{\sqrt{2\pi n p q}} \exp \left\{ - \frac{(m - np)^2}{2npq}\right\}
    \end{gather}
\end{theorem*}
\begin{proof}
    Нехай $\xi$ задає кількість успіхів у заданій схемі Бернуллі, тоді за інтегральною теоремою Муавра-Лапласа
    $P_n(m) = P\left\{ m-\frac{1}{2} < \xi < m+\frac{1}{2}\right\} \approx 
    \Phi\left(\frac{m+\frac{1}{2}- np}{\sqrt{n p q}}\right) - \Phi\left(\frac{m-\frac{1}{2} - np}{\sqrt{n p q}}\right)$.
    За означенням похідної $\Phi(x + \varepsilon) - \Phi(x - \varepsilon) \approx 2 \varepsilon \Phi'(x)$ при малих $\varepsilon$,
    причому $\Phi'(x) = \frac{1}{\sqrt{2\pi}} \exp \left\{ - \frac{x^2}{2}\right\}$.
    Звідси 
    $\Phi\left(\frac{m+\frac{1}{2}- np}{\sqrt{n p q}}\right) - \Phi\left(\frac{m-\frac{1}{2} - np}{\sqrt{n p q}}\right) \approx
    \frac{1}{\sqrt{2\pi n p q}} \exp \left\{ - \frac{(m - np)^2}{2npq}\right\}$.
\end{proof}
\begin{remark}
    Так само, як інтегральна теорема, локальна теорема Муавра-Лапласа на практиці застосовується при $n \geq 50$ та $npq \geq 10$ (або $npq \geq 20$).
\end{remark}
\begin{example}
    Стрілець робить $100$ пострілів по мішені з ймовірністю влучення $0.7$, $\xi$ --- кількість влучень. Знайти наближено
    $P\left\{ \xi = 69\right\}$ та порівняти з точним значенням.

    \noindent Маємо $npq = 21$, $P\left\{ \xi = 69\right\} \approx \frac{1}{\sqrt{42\pi}} \exp\left\{ -\frac{1}{42}\right\} \approx 0.085$.
    Підрахунок за точною формулою $C_{100}^{69} 0.7^{69} 0.3^{31}$ дає значення $0.084$.
\end{example}

\subsection{Гранична теорема Пуассона}
Розглянемо граничу теорему для схеми Бернуллі, яка застосовується для наближених обчислень, коли умова $npq \geq 10$ 
для застосування локальної теореми Муавра-Лапласа не виконується.
\begin{theorem*}
    Задано нескінченну серію схем Бернуллі: перша складається з одного випробування $B_1$ з імовірністю успіху $p_1$, 
    друга --- з двох випробувань $B_2^1$ та $B_2^2$ з імовірністю успіху $p_2$, 
    третя --- з трьох випробувань $B_3^1$, $B_3^2$ та $B_3^3$ з імовірністю успіху $p_3$ і так далі.
    Випадкова величина $\xi_n \sim \mathrm{Bin}(n, p_n)$ --- кількість успіхів в $n$-тій схемі Бернуллі.
    Існує таке $a > 0$, що $\underset{n\to\infty}{\lim} n p_n = a$.
    Тоді для будь-якого цілого $m \geq 0$:
    \begin{gather}
        P\left\{\xi_n = m\right\} \to \frac{a^m}{m!} e^{-a}, \; n\to \infty
    \end{gather}
\end{theorem*}
\begin{proof}
    Нехай $\xi \sim \mathrm{Poiss}(a)$. Покажемо $\xi_n \overset{\mathrm{F}}{\longrightarrow} \xi, n\to\infty$, скориставшись твердженням теореми Леві (\ref{levi_theor}).
    Характеристичні функції $\xi_n$ та $\xi$ рівні, відповідно, $\chi_{\xi_n}(t) = \left(p_n e^{it} + 1 - p_n \right)^n$ та $\chi_{\xi}(t) = e^{a\left( e^{it} - 1\right)}$.
    З умови $\underset{n\to\infty}{\lim} n p_n = a$ та нерівності $\left| z^n - w^n\right| \leq n\cdot\left| z - w\right|$ для $z, w \in \mathbb{C}$, $|z|\leq1$, $|w|\leq1$:
    $$
    \left| \left(p_n e^{it} + 1 - p_n \right)^n - \left(\frac{a}{n} e^{it} + 1 - \frac{a}{n} \right)^n\right| \leq
    n \cdot \left| p_n - \frac{a}{n}\right| \left| e^{it} - 1\right| \to 0, \; n\to\infty
    $$
    Тому маємо
    $$
    \underset{n\to\infty}{\lim} \chi_{\xi_n}(t) = \underset{n\to\infty}{\lim} \left(\frac{a}{n} e^{it} + 1 - \frac{a}{n} \right)^n = 
    \underset{n\to\infty}{\lim} \left(1 + \frac{a}{n} \left(e^{it} - 1\right) \right)^n = e^{a\left( e^{it} - 1\right)} = \chi_{\xi}(t)
    $$
    Отже, $\xi_n \overset{\mathrm{F}}{\longrightarrow} \xi, n\to\infty$. Функція розподілу $\xi$ неперервна в будь-якій нецілій точці, тому для цілих $m \geq 0$:
    \begin{gather*}
        P\left\{ m - \frac{1}{2} \leq \xi_n < m + \frac{1}{2}\right\} = F_{\xi_n}\left( m + \frac{1}{2}\right) -
        F_{\xi_n}\left( m - \frac{1}{2}\right) \to \\ \to F_{\xi}\left( m + \frac{1}{2}\right) -
        F_{\xi}\left( m - \frac{1}{2}\right) = P\left\{ m - \frac{1}{2} \leq \xi < m + \frac{1}{2}\right\}
        P\left\{ \xi = m\right\} = \frac{a^m}{m!} e^{-a}, \; n\to \infty
    \end{gather*}
    що і треба було довести.
\end{proof}
\begin{exercise}
    Довести цю теорему без використання характеристичних функцій, показавши
    $$C_n^m p_n^m (1-p_n)^{n-m} \to \frac{a^m}{m!}e^{-a}, n\to\infty$$
\end{exercise}
На практиці цією теоремою користуються, якщо $n \geq 50$ та $np \leq 10$ 
(або $n$ --- достатньо велике, $p$ --- достатньо мале, причому $1 < np < 20$):
ймовірність отримати $m$ успіхів у схемі Бернуллі з $n$ випробуваннями та ймовірністю успіху $p$
$P_n(m) \approx \frac{(np)^m}{m!} e^{-np}, \; m = 0, ..., n$.
\begin{example}
    Стрілець робить $100$ пострілів по мішені. 
    Ймовірність влучення при одному пострілі становить $0.98$. 
    Знайти наближено ймовірність того, що вдалих пострілів буде не більше $97$, та порівняти з точним значенням.
    
    \noindentЩоб звести задачу до використання теореми Пуассона, шукатимемо ймовірність, що промахів буде щонайменше $3$ (ймовірність
    промаху --- $0.02$). $\xi$ --- кількість промахів,
    $P\left\{ \xi \geq 3\right\} = 1 - P\left\{ \xi \leq 2\right\} = 
    1 - p_{100}(0) - p_{100}(1) - p_{100}(2)
    \approx 1 - \left( \frac{2^0}{0!} + \frac{2^1}{1!} + \frac{2^2}{2!}\right)e^{-2} = 1 - 5e^{-2} \approx 0.32332$.
    Підрахунок за точною формулою $1 - \sum\limits_{k=0}^2 C_{100}^k 0.02^k 0.98^{100-k}$ дасть значення $0.32331$. Нескладно
    перевірити, що застосування локальної теореми Муавра-Лапласа дало б менш точну відповідь $0.36049$.
\end{example}