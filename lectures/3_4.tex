% !TEX root = ../main.tex

\section{Умовні закони розподілу ВВ}

Розглядаємо випадок $n = 2$, $\vec{\xi} = \left(\xi_1, \xi_2\right)^T$.

\begin{definition}
    \emph{Умовним законом розподілу} $\xi_1$ називається закон розподілу 
    $\xi_1$ за умови того, що $\xi_2$ набула відповідного значення (ДВВ) 
    або потрапила в деякий проміжок (НВВ) (для $\xi_2$ аналогічно).
\end{definition}

\begin{definition}
    Універсальним умовним законом розподілу ВВ є \emph{умовна функція 
    розподілу}:
    \begin{equation*}
        F_{\xi_1}(x/y) = P({\xi_1 < x}/{\xi_2 < y}) = 
        \frac{P\left\{\xi_1 < x, \xi_2 < y\right\}}
        {P\left\{\xi_2 < y\right\}} = 
        \frac{F_{\vec{\xi}}(x, y)}{F_{\xi_2}(y)}
    \end{equation*}
\end{definition}
\begin{remark}
    З означення випливає, що 
    $F_{\vec{\xi}}(x,y) = F_{\xi_1}(x)F_{\xi_2}(y/x)$
    та 
    $F_{\vec{\xi}}(x,y) = F_{\xi_2}(y)F_{\xi_1}(x/y)$.
\end{remark}
\subsection{Умовний закон розподілу дискретного випадкового вектора}
Знову розглядаємо випадок $n=2$ та $\vec{\xi} = (\xi_1, \xi_2)^T$.

Умовні розподіли координат задаються $P\left\{\xi_1 = x_i / \xi_2 = y_j\right\} = 
\frac{P\left\{\xi_1 = x_i , \xi_2 = y_j\right\}}
{P\left\{\xi_2 = y_j\right\}}$ та \\ 
$P\left\{\xi_2 = y_j / \xi_1 = x_i\right\} = 
\frac{P\left\{\xi_1 = x_i , \xi_2 = y_j\right\}}
{P\left\{\xi_1 = x_i\right\}}$.

\begin{example}
    Дискретний випадковий вектор має закон розподілу:

    \begin{tabular}{ccc}
        \begin{tabular}{|c|c|c|c|}
            \hline
            \diagbox{$\xi_2$}{$\xi_1$} & $-1$ & $0$ & $1$\\
            \hline
            0 & $0.1$ & $0.2$ & $0.1$ \\
            \hline
            $1$ & $0.2$ & $0.3$ & $0.1$ \\
            \hline
        \end{tabular}
        &
        \begin{tabular}{|c|c|c|c|}
            \hline
            $\xi_1$ & $-1$ & $0$ & $1$ \\
            \hline
            $p$ & $0.3$ & $0.5$ & $0.2$ \\
            \hline
        \end{tabular}
        &
        \begin{tabular}{|c|c|c|}
            \hline
            $\xi_2$ & $0$ & $1$ \\
            \hline
            $p$ & $0.4$ & $0.5$ \\
            \hline
        \end{tabular}
    \end{tabular}
\end{example}

\begin{tabular}{|c|c|c|c|}
    \hline
    $\xi_1$ & $-1$ & $0$ & $1$ \\
    \hline
    $P\left\{\xi_1 / \xi_2 = 0\right\}$ & $1/4$ 
    & $2/4$ & $1/4$ \\
    \hline
    $P\left\{\xi_1 / \xi_2 = 1\right\}$ & $2/6$ 
    & $3/6$ & $1/6$ \\
    \hline
\end{tabular}
--- умовний закон розподілу $\xi_1$.

\begin{tabular}{|c|c|c|}
    \hline
    $\xi_2$ & $0$ & $1$ \\
    \hline
    $P\left\{\xi_2 / \xi_1 = -1\right\}$ & $1/3$ 
    & $2/3$ \\
    \hline
    $P\left\{\xi_2 / \xi_1 = 0\right\}$ & $2/5$ 
    & $3/5$\\
    \hline
    $P\left\{\xi_2 / \xi_1 = 1\right\}$ & $1/2$ 
    & $1/2$\\
    \hline
\end{tabular}
--- умовний закон розподілу $\xi_2$.

\subsection{Умовне математичне сподівання дискретного випадкового вектора}
\begin{definition}
    \emph{Умовним математичним сподіванням} випадкової величини $\xi_1$ 
    є математичне сподівання цієї випадкової величини за умови, що 
    $\xi_2$ набула певного значення.
    $$E(\xi_1 / \xi_2 = y_j) =
\sum\limits_{i=1}^{n(\infty)}x_i 
P\left\{\xi_1 = x_i / \xi_2 = y_j\right\}, 
E(\xi_2 / \xi_1 = x_i) = \sum\limits_{j=1}^{n(\infty)}y_j 
P\left\{\xi_2 = y_j / \xi_1 = x_i\right\}$$
\end{definition}

\begin{remark}
    Якщо $\xi_1$ та $\xi_2$ незалежні, то $E(\xi_1 / \xi_2 = y_j) 
    = E\xi_1$, $E(\xi_2 / \xi_1 = x_i) = E\xi_2$.
\end{remark}

Умовні математичні сподівання координат дискретного випадкового вектора є дискретними випадковими величинами, оскільки 
приймають декілька значень з певними ймовірностями, тому можна скласти 
їх закон розподілу.
\begin{example}
    Продовження попереднього прикладу:

    \begin{tabular}{c c}
        \begin{tabular}{|c|c|c|}
            \hline
            $E(\xi_1 / \xi_2)$ & $-1/6$ & $0$ \\
            \hline
            $p$ & $0.6$ & $0.4$ \\
            \hline
        \end{tabular}
        &
        \begin{tabular}{|c|c|c|c|}
            \hline
            $E(\xi_2 / \xi_1)$ & $1/3$ & $3/5$ 
            & $1/2$ \\
            \hline
            $p$ & $0.3$ & $0.5$ & $0.2$ \\
            \hline
        \end{tabular}
    \end{tabular}
\end{example}

Крім умовного математичного сподівання вводяться так звані 
\emph{умовні дисперсії} (міра розсіювання однієї випадкової величини 
за умови того, що інша набула певне значення)

\begin{equation*}
    D(\xi_1 / \xi_2 = y_j) := 
    E((\xi_1 - E(\xi_1 / \xi_2 = y_j))/\xi_2 = y_j)
\end{equation*}

\subsection{Умовні закони розподілу неперервних випадкових величин}

У випадку $n = 2$, $\vec{\xi} = (\xi_1, \xi_2)^T$: 
\begin{equation*}
    P\{x \leq \xi_1 < x + \Delta x / y \leq \xi_2 < y + \Delta y\} = 
    \frac{P\{\vec{\xi} \in \Pi\}}{P\{y \leq \xi_2 \leq y+\Delta y\}} = 
    \frac{f_{\vec{\xi}}(x,y)\Delta x \Delta y + 
    o(\sqrt{\Delta x^2 + \Delta y^2})}{f_{\xi_2}(y)\Delta y + o(\Delta y)}
\end{equation*}
\begin{definition}
    \emph{Умовною щільністю розподілу} $\xi_1$ називається функція 
    вигляду:
    \begin{equation*}
        f_{\xi_1}(x/y) = \frac{f_{\vec{\xi}}(x, y)}{f_{\xi_2}(y)}
    \end{equation*}
    Для $\xi_2$ - аналогічно.
\end{definition}

\begin{remark}
    Умовну щільність розподілу можна інтерпретувати як лінію перетину 
    поверхні ймовірностей та площини $y = y_{\text{знач.}}$ або
    $x = x_{\text{знач.}}$ відповідно.
\end{remark}

Умовні щільності мають властивості, аналогічні безумовним щільностям.

\noindent \textbf{Властивості}
\begin{enumerate}
    \item $f_{\xi_1}(x / y) \geq 0$
    \item $f_{\xi_1}(x/y) = \frac{f_{\vec{\xi}}(x,y)}
    {\int\limits_{-\infty}^{+\infty}f_{\vec{\xi}}(x,y)dx}$
    \item $\int\limits_{-\infty}^{+\infty} f_{\xi_1}(x / y) dx = 
    \int\limits_{-\infty}^{+\infty}\frac{f_{\vec{\xi}}(x,y)}
    {\int\limits_{-\infty}^{+\infty}f_{\vec{\xi}}(x,y)dx}dx = 1$
\end{enumerate}
Для $\xi_2$ --- аналогічно.

\subsection{Умовне математичне сподівання неперервного випадкового вектора}
\noindent \textbf{Формули}

$E(\xi_1 / \xi_2) = \int\limits_{-\infty}^{+\infty} xf_{\xi_1}(x/y)dx$

$E(\xi_2 / \xi_1) = \int\limits_{-\infty}^{+\infty} yf_{\xi_2}(y/x)dx$

\begin{definition}
    Функції $\phi(y) = E(\xi_1 / \xi_2 = y)$ та 
    $\psi(x) = E(\xi_2 / \xi_1 = x)$ називаються лініями регресії.
\end{definition}
\subsection{Формули повного матсподівання та дисперсії}
\begin{proposition}[формула повного матсподівання]
    $E(E(\xi_1 / \xi_2)) = E\xi_1$, $E(E(\xi_2 / \xi_1)) = E\xi_2$
\end{proposition}
\begin{proof}
    Дискретний випадок:

    $E(E(\xi_1 / \xi_2)) = \sum\limits_{j = 1}^m 
    \left(
        \sum\limits_{i=1}^n x_i P\{\xi_1 = x_i / \xi_2 = y_j\}
    \right) P\{\xi_2 = y_j\} = 
    \sum\limits_{i=1}^n x_i P\{\xi_1 = x_i\} = E\xi_1$

    Неперервний випадок:

    $E(E(\xi_1 / \xi_2)) = \int\limits_{-\infty}^{+\infty} 
    \left(
        \int\limits_{-\infty}^{+\infty} x f_{\xi_1}(x / y) dx
    \right) f_{\xi_2}(y) dy 
    = \int\limits_{-\infty}^{+\infty} x f_{\xi_1}(x) dx = E\xi_1$

    $\xi_2$ --- аналогічно.
\end{proof}

В літературі часто використовується назва "Adam's law".
\begin{example}
    Продовження минулого прикладу:

    $E\xi_1 = -0.3 + 0.2 = -0.1$, $E(E(\xi_1 / \xi_2)) = -\frac{1}{6}
    \cdot0.6 = -0.1$

    $E\xi_2 = 0.6$, $E(E(\xi_2 / \xi_1)) = 0.3\cdot\frac{2}{3} + 
    \frac{1}{2}\cdot\frac{3}{5} + \frac{1}{2} \cdot\frac{2}{10} = 
    0.6$
\end{example}

\begin{proposition}[формула повної дисперсії]
    Нехай $\xi_1$, $\xi_2$ - випадкові величини та $D\xi_1 < \infty$. 
    Тоді:
    $D\xi_1 = E(D(\xi_1/\xi_2)) + D(E(\xi_1 / \xi_2))$.
\end{proposition}

\begin{proof}
    $D\xi_1 = E\xi_1^2 - (E \xi_1)^2 = 
    E(E(\xi_1^2 / \xi_2)) - (E(E(\xi_1/\xi_2)))^2 = 
    E(D(\xi_1/\xi_2) + (E(\xi_1/\xi_2))^2) - (E(E(\xi_1/\xi_2)))^2 = 
    E(D(\xi_1 / \xi_2)) + (E(E(\xi_1/\xi_2)^2) - 
    (E(E(\xi_1 / \xi_2)))^2) = E(D(\xi_1/\xi_2)) + D(E(\xi_1/\xi_2))$
\end{proof}


В літературі часто використовується назва "Eve's law".

\subsection{Незалежність координат випадкового вектора}

$n=2$, $\vec{\xi} = \left(\xi_1, \xi_2\right)^T$

Необхідною і достатньою умовою незалежності, як вже було розглянуто, є 
$F_{\vec{\xi}}(x, y) = F_{\xi_1}(x)F_{\xi_2}(y)$

З цього випливає те, що $F_{\xi_1}(x/y) = F_{\xi_1}(x)$, 
$F_{\xi_2}(y/x) = F_{\xi_2}(y)$.

\noindent\textbf{Дискретний випадок: }

$P\{\xi_1 = x_i, \xi_2 = y_j\} = P\{\xi_1 = x_i\}P\{\xi_2 = y_j\}$ 
$\forall i,j$
З цього випливає, що:

\begin{enumerate}
    \item $P\{\xi_1 = x_i / \xi_2 = y_j\} = 
    P\{\xi_1 = x_i\}P\{\xi_2 = y_j\}$
    \item $E(\xi_1 / \xi_2 = y_j) = E\xi_1$
    
    $E(\xi_2 / \xi_1 = x_i) = E\xi_2$
\end{enumerate}

\noindent\textbf{Неперервний випадок: }
$f_{\vec{\xi}}(x, y) = f_{\xi_1}(x)f_{\xi_2}(y)$
З цього випливає, що:

\begin{enumerate}
    \item $f_{\xi_1}(x/y) = f_{\xi_1}(x)$
    
    $f_{\xi_2}(y/x) = f_{\xi_2}(y)$
    \item $E(\xi_1 / \xi_2 = y) = E\xi_1$
    
    $E(\xi_2 / \xi_1 = x) = E\xi_2$
\end{enumerate}

Для n-вимірного випадку ($n > 2$):

$\forall i,j$: $F_{\xi_i\xi_j}(x_i, x_j) = F_{\xi_i}(x_i)F_{\xi_j}(x_j)$ - 
попарна незалежність.

\begin{definition}
    Координати $\xi_1, \xi_2, ..., \xi_n$ є незалежними у сукупності 
    тоді і тільки тоді коли $F_{\vec{\xi}}(\vec{x}) = 
    \prod\limits_{k=1}^n F_{\xi_k}(x_k)$.
\end{definition}
\begin{equation*}
    f_{\vec{\xi}}(\vec{x}) = \frac{\partial^nF_{\vec{\xi}}(\vec{x})}
    {\partial x_1...\partial x_n} \Rightarrow 
    f_{\vec{\xi}}(\vec{x}) = \prod\limits_{k=1}^n f_{\xi_k}(x_k)
\end{equation*}

\begin{proposition}
    Незалежність у сукупності $\Rightarrow$ попарна незалежність
    (зворотнє, як правило, невірно).
\end{proposition}
\begin{proof}
    $F_{\xi_i\xi_j}(x_i, x_j) = F_{\vec{\xi}}(+\infty, ..., +\infty, x_i, 
    +\infty, ..., +\infty, x_j, +\infty, ..., +\infty) = $

    $= F_{\xi_1}(+\infty)...F_{\xi_{i-1}}(+\infty)F_{\xi_i}(x_i)
    F_{\xi_{i+1}}(+\infty)...F_{\xi_{j-1}}(+\infty)F_{\xi_{j}}(x_j)
    F_{\xi_{j+1}}(+\infty)F_{\xi_n}(+\infty) = $
    
    $= F_{\xi_i}(x_i)F_{\xi_j}(x_j)$
\end{proof}

Таким чином, з незалежності у сукупності випливає некорельованість.