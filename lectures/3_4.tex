% !TEX root = ../main.tex

\section{Умовні закони розподілу ВВ}

Розглянемо випадок $n = 2$, $\vec{\xi} = \left(\xi_1, \xi_2\right)^T$

\begin{definition}
    \emph{Умовним законом розподілу} $\xi_1$ називається закон розподілу 
    $\xi_1$ за умови того, що $\xi_2$ набула відповідного значення (ДВВ) 
    або потрапила в деякий проміжок (НВВ) (для $\xi_2$ - аналогічно).
\end{definition}

\begin{definition}
    Універсальним умовним законом розподілу ВВ є \emph{умовна функція 
    розподілу}:
    \begin{equation*}
        F_{\xi_1}(x/y) = P({\xi_1 < x}/{\xi_2 < y}) = 
        \frac{P\left\{\xi_1 < x, \xi_2 < y\right\}}
        {P\left\{\xi_2 < y\right\}} = 
        \frac{F_{\vec{\xi}}(x, y)}{F_{\xi_2}(y)}
    \end{equation*}
\end{definition}
\begin{remark}[1]
    Для $\xi_2$ формулювання та формули абсолютно аналогічні.
\end{remark}
\begin{remark}[2]
    З визначення випливає, що 
    $F_{\vec{\xi}}(x,y) = F_{\xi_1}(x)F_{\xi_2}(y/x)$
    та 
    $F_{\vec{\xi}}(x,y) = F_{\xi_2}(y)F_{\xi_1}(x/y)$.
\end{remark}