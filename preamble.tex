%%% Работа с языком
%\usepackage{cmap}					    % поиск в PDF				    % русские буквы в формулах
\usepackage[T2A]{fontenc}			% кодировка
\usepackage[utf8]{inputenc}             % кодировка исходного текста
\usepackage[english,ukrainian]{babel}	% локализация и переносы	    
\usepackage{indentfirst}

\usepackage[a4paper, top=25mm, bottom=25mm, left=30mm, right=30mm]{geometry}

\usepackage{amsmath,amsfonts,amssymb,amsthm,mathtools} % AMS

\usepackage{multicol} % Несколько колонок

%%%% Теоремы, определения и т.д. и т.п.
\theoremstyle{plain} % Это стиль по умолчанию, его можно не переопределять.
\newtheorem{theorem}{Теорема}[section]
\newtheorem{proposition}[theorem]{Твердження}
 
\theoremstyle{definition} % "Определение"
\newtheorem{definition}{Визначення}[section]
\newtheorem{example}{Приклад}[section]

\theoremstyle{remark}
\newtheorem*{remark}{Зауваження}


%%%% Картинки
\usepackage{graphicx}
\graphicspath{{pictures/}}
\DeclareGraphicsExtensions{.pdf,.png,.jpg}