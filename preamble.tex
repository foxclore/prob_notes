%%% Работа с языком
%\usepackage{cmap}					    % поиск в PDF				    % русские буквы в формулах
\usepackage[T2A]{fontenc}			% кодировка
\usepackage[utf8]{inputenc}             % кодировка исходного текста
\usepackage[english,ukrainian]{babel}	% локализация и переносы	    
\usepackage{indentfirst}

\usepackage[a4paper, top=25mm, bottom=25mm, left=30mm, right=30mm]{geometry}

\usepackage{amsmath,amsfonts,amssymb,amsthm,mathtools} % AMS

\usepackage{makecell}% Прикольная настройка ячеек в tabular 
%http://ctan.math.utah.edu/ctan/tex-archive/macros/latex/contrib/makecell/makecell-rus.pdf

\usepackage{diagbox}
% Разделение клеток tabular по диагонали

\usepackage{multicol} % Несколько колонок
\usepackage{wrapfig} % Картинки посреди текста
\usepackage{chngcntr} % для нумерации формул
\usepackage{enumitem} % чтоб можно было делать римскую нумерацию
\usepackage{mdwlist}
\usepackage{adjustbox}
\setlist[enumerate]{nosep}

%%%% Теоремы, определения и т.д. и т.п.
\theoremstyle{plain} % Это стиль по умолчанию, его можно не переопределять.
\newtheorem{theorem}{Теорема}
\newtheorem{proposition}{Твердження}
\renewcommand{\qedsymbol}{$\blacktriangle$}
 
\theoremstyle{definition} % "Определение"
\newtheorem{definition}{Означення}[section]
\newtheorem*{definition*}{Означення}
\newtheorem*{example}{Приклад}

\theoremstyle{remark}
\newtheorem*{remark}{Зауваження}
\newtheorem*{exercise}{Вправа}

\counterwithout{equation}{chapter} % нумерация формул
\counterwithin*{equation}{section}

%%%% Картинки
\usepackage{graphicx}
\usepackage{tikz}
\usetikzlibrary{patterns}
\usetikzlibrary{shapes.geometric}
\usetikzlibrary{arrows.meta}
\usetikzlibrary{shapes.geometric}
\graphicspath{{pictures/}}
\DeclareGraphicsExtensions{.pdf,.png,.jpg}